
% Written in 1983 by Ken Kahn and Mats Carlsson.
% Some macros Copyright (C) 1986 by LMI Inc.  All rights reserved.

% I want 12pt as the basic default size.

\font\twlrm  = amr10    scaled \magstep1 % roman  
\font\twlmi  = ammi10   scaled \magstep1 % math italic
\font\twlsy  = amsy10   scaled \magstep1 % math symbols

\font\egtrm  = amr8               % roman
\font\egtmi  = ammi8              % math italic
\font\egtsy  = amsy8              % math symbols

\font\sixrm  = amr6               % roman
\font\sixmi  = ammi6              % math italic
\font\sixsy  = amsy6              % math symbols

\textfont0\twlrm \scriptfont0\egtrm \scriptscriptfont0\sixrm
\textfont1\twlmi \scriptfont1\egtmi \scriptscriptfont1\sixmi
\textfont2\twlsy \scriptfont2\egtsy \scriptscriptfont2\sixsy
\textfont3\tenex \scriptfont3\tenex \scriptscriptfont3\tenex

\baselineskip 14.5pt

% Customize xitem.

\def\mykindex{\smallbreak \parsearg\mykindexx}
\def\mykindexx #1{\dosubind {kw}{#1}{for {\bf \lastfunction}}}

% Handy macros for DEC-10 Prolog built-ins.

\def\Bgg{$\gg$}
\def\Bll{$\ll$}

% \defpredicate.

\def\defpredicateheader #1#2{\doind {pr}{#1}% Make entry in predicate index
\gdef\lastfunction{#1}%
\dosetq {#1-pred}{page}%
\begingroup\defname {#1}{Predicate}%
\defunargs {#2}\endgroup %
}

\def\defpredicate{%
\defparsebody\Edefpredicate\defpredicatex\defpredicateheader}
\def\defpredicatex #1 {\errmessage{@defpredicatex in invalid context}}

% \defineoption.

\def\defineoptionheader #1#2{\dosubind {kw}{#1}{for {\bf define-predicate}}
\gdef\lastfunction{#1}%
\dosetq {#1-defpredopt}{page}%
\begingroup\defname {#1}{Define-Predicate Option}%
\defunargs {#2}\endgroup %
}

\def\defineoption{%
\defparsebody\Edefineoption\defineoptionx\defineoptionheader}
\def\defineoptionx #1 {\errmessage{@defineoptionx in invalid context}}

% \deftoplevel.

\def\windex {\woindex}
\def\deftoplevelheader #1#2{\doind {wo}{#1}% Make entry in world index
\gdef\lastfunction{#1}%
\dosetq {#1-toplevel}{page}%
\begingroup\defname {#1}{Top Level}%
\defunargs {#2}\endgroup %
}

\def\deftoplevel{%
\defparsebody\Edeftoplevel\deftoplevelx\deftoplevelheader}
\def\deftoplevelx #1 {\errmessage{@deftoplevelx in invalid context}}

%  Macro to print out Month, Day and Year
%  Adapted from TeXBook \today macro

\def\datestamp{\ifcase\month\or
  January\or February\or March\or April\or May\or June\or 
  July\or August\or September\or October\or November\or December\fi
  \space\number\day, \number\year}

%  Similar macro; adds time of day, too.

\def\timestamp{\time\kern5pt\datestamp}

%  Macros for accessing useful fonts

\def\dfn{\parsearg\iF} % font for newly defined concepts
\def\code{\parsearg\codeX}
\def\codeX #1{{\li #1}} % font for Lisp code

\def\keyF#1{{\tt #1}}
\def\key{\parsearg\keyF}
\font\tteight=amtt8				       % MC
\font\ittten=amtt10				       % MC, for now.
\def\ittF#1{{\ittten #1}} % italic tt font -- used for representing italic 
                          % displays on LISPM screen
\def\itt{\parsearg\ittF}
\def\zfF#1{{\li #1}} % Font for ZMail fields
\def\zf{\parsearg\zfF}


%% Here's how to get my old PATBO macroes to have delimiters.

% \def\I{\parsearg\iF} \let\i=\I \def\iF#1{{\sl #1}}

%  Macros to make it easier to put Zmacs control sequences in documentation
%  The macros do automatic uppercasing of their arguments

\def\ctrlF#1{{\keyF {CTRL-\uppercase{#1}}}}
\def\metaF#1{{\keyF {META-\uppercase{#1}}}}
\def\hyperF#1{{\keyF {HYPER-\uppercase{#1}}}}
\def\superF#1{{\keyF {SUPER-\uppercase{#1}}}}
\def\ctrlxF#1{{\keyF {CTRL-X\tie \uppercase{#1}}}}
\def\cctrlxF#1{{\keyF {CTRL-X\tie CTRL-\uppercase{#1}}}}
\def\metaxF#1{{\keyF {META-X\tie #1}}}
\def\cmF#1{{\keyF {CTRL-META-\uppercase{#1}}}}
\def\cmxF#1{{\keyF {CTRL-META-X\tie #1}}}
\def\chF#1{{\keyF {CTRL-HYPER-\uppercase{#1}}}}
\def\hmF#1{{\keyF {HYPER-META-\uppercase{#1}}}}
\def\greekF#1{{\keyF {GREEK-\uppercase{#1}}}}

\def\ctrl{\parsearg\ctrlF}
\def\meta{\parsearg\metaF}
\def\hyper{\parsearg\hyperF}
\def\super{\parsearg\superF}
\def\ctrlx{\parsearg\ctrlxF}
\def\cctrlx{\parsearg\cctrlxF}
\def\metax{\parsearg\metaxF}
\def\cm{\parsearg\cmF}
\def\cmx{\parsearg\cmxF}
\def\ch{\parsearg\chF}
\def\hm{\parsearg\hmF}
\def\greek{\parsearg\ctrlF}

% Similar macro for X in ZMail

\def\xF#1{{\keyF {X\tie #1}}}
\def\x{\parsearg\xF}

%  Macros to talk about ``the CTRL key'' or ``META-X'' commands''

\def\ckey{{\keyF CTRL}}
\def\mkey{{\keyF META}}
\def\cxkey{{\keyF CTRL-X}}
\def\mxkey{{\keyF META-X}}
\def\cmxkey{{\keyF CTRL-META-X}}
\def\hkey{{\keyF HYPER}}
\def\skey{{\keyF SUPER}}
\def\gkey{{\keyF GREEK}}
\def\tkey{{\keyF TOP}}

%  Macro used to draw boxes around arbitrary strings of text
%  Used to represent mouse sensitive items

\def\testboxF#1{\leavevmode\thinspace\hbox{\vrule\vtop{\vbox{\hrule\kern1pt
	   \hbox{\vphantom{\tt/}\thinspace{\keyF {#1}}\thinspace}}
	\kern1pt\hrule}\vrule}\thinspace} % control sequence token

\def\testbox{\parsearg\testboxF}

%  More mnemonic ways to access this macro

\let\mousebox=\testbox
\let\mouseditem=\testbox
\let\mousesensitive=\testbox

%  Macros for representing mouse buttons in documentation

%  \mouse gives an upside down U; this represents ANY mouse button

\def\mouse{\smash{\raise.127in\hbox{{\tenex \char'134}}}}

%  These macros print single mouse clicks on particular keys
%  A single character is put in the upside-down U
%  R is put in for the right mouse button
%  L for the left mouse button
%  M for the middle mouse button

\def\mouseright{\kern2pt{\keyF R}\kern-8.25pt\smash{\raise.127in\hbox{{\tenex %
	\char'134}}}}

\def\mouseleft{\kern2pt{\keyF L}\kern-8.25pt\smash{\raise.127in\hbox{{\tenex %
        \char'134}}}}

\def\mousemiddle{\kern2pt{\keyF M}\kern-8.25pt\smash{\raise.127in\hbox{{\tenex %
	\char'134}}}}

%  These macros print double mouse clicks
%  R, L, M have the same meaning as above
%  2 is added below the letter in the upside down U

\def\mousetwo{{\tteight 2}}

\def\mousetimes{$\times$}

\def\greq{$\geq$}

\def\puparrow{$\uparrow$}

\def\pdownarrow{$\downarrow$}

\def\pdash{$-$}

\def\noop{}

\def\mouserighttwice{\kern2pt{\keyF R}\kern-4.5pt %
     \smash{\lower6pt %
      \hbox{{\tteight 2}}}\kern-8pt\smash{\raise.127in\hbox{{\tenex %
	\char'134}}}}

\def\mouselefttwice{\kern2pt{\keyF L}\kern-4.5pt %
     \smash{\lower6pt %
      \hbox{{\tteight 2}}}\kern-8pt\smash{\raise.127in\hbox{{\tenex %
	\char'134}}}}

\def\mousemiddletwice{\kern2pt{\keyF M}\kern-4.5pt %
     \smash{\lower6pt %
      \hbox{{\tteight 2}}}\kern-8pt\smash{\raise.127in\hbox{{\tenex %
	\char'134}}}}

\def\mouseonF#1{{\mouse}\testboxF{#1}}

\def\mouserightonF#1{{\mouseright}\testboxF{#1}}

\def\mouseleftonF#1{{\mouseleft}\testboxF{#1}}

\def\mousemiddleonF#1{{\mousemiddle}\testboxF{#1}}

\def\mouserighttwiceonF#1{{\mouserighttwice}\testboxF{#1}}

\def\mouselefttwiceonF#1{{\mouselefttwice}\testboxF{#1}}

\def\mousemiddletwiceonF#1{{\mousemiddletwice}\testboxF{#1}}

\def\mouseon{\parsearg\mouseonF}
\def\mouserighton{\parsearg\mouserightonF}
\def\mouselefton{\parsearg\mouseleftonF}
\def\mousemiddleon{\parsearg\mousemiddleonF}
\def\mouserighttwiceon{\parsearg\mouserighttwiceonF}
\def\mouselefttwiceon{\parsearg\mouselefttwiceonF}
\def\mousemiddletwiceon{\parsearg\mousemiddletwiceonF}

%  \termkey draws a lozenge around a string of words
%  It is meant to be used to represent function keys

\def\termkey#1{$\langle\kern-1pt\overline{\underline{\lower0.7pt\hbox{\keyF
    {#1}}\lower.29ex\hbox{}}\raise6pt\hbox{}}\kern-0.5pt\rangle$}

\def\lozenge{\termkey{\kern36pt\vbox{\kern1.4ex}}}
\let\lozengestringF=\termkey
\def\lozengestring{\parsearg\lozengestringF}

%  Hand-tuned predefined versions of function keys
%  Space at left and right edges has been adjusted 
%  To make them come out looking right

\def\abort{\termkey{ABORT}}
\def\altmode{\termkey{ALT\tie MODE}}
\def\breakkey{\termkey{BREAK}}
\def\call{\termkey{CALL}}
\def\clrput{\termkey{CLEAR\tie INPUT}}
\def\clrscrn{\termkey{CLEAR\tie SCREEN}}
\def\delete{\termkey{DELETE}}
\def\endkey{\termkey{\kern1pt{}END}}
\def\help{\termkey{\kern1pt{}HELP}}
\def\hldput{\termkey{HOLD\tie OUTPUT}}
\def\linekey{\termkey{LINE}}
\def\macro{\termkey{MACRO}}
\def\modelock{\termkey{MODE\tie LOCK}}
\def\network{\termkey{NETWORK}}
\def\overstrike{\termkey{OVER\tie STRIKE}}
\def\quotekey{\termkey{\smash{Q}UOTE}}
\def\resume{\termkey{RESUME}}
\def\return{\termkey{RETURN}}
\def\rubout{\termkey{RUB\tie OUT}}
\def\status{\termkey{\kern0.5pt{}STATUS}}
\def\stopput{\termkey{STOP\tie OUTPUT}}
\def\system{\termkey{SYSTEM}}
\def\tab{\termkey{TAB}}
\def\terminal{\termkey{\kern0.8pt{}TERMINAL}}

\def\spacekey{\termkey{SPACE}}

% Z-29 keys

\def\backspace{\termkey{BACK\tie SPACE}}
\def\enter{\termkey{ENTER}}
\def\erase{\termkey{ERASE}}
\def\escape{\termkey{ESC}}
\def\home{\termkey{HOME}}
\def\linefeed{\termkey{LINE\tie FEED}}
\def\reset{\termkey{RESET}}
\def\scroll{\termkey{SCROLL}}
\def\setup{\termkey{SET\tie UP}}

% Some more key icons for LM-Prolog User Manual.

\def\romani{\termkey{I}}
\def\romanii{\termkey{II}}
\def\romaniii{\termkey{III}}
\def\romaniv{\termkey{IV}}
\def\handup{\termkey{HAND UP}}
\def\handdown{\termkey{HAND DOWN}}
\def\trianglekey{\termkey{ $\triangle$ }}
\def\circlekey{\termkey{ $\circ$ }}
\def\squarekey{\termkey{ $\ensquare$ }}
\def\ensquare{\hbox{\vrule%
		    \vbox{\hrule width .5em\vskip .5em\hrule width .5em}%
		    \vrule}}

\def\csh{\parsearg\cshF}
\def\cshF#1{{\keyF {CTRL-SHIFT-\uppercase{#1}}}}

% Function keys that take arguments

\def\helponF#1{{\help}\tie {\keyF #1}}
\def\helpon{\parsearg\helponF}
\def\netonF#1{{\network}\tie {\keyF #1}}
\def\neton{\parsearg\netonF}
\def\sysonF#1{{\system}\tie {\keyF #1}}
\def\syson{\parsearg\sysonF}
\def\sysctrlonF#1{{\system}\tie {\keyF {CTRL-#1}}}
\def\sysctrlon{\parsearg\sysctrlonF}
\def\isysonF#1{{\system}\tie {\iiF {#1}}}
\def\isyson{\parsearg\isysonF}
\def\isysctrlonF#1{{\system}\tie {\keyF {CTRL-}}{\iiF {#1}}}
\def\isysctrlon{\parsearg\isysctrlonF}
\def\termonF#1{{\terminal}\tie {\keyF #1}}
\def\termon{\parsearg\termonF}
\def\itermonF#1{{\terminal}\tie {\iiF {#1}}}
\def\itermon{\parsearg\itermonF}

%  Internal macros used to build \choicebox

\def\side{\vrule height8pt depth2pt width3pt}
\def\base{\vrule height1pt depth2pt width4pt}

%  \choicebox draws a square box with thick borders, like the
%  choice boxes on the LISPM

\def\choicebox{\side\base\kern-4pt\raise7.1pt\hbox{\base}\side}

%  \choiceon creates a label for a \choicebox (in \keyF font)
%  followed by a \choicebox

\def\choiceonF#1{{\keyF {#1}}\tie {\choicebox}}
\def\choiceon{\parsearg\choiceonF}

%  Internal macros used to build mouse corners

\def\mside{\vrule height8pt depth0pt width0.4pt}
\def\mbase{\vrule height0.4pt depth0pt width8pt}

%  Macros for mouse corners

\def\leftmousecorner{\mside\raise7.5pt\hbox{\mbase}}
\def\rightmousecorner{\mbase\mside}

%  Miscellaneous special characters

\def\puparrow{$\uparrow$}
\def\pdownarrow{$\downarrow$}
\def\prightarrow{$\rightarrow$}
\def\pleftarrow{$\leftarrow$}
\def\pleftrightarrow{$\leftrightarrow$}
\def\pwedge{$\wedge$}
\def\pvarepsilon{$\varepsilon$}
\def\plambda{$\lambda$}
\def\pgeq{$\geq$}
\def\pleq{$\leq$}
\def\pneq{$\not =$}

\message{Basics,}
\chardef\other=12
\parskip=1.5pt
% pd \advance\topskip by 1.2cm
% Compensate for losing imagen.

% pd \advance \hoffset 1in
% pd \advance \voffset 1in

\hyphenation{ap-pen-dix zeta-lisp}

% Copyright notice
\def\Xcopyrightdate{\number\year}

% Margin to add to right of even pages, to left of odd pages.
\newdimen \bindingoffset  \bindingoffset=0pt
\newdimen \normaloffset   \normaloffset=\hoffset
\newdimen\pagewidth \newdimen\pageheight
\pagewidth=\hsize \pageheight=\vsize

% \onepageout takes a vbox as an argument.  Note that \pagecontents
% does insertions itself, but you have to call it yourself.
\chardef\PAGE=255  \output={\onepageout{\pagecontents\PAGE}}
\def\onepageout#1{\hoffset=\normaloffset
\ifodd\pageno  \advance\hoffset by \bindingoffset
\else \advance\hoffset by -\bindingoffset\fi
\shipout\vbox{{\let\hsize=\pagewidth \makeheadline} \pagebody{#1}%
 {\let\hsize=\pagewidth \makefootline}}
\advancepageno \ifnum\outputpenalty>-20000 \else\dosupereject\fi}
\def\pagebody#1{\vbox to\pageheight{\boxmaxdepth=\maxdepth #1}}
{\catcode`\@ =11
\gdef\pagecontents#1{\ifvoid\topins\else\unvbox\topins\fi
\dimen@=\dp#1 \unvbox#1
\ifvoid\footins\else\vskip\skip\footins\footnoterule \unvbox\footins\fi
\ifr@ggedbottom \kern-\dimen@ \vfil \fi}
}

% Parse an argument, then pass it to #1.
% The argument can be delimited with [...] or with "..."
% or it can be a whole line.
% #1 should be a macro which expects
% an ordinary undelimited TeX argument.

\def\parsearg #1{\let\next=#1\begingroup\obeylines\futurelet\temp\parseargx}

\def\parseargx{%
\ifx \obeyedspace\temp \aftergroup\parseargdiscardspace \else%
\ifx [\temp \aftergroup\parseargbracket\else%
\ifx (\temp \aftergroup\parseargparen\else%
\ifx \activedoublequote\temp \aftergroup\parseargdoublequote\else%
\ifx \mylbrace\temp \aftergroup\parseargbrace\else%
\aftergroup \parseargline %
\fi\fi\fi\fi\fi \endgroup}

{\obeyspaces %
\gdef\parseargdiscardspace {\begingroup\obeylines\futurelet\temp\parseargx}}

\gdef\obeyedspace{\ }

\def\parseargbracket [#1]{\next {#1}} \def\parseargparen (#1){\next {#1}}
{\catcode `\"=\active
\gdef\parseargdoublequote "#1"{\next {#1}}}

{\catcode `\[=1 \catcode `\]=2 \catcode `\}=\active
\catcode `\{=\active
\gdef\parseargbrace {#1}[\next [#1]]]

\def\parseargline{\begingroup \obeylines \parsearglinex}
{\obeylines %
\gdef\parsearglinex #1^^M{\endgroup \next {#1}}}

%% These are used to keep @begin/@end levels from running away
%% Call \inENV within environments (after a \begingroup)
\newif\ifENV \ENVfalse \def\inENV{\ifENV\relax\else\ENVtrue\fi}
\def\ENVcheck{%
\ifENV\errmessage{Still within an environment.  Type Return to continue.}
\endgroup\fi} % This is not perfect, but it should reduce lossage

% @begin foo  is the same as @foo, for now.
\newhelp\EMsimple{Type <Return> to continue}

\outer\def\begin{\parsearg\beginxxx}

\def\beginxxx #1{%
\expandafter\ifx\csname #1\endcsname\relax
{\errhelp=\EMsimple \errmessage{Undefined command @begin #1}}\else
\csname #1\endcsname\fi}

%% @end foo executes the definition of \Efoo.
%% foo can be delimited by doublequotes or brackets.
\let\ptexend = \end

\def\end{\parsearg\endxxx}

\def\endxxx #1{%
\expandafter\ifx\csname E#1\endcsname\relax
\expandafter\ifx\csname #1\endcsname\relax
\errmessage{Undefined command @end #1}\else
\errorE{#1}\fi\fi
\csname E#1\endcsname}
\def\errorE#1{
{\errhelp=\EMsimple \errmessage{@end #1 not within #1 environment}}}

% Simple aliases that make some plain tex constructs available.

\let\nopara=\noindent

\let\ptexnobreak=\nobreak
\def\nobreak{\par\penalty 10000}

% Single-spacing is done by various environments.

\newskip\singlespaceskip \singlespaceskip = \baselineskip
\def\singlespace{\baselineskip=\singlespaceskip}

% @@ prints an @ -- for Scribe compatibility.
% Kludge this until the fonts are right (grr).
\def\@{{\sf @}}

% @: is a Scribe command to cause end-of-sentence whitespace.
\def\:{\spacefactor=3000 }

% @* forces a line break.
\let\ptexstar=\* \def\*{\hfil\break}

% @. is an abbreviation period.
\let\ptexdot=\. \def\.{.\spacefactor=1000 }

% @# leaves space for a special character.
\let\ptexnumsign=\# \def\#{\write19{(@##)}\hbox to 0.7em{\hfil}}

% @w prevents a word break
\def\w{\parsearg\atomx} \def\atomx #1{\hbox{#1}}

% Save the essence of & for tabular environments as @\ for BoTeX
% Say @settabs 4 @columns, then @<col0@\col1@\col2@\col3@cr
\let\\=& \let\<=\+
% @+ raises its argument, @- lowers.
\let\ptexplus=\+
\def\+{\parsearg\plusxxx} \def\plusxxx #1{\raise 1ex\hbox{#1}}

\let\ptexminus=\-
\def\-{\parsearg\minusxxx} \def\minusxxx #1{\lower 1ex\hbox{#1}}

% @group ... @end group  forces ... to be all on one page.

\def\group{\begingroup% \inENV ???
\def \Egroup{\egroup\endgroup}
\vbox\bgroup}

% @br   forces paragraph break

\let\br = \par

% @page    forces the start of a new page

\def\page{\par\vfill\supereject}

% @exdent n text....
% outputs text on separate line in roman font, starting at standard page margin
% The argument n is ignored.  This is most likely to be right
% for the ways @exdent actually appears in Bolio files.

\def\exdent{\errmessage{@exdent in filled text}}
  % @lisp, etc, define \exdent locally from \internalexdent

{\obeyspaces
\gdef\internalexdent #1 {\parsearg\exdentzzz}}

\def\exdentzzz #1{{\advance \leftskip by -\lispnarrowing
\advance \hsize by -\leftskip
\advance \hsize by -\rightskip
\leftline{{\rm#1}}}}

% @include file    insert text of that file as input.

\def\include{\parsearg\includexxx}

\def\includexxx #1{{\def\thisfile{#1}\input #1
}}

\def\thisfile{}

% @need space-in-mils
% forces a page break if there is not space-in-mils remaining.

\newdimen\mil  \mil=0.001in

\def\need{\parsearg\needx}

\def\needx #1{\par %
\begingroup %
\dimen0=\pagetotal %
\advance \dimen0 by #1\mil %
\ifdim \dimen0>\pagegoal \vfill\eject \fi %
\endgroup}

% @setpageheight and @setpagewidth
% These are no longer really necessary
% since you can now do @hsize = or @vsize =

\def\setpagewidth #1 {\global\hsize #1}
\def\setpageheight #1 {\global\vsize #1}

% @center line   outputs that line, centered

\def\center{\parsearg\centerxxx}

\def\centerxxx #1{{\advance\hsize by -\leftskip
\advance\hsize by -\rightskip
\centerline{#1}}}

% @sp n   outputs n lines of vertical space

\def\sp{\parsearg\spacexxx}

\def\spacexxx #1{\par \vskip #1\baselineskip}

% @comment ...line which is ignored...
% @c is the same as @comment
% @ignore ... @end ignore  is another way to write a comment

\def\comment{\parsearg \commentxxx}

\def\commentxxx #1{}

\let\c=\comment

\long\def\ignore #1\end ignore{}

% Document-version conditionals
\message{conditionals,}

% @setflag foo   sets flag "foo"
% @clearflag foo   clears flag "foo"
% @defaultsetflag foo    sets flag if never been explicitly set or cleared
% @ifset [foo]...@end if  includes body if foo is set
% @ifclear [foo]...@end if   includes body if foo is not set.
% A flag is initially clear by default.

% A flag is represented by the value of a control sequence
% whose name is F followed by the flag name.
% If its definition is \relax, it has never been explicitly set or cleared.
% This counts as "clear" for everything except @defaultsetflag.
% It is the initial state.
% Flags explicitly set or cleared have these definitions:
\def\flagtrue{true}
\def\flagfalse{false}

\def\setflag{\parsearg\setflagxxx}
\def\setflagxxx #1{\expandafter\let\csname F#1\endcsname=\flagtrue}

\def\clearflag{\parsearg\clearflagxxx}
\def\clearflagxxx #1{\expandafter\let\csname F#1\endcsname=\flagfalse}

\def\defaultsetflag{\parsearg\defaultsetflagxxx}
\def\defaultsetflagxxx #1{%
\expandafter\ifx\csname F#1\endcsname\relax
\expandafter\let\csname F#1\endcsname=\flagtrue
\fi}

\def\ifset{\parsearg\ifsetxxx}

\def\ifsetxxx #1{%
\expandafter\ifx\csname F#1\endcsname\flagtrue \let\next=\relax \else
\let\next=\iffails \fi \next}

\def\ifnotset{\parsearg\ifnotsetxxx}
\def\ifclear{\parsearg\ifnotsetxxx}

\def\ifnotsetxxx #1{%
\expandafter\ifx\csname F#1\endcsname\flagtrue \let\next=\iffails \else
\let\next=\relax \fi \next}

\def\iffails #1\end if{}
\def\Eif{}

\message{fonts,}
% Font-change commands.
\font\btwelve=ambx10 at 12pt
%% Try out Computer Modern fonts at a little bigger than \magstephalf
\font\tenrm=amr10 scaled 1095
\font\tentt=amtt10 scaled \magstephalf
\font\tenbf=amb10 scaled 1095
\font\tenit=amti10 scaled 1100
\font\tensl=amsl10 scaled 1095
% \font\tensf=amss10 scaled 1100 
\font\tensf=amss10 scaled 1095
\def\li{\sf}
\font\tensc=amcsc10 scaled \magstephalf

\font\defbf=ambx7 scaled \magstep2
\let\deftt=\tentt
% removed by PD \font\twelvesf=amss12
% Two frobs added by MC:
\font\twelvebf=ambx10 scaled \magstep 1 % 12pt
\font\twelvesf=amss10 scaled \magstep 1 % 12pt

% Font for title
\font\titlerm = ambx10 scaled 2986

% Fonts for indices
\font\indit=ammi9 \font\indrm=amr9
\def\indbf{\indrm} \def\indsl{\indit}
\def\indexfonts{\let\it=\indit \let\sl=\indsl \let\bf=\indbf \let\rm=\indrm}

\font\secrm=amb10 scaled 1440
\font\secit=amti10 scaled 1400
\font\secsl=amsl10 scaled 1400
\let\secbf=\secrm

\font\chaprm=ambx10 scaled 1728
\font\chapit=amti10 scaled 1800
\let\chapbf=\chaprm

\font\ssecrm=amb10 scaled 1200
\font\ssecit=amti10 scaled 1200
\font\ssecsl=amsl10 scaled 1200
\let\ssecbf=\ssecrm

\def\textfonts{\let\rm=\tenrm\let\it=\tenit\let\sl=\tensl\let\bf=\tenbf%
\let\sc=\tensc\let\sf=\tensf}
\def\chapfonts{\let\rm=\chaprm\let\it=\chapit\let\sl=\chapsl\let\bf=\chapbf}
\def\secfonts{\let\rm=\secrm\let\it=\secit\let\sl=\secsl\let\bf=\secbf}
\def\subsecfonts{\let\rm=\ssecrm\let\it=\ssecit\let\sl=\ssecsl\let\bf=\ssecbf}
% Count depth in font-changes, for error checks
\newcount\fontdepth \fontdepth=0

\def\fontspec #1{\endgroup
\if #1$\epsilon$\else \if #1*\fontpop \global\advance\fontdepth by -1 \else
\global\advance\fontdepth by 1 %
\begingroup\let\fontpop=\endgroup\fontselect#1\fi\fi}

%% This is zero-based; things actually start out in 0 (=rm by default) now.
%% Font 9 is always \titlerm for hysterical reasons
\def\nofont{\errmessage{No such font.  Type Return to continue.}}
\def\fontvector#1#2#3#4#5#6#7#8#9{
\def\FMzero{#1} \def\FMone{#2} \def\FMtwo{#3} \def\FMthree{#4} \def\FMfour{#5}
\def\FMfive{#6} \def\FMsix{#7} \def\FMseven{#8} \def\FMeight{#9}}
\fontvector \rm \bf \sl \li \chaprm \chaprm \nofont \tt \nofont
%% Ever hear of AREF ? Actually, \ifcase seems to lose here
\def\fontselect#1{\if #10\FMzero\else\if #11\FMone\else\if#12\FMtwo\else%
\if #13\FMthree\else\if #14\FMfour\else\if #15\FMfive\else\if #16\FMsix\else%
\if #17\FMseven\else\if #18\FMeight\else\if #19\titlerm
\else\errmessage{Illegal font selector (#1)}\fi\fi\fi\fi\fi\fi\fi\fi\fi\fi}

\def\fontpop{\errmessage{extra ^F*}}

\def\levelcheck{\ifnum \fontdepth > 0 %
\errmessage{Unmatched font-changes before this point}\fi\ENVcheck}

\catcode`\^^F=\active
\def\ctlf{\begingroup\catcode`\^^F=\other\fontspec}
\let^^F=\ctlf

%% Add scribe-like font environments, plus @l for inline lisp (usually sans
%% serif) and @ii for TeX italic
\let\ptexb=\b \let\ptexc=\c \let\ptexi=\i \let\ptext=\t
\let\ptexl=\l \let\ptexL=\L
\def\I{\parsearg\iF} \let\i=\I \def\iF#1{{\sl #1}}
\def\R{\parsearg\rF} \let\r=\R \def\rF#1{{\rm #1}}
\def\B{\parsearg\bF} \let\b=\B \def\bF#1{{\bf #1}}
\def\L{\parsearg\lF} \let\l=\L \def\lF#1{{\li #1}}
\def\s{\parsearg\sF}	       \def\sF#1{{\sc #1}}
\def\T{\parsearg\tF} \let\t=\T \def\tF#1{{\tt #1}}
\def\II{\parsearg\iiF} \let\ii=\II \def\iiF#1{{\it #1}}

% ^Q quote next character
\catcode`\^^Q=\active

\def\ctlq{\begingroup
\catcode `\@=\other 
\afterassignment\quotechar\let\nextchar= }

\let^^Q=\ctlq

\def\quotechar{%
\ifx \nextchar\ref $\otimes$\else
\ifx \nextchar\ctlq $\supset$\else
\ifx \nextchar\ctlf $\epsilon$\else
\if \nextchar-$-$\else
\if \nextchar\space\penalty 10000\ \else
\if \nextchar,\penalty 10000\ \else
\nextchar\spacefactor=1000
\fi\fi\fi\fi\fi\fi
\endgroup}

\def\tie{\penalty 10000\ }     % Save plain tex definition of ~.

% Define the random MIT characters

\catcode `\^^@=\active \def^^@{$\cdot$}
\catcode `\=\active \def{$\downarrow$}
\catcode `\=\active \def{$\alpha$}
\catcode `\^^C=\active \def^^C{$\beta$}
\catcode `\^^D=\active \def^^D{$\wedge$}
\catcode `\^^E=\active \def^^E{$\neg$}
\catcode `\^^G=\active \def^^G{$\pi$}
\catcode `\^^H=\active \def^^H{$\lambda$}
\catcode `\=\active \def{$\uparrow$}
\catcode `\^^L=\active \def^^L{$\plusminus$}
\catcode `\=\active \def{$\infty$}
\catcode `\=\active \def{$\partial$}
\catcode `\=\active \def{$\subset$}
\catcode `\=\active \def{$\cap$}
\catcode `\=\active \def{$\cup$}
\catcode `\=\active \def{$\forall$}
\catcode `\=\active \def{$\exists$}
\catcode `\=\active \def{$\rightleftharpoons$}
\catcode `\=\active \def{$\leftarrow$}
\catcode `\=\active \def{$\rightarrow$}
\catcode `\=\active \def{$\neq$}
\catcode `\^^[=\active \def^^[{$\diamondsuit$}
\catcode `\^^\=\active \def^^\{$\leq$}
\catcode `\^^]=\active \def^^]{$\geq$}
\catcode `\^^^=\active \def^^^{$\equiv$}
\catcode `\^^_=\active \def^^_{$\vee$}

% Font for variables

\let\vf=\sl

% Font(s) for functions being defined
\let\defsf=\twelvesf
\def\df{\let\tt=\deftt \dffont}
\def\DEFbf{\global\let\dffont=\defbf}
\def\DEFsf{\global\let\dffont=\defsf}
\DEFbf

\message{page headings,}

%%% Set up page headings and footings.

\let\thispage=\folio

\newtoks \evenheadline    % Token sequence for heading line of even pages
\newtoks \oddheadline     % Token sequence for heading line of odd pages
\newtoks \evenfootline    % Token sequence for footing line of even pages
\newtoks \oddfootline     % Token sequence for footing line of odd pages

% Now make Tex use those variables
\headline={{\textfonts\rm \ifodd\pageno \the\oddheadline \else \the\evenheadline \fi}}
\footline={{\textfonts\rm \ifodd\pageno \the\oddfootline \else \the\evenfootline \fi}}

% Commands to set those variables.
% For example, this is what  @headings on  does
% @evenheading @thistitle|@thispage|@thischapter
% @oddheading @thischapter|@thispage|@thistitle
% @evenfooting @thisfile||
% @oddfooting ||@thisfile

\def\evenheading{\parsearg\evenheadingxxx}
\def\oddheading{\parsearg\oddheadingxxx}
\def\everyheading{\parsearg\everyheadingxxx}

\def\evenfooting{\parsearg\evenfootingxxx}
\def\oddfooting{\parsearg\oddfootingxxx}
\def\everyfooting{\parsearg\everyfootingxxx}

\def\evenheadingxxx #1{\evenheadingyyy #1\|\|\|\|\finish}
\def\evenheadingyyy #1\|#2\|#3\|#4\finish{%
\global\evenheadline={\rlap{\centerline{#2}}\line{#1\hfil#3}}}

\def\oddheadingxxx #1{\oddheadingyyy #1\|\|\|\|\finish}
\def\oddheadingyyy #1\|#2\|#3\|#4\finish{%
\global\oddheadline={\rlap{\centerline{#2}}\line{#1\hfil#3}}}

\def\everyheadingxxx #1{\everyheadingyyy #1\|\|\|\|\finish}
\def\everyheadingyyy #1\|#2\|#3\|#4\finish{%
\global\evenheadline={\rlap{\centerline{#2}}\line{#1\hfil#3}}
\global\oddheadline={\rlap{\centerline{#2}}\line{#1\hfil#3}}}

\def\evenfootingxxx #1{\evenfootingyyy #1\|\|\|\|\finish}
\def\evenfootingyyy #1\|#2\|#3\|#4\finish{%
\global\evenfootline={\rlap{\centerline{#2}}\line{#1\hfil#3}}}

\def\oddfootingxxx #1{\oddfootingyyy #1\|\|\|\|\finish}
\def\oddfootingyyy #1\|#2\|#3\|#4\finish{%
\global\oddfootline={\rlap{\centerline{#2}}\line{#1\hfil#3}}}

\def\everyfootingxxx #1{\everyfootingyyy #1\|\|\|\|\finish}
\def\everyfootingyyy #1\|#2\|#3\|#4\finish{%
\global\evenfootline={\rlap{\centerline{#2}}\line{#1\hfil#3}}
\global\oddfootline={\rlap{\centerline{#2}}\line{#1\hfil#3}}}

% @headings on   turns them on.
% @headings off  turns them off.
% By default, they are off.

\def\headings #1 {\csname HEADINGS#1\endcsname}

\def\HEADINGSoff{
\global\evenheadline={\hfil} \global\evenfootline={\hfil}
\global\oddheadline={\hfil} \global\oddfootline={\hfil}}
\HEADINGSoff
% When we turn headings on, set the page number to 1,
% Put current file name in lower left corner,
% Put chapter name in upper right corner, document title in upper left,
% and page number in the center.
\def\HEADINGSon{
\global\pageno=1
\global\evenfootline={\thisfile\hfil}
\global\oddfootline={\thisfile\hfil}
\global\evenheadline={\rlap{\centerline{\folio}}\line{\thistitle\hfil\thischapter}}
\global\oddheadline={\rlap{\centerline{\folio}}\line{\thistitle\hfil\thischapter}}
}

% Subroutines used in generating headings
\def\today{\ifcase\month\or
January\or February\or March\or April\or May\or June\or
July\or August\or September\or October\or November\or December\fi
\space\number\day, \number\year}

% @settitle line...  specifies the title of the document, for headings
% It generates no output of its own

\def\thistitle{No Title}
\def\settitle{\parsearg\settitlezz}
\def\settitlezz #1{\gdef\thistitle{#1}}

\message{tables,}

% Tables -- @table, @ftable, @item(x), @kitem(x), @xitem(x).

% default indentation of table text
\newdimen\tableindent \tableindent=.8in
% default indentation of @itemize and @enumerate text
\newdimen\itemindent  \itemindent=.8in
% margin between end of table item and start of table text.
\newdimen\itemmargin  \itemmargin=.1in

% used internally for \itemindent minus \itemmargin
\newdimen\itemmax

% Note @table and @ftable define @item, @itemx, etc., with these defs.
% They also define \itemindex
% to index the item name in whatever manner is desired (perhaps none).

\def\internalBitem{\smallbreak \parsearg\itemzzz}
\def\internalBitemx{\par \parsearg\itemzzz}

\def\internalBxitem "#1"{\def\xitemsubtopix{#1} \smallbreak \parsearg\xitemzzz}
\def\internalBxitemx "#1"{\def\xitemsubtopix{#1} \par \parsearg\xitemzzz}

\def\internalBkitem{\smallbreak \parsearg\kitemzzz}
\def\internalBkitemx{\par \parsearg\kitemzzz}

\def\kitemzzz #1{\dosubind {kw}{#1}{for {\bf \lastfunction}}\itemzzz {#1}}

\def\xitemzzz #1{\dosubind {kw}{#1}{for {\bf \xitemsubtopic}}\itemzzz {#1}}

\def\itemzzz #1{\begingroup %
\advance \hsize by -\rightskip %
\advance \hsize by -\leftskip %
\setbox0=\hbox{{\itemfont #1}}%
\itemindex{#1}%
\parskip=0in %
\noindent\vadjust{\penalty 800}%
\ifdim \wd0>\itemmax %
\hbox to \hsize{\hskip -\tableindent\box0\hss}\ %
\else %
\hbox to 0pt{\hskip -\tableindent\box0\hss}%
\fi %
\endgroup %
}

\def\item{\errmessage{@item while not in a table}}
\def\itemx{\errmessage{@itemx while not in a table}}
\def\kitem{\errmessage{@kitem while not in a table}}
\def\kitemx{\errmessage{@kitemx while not in a table}}
\def\xitem{\errmessage{@xitem while not in a table}}
\def\xitemx{\errmessage{@xitemx while not in a table}}

%% Contains a kludge to get @end[description] to work
\def\description{\tablez{\dontindex}{1}{}{}{}{}}

\def\table{\begingroup\inENV\obeylines\obeyspaces\tablex}
{\obeylines\obeyspaces%
\gdef\tablex #1^^M{%
\tabley\dontindex#1        \endtabley}}

\def\ftable{\begingroup\inENV\obeylines\obeyspaces\ftablex}
{\obeylines\obeyspaces%
\gdef\ftablex #1^^M{%
\tabley\fnitemindex#1        \endtabley}}

\def\dontindex #1{}
\def\fnitemindex #1{\doind {fn}{#1}}%

{\obeyspaces %
\gdef\tabley#1#2 #3 #4 #5 #6 #7\endtabley{\endgroup%
\tablez{#1}{#2}{#3}{#4}{#5}{#6}}}

\def\tablez #1#2#3#4#5#6{%
\aboveenvbreak %
\begingroup %
\def\Edescription{\Etable}% Neccessary kludge.
\let\itemindex=#1%
\ifnum 0#3>0 \advance \leftskip by #3\mil \fi %
\ifnum 0#4>0 \tableindent=#4\mil \fi %
\ifnum 0#5>0 \advance \rightskip by #5\mil \fi %
\let\itemfont=\bf %
\edef\itemfont{\fontselect{#2}}
\itemmax=\tableindent %
\advance \itemmax by -\itemmargin %
\advance \leftskip by \tableindent %
\parindent = 0pt
\parskip = \smallskipamount
\ifdim \parskip=0pt \parskip=2pt \fi%
\def\Etable{\par\endgroup\afterenvbreak}%
\let\item = \internalBitem %
\let\itemx = \internalBitemx %
\let\kitem = \internalBkitem %
\let\kitemx = \internalBkitemx %
\let\xitem = \internalBxitem %
\let\xitemx = \internalBxitemx %
}

% This is the counter used by @enumerate, which is really @itemize

\newcount \itemno

\def\itemize{\parsearg\itemizezzz}

\def\itemizezzz #1{\itemizey {#1}{\Eitemize}}

\def\itemizey #1#2{%
\aboveenvbreak %
\begingroup %
\itemno = 0 %
\itemmax=\itemindent %
\advance \itemmax by -\itemmargin %
\advance \leftskip by \itemindent %
\parindent = 0pt
\parskip = \smallskipamount
\ifdim \parskip=0pt \parskip=2pt \fi%
\def#2{\par\endgroup\afterenvbreak}%
\def\itemcontents{#1}%
\let\item=\itemizeitem
}

\let\ptexbullet=\bullet
\def\bullet{$\ptexbullet$}

\def\enumerate{\itemizey{\the\itemno.}\Eenumerate}

% Definition of @item while inside @itemize.

\def\itemizeitem{%
\advance\itemno by 1
\par
\smallbreak 
\ifhmode \errmessage{\in hmode at itemizeitem}\fi
{\parskip=0in \hskip 0pt
\hbox to 0pt{\hss \itemcontents\hskip \itemmargin}%
\vadjust{\penalty 300}%
 }}

\message{figures,} % Floating insertions, basically
\newcount\figno \figno=0
\def\basicaption{\parsearg\captionx}
\def\captionx #1{\global\advance\figno by 1%
\par{\textfonts\line{\hfil{\bf Figure \the\figno .} #1\hfil}}}
\def\filcaption{\vfil\basicaption}
\def\iEfigure{\smallskip\hrule\endinsert}
\def\fullpagefigure{%
\pageinsert\inENV\let\caption=\filcaption\let\Efullpagefigure=\iEfigure\hrule\smallskip}
\def\figure{% This tries to be near where the command was
\midinsert\inENV\let\caption=\basicaption\let\Efigure=\iEfigure\hrule\smallskip}
\def\topfigure{% This tries to be near the top of a page
\topinsert\inENV\let\caption=\basicaption\let\Etopfigure=\iEfigure\hrule\smallskip}

\message{footnotes,}% Footnotes

\newcount \footnoteno

\def\supereject{\par\penalty -20000\footnoteno =0 }

\let\ptexfootnote=\footnote

{\catcode `\@=11
\gdef\footnote{\global\advance \footnoteno by \@ne
\edef\thisfootno{$^{\the\footnoteno}$}%
\let\@sf\empty
\ifhmode\edef\@sf{\spacefactor\the\spacefactor}\/\fi
\thisfootno\@sf\parsearg\footnotezzz}

\gdef\footnotezzz #1{\insert\footins{
\interlinepenalty\interfootnotelinepenalty
\splittopskip\ht\strutbox % top baseline for broken footnotes
\splitmaxdepth\dp\strutbox \floatingpenalty\@MM
\leftskip\z@skip \rightskip\z@skip \spaceskip\z@skip \xspaceskip\z@skip
\footstrut\hang\textindent{\thisfootno}#1\strut}}

} %end \catcode `\@=11

\message{indexing,}
% Index generation facilities

% Define \newwrite to be identical to plain tex's \newwrite
% except not \outer, so it can be used within \newindex.
{\catcode`\@=11
\gdef\newwrite{\alloc@7\write\chardef\sixt@@n}}

% \newindex {foo} defines an index named foo.
% It automatically defines \fooindex such that
% \fooindex ...rest of line... puts an entry in the index foo.
% It also defines \fooindfile to be the number of the output channel for
% the file that	accumulates this index.  The file's extension is foo.
% The name of an index should be no more than 2 characters long
% for the sake of vms.

% @defindex foo  is the form you write in a bolio file.

\def\defindex {\parsearg\newindex}

\def\newindex #1{
\expandafter\newwrite \csname#1indfile\endcsname% Define number for output file
\openout \csname#1indfile\endcsname \jobname.#1	% Open the file
\expandafter\xdef\csname#1index\endcsname{%	% Define \xxxindex
\noexpand\doindex {#1}}
}

% @synindex foo bar    makes index foo feed into index bar.
% Do this instead of @defindex foo if you don't want it as a separate index.
\def\synindex #1 #2 {%
\expandafter \xdef \csname#1index\endcsname {\noexpand\csname#2index\endcsname}%
\expandafter \chardef \csname#1indfile\endcsname %
= \csname#2indfile\endcsname}

% Define \doindex, the driver for all \fooindex macros.
% Argument #1 is generated by the calling \fooindex macro,
%  and it is "foo", the name of the index.

% \doindex just uses \parsearg; it calls \doind for the actual work.
% This is because \doind is more useful to call from other macros.

% There is also \dosubind {index}{topic}{subtopic}
% which makes an entry in a two-level index such as the operation index.

\def\doindex#1{\edef\indexname{#1}\parsearg\singleindexer}
\def\singleindexer #1{\doind{\indexname}{#1}}

\def\indexdummies{%
\def\bf{\realbackslash bf }
\def\rm{\realbackslash rm }
\def\sl{\realbackslash sl texindex.}
}

% To define \realbackslash, we must make \ not be an escape.
% We must first make another character (@) an escape
% so we do not become unable to do a definition.

{\catcode`\@=0 \catcode`\\=\other
@gdef@realbackslash{\}}

\def\doind #1#2{%
{\indexdummies  % Must do this here, since \bf, etc expand at this stage
\let\folio=0 \edef\temp{ % Expand all macros now EXCEPT \folio
\write \csname#1indfile\endcsname{%
\realbackslash entry {#2}{\folio}{#2}}}%
\temp}}

\def\dosubind #1#2#3{%
{\indexdummies  % Must do this here, since \bf, etc expand at this stage
\let\folio=0 \edef\temp{ %
\write \csname#1indfile\endcsname{%
\realbackslash entry {#2 #3}{\folio}{#2}{#3}}}%
\temp}}

% The index entry written in the file actually looks like
%  \entry {sortstring}{page}{topic}
% or
%  \entry {sortstring}{page}{topic}{subtopic}
% The texindex program reads in these files and writes files
% containing these kinds of lines:
%  \initial {c}
%     before the first topic whose initial is c
%  \entry {topic}{pagelist}
%     for a topic that is used without subtopics
%  \primary {topic}
%     for the beginning of a topic that is used with subtopics
%  \secondary {subtopic}{pagelist}
%     for each subtopic.

% Define the user-accessible indexing commands 
% @findex, @vindex, @kindex, @cindex.

\def\findex {\fnindex}
\def\kindex {\kwindex}
\def\cindex {\cpindex}
\def\vindex {\vrindex}

\def\cindexsub {\begingroup\obeylines\cindexsub}
{\obeylines %
\gdef\cindexsub "#1" #2^^M{\endgroup %
\dosubind{cp}{#2}{#1}}}

% Define the macros used in formatting output of the sorted index material.

% This is what you call to cause a particular index to get printed.
% Write
% @unnumbered Function Index
% @printindex fn

\def\printindex{\parsearg\doprintindex}

\def\doprintindex#1{\tex %
\catcode`\%=\other\catcode`\&=\other\catcode`\#=\other
\catcode`\@=\other\catcode`\$=\other\catcode`\_=\other
\catcode`\~=\other
\indexfonts\rm \tolerance=9500 \advance\baselineskip -1pt
\begindoublecolumns
\openin 1 \jobname.#1s
\ifeof 1 \else \closein 1 \input \jobname.#1s
\fi
\enddoublecolumns
\Etex}

% These macros are used by the sorted index file itself.
% Change them to control the appearance of the index.

\outer\def\initial #1{\bigbreak\line{\secbf#1\hfill}\kern 2pt\penalty3000}

\outer\def\entry #1#2{
{\parfillskip=0in \parskip=0in \parindent=0in
\hangindent=1in \hangafter=1%
\noindent\hbox{#1}\leaders\Dotsbox\hskip 0pt plus 1filll #2\par
}}

\def\primary #1{\line{#1\hfil}}

\newskip\secondaryindent \secondaryindent=0.5cm

\def\secondary #1#2{
{\parfillskip=0in \parskip=0in
\hangindent =1in \hangafter=1
\noindent\hskip\secondaryindent\hbox{#1}\leaders\Dotsbox\hskip 0pt plus 1filll#2\par
}}

%% Define two-column mode, which is used in indexes.
%% Adapted from the TeXBook, page 416
\catcode `\@=11

\newbox\partialpage

\def\begindoublecolumns{\begingroup
  \output={\global\setbox\partialpage=\vbox{\unvbox255\kern -\topskip \kern \baselineskip}}\eject
  \output={\doublecolumnout} \hsize=3.11in \vsize=19.1in}
\def\enddoublecolumns{\output={\balancecolumns}\eject
  \endgroup \pagegoal=\vsize}

\def\doublecolumnout{\splittopskip=\topskip \splitmaxdepth=\maxdepth
  \dimen@=\pageheight \advance\dimen@ by-\ht\partialpage
  \setbox0=\vsplit255 to\dimen@ \setbox2=\vsplit255 to\dimen@
  \onepageout\pagesofar \unvbox255 \penalty\outputpenalty}
\def\pagesofar{\unvbox\partialpage %
  \wd0=\hsize \wd2=\hsize \hbox to\pagewidth{\box0\hfil\box2}}
\def\balancecolumns{\setbox0=\vbox{\unvbox255} \dimen@=\ht0
  \advance\dimen@ by\topskip \advance\dimen@ by-\baselineskip
  \divide\dimen@ by2 \splittopskip=\topskip
  {\vbadness=10000 \loop \global\setbox3=\copy0
    \global\setbox1=\vsplit3 to\dimen@
    \ifdim\ht3>\dimen@ \global\advance\dimen@ by1pt \repeat}
  \setbox0=\vbox to\dimen@{\unvbox1}  \setbox2=\vbox to\dimen@{\unvbox3}
  \pagesofar}

\catcode `\@=\other
\message{sectioning,}
% Define chapters, sections, etc.

\outer\def\chapter{\parsearg\chapx}
\outer\def\unnumbered{\parsearg\chapxn}
\outer\def\appendix{\parsearg\chapapx}
\outer\def\section{\parsearg\secx}
\outer\def\appendixsection{\parsearg\apsecx}
\outer\def\unnumberedsec{\parsearg\secxn}
\outer\def\subsection{\parsearg\subsecx}
\outer\def\subsubsection{\parsearg\subsubsecx}

\newcount \chapno
\newcount \secno
\newcount \subsecno
\newcount \subsubsecno

% This counter is funny since it counts through charcodes of letters A, B, ...
\newcount \appendixno  \appendixno = `\@
\def\appendixletter{\char\the\appendixno}

\newwrite \contentsfile
\openout \contentsfile = \jobname.toc

% Each @chapter defines this as the name of the chapter.
% page headings and footings can use it.  @section does likewise

\def\thischapter{} \def\thissection{}
\def\seccheck#1{\levelcheck \if \pageno<0 %
\errmessage{@#1 not allowed after generating table of contents}\fi
%
}

\def\chapx #1{\seccheck{chapter}%
\secno=0 \subsecno=0 \subsubsecno=0 \global\advance \chapno by 1 \message{Chapter \the\chapno}%
\chapmacro {#1}{\the\chapno}%
\gdef\thissection{#1}\gdef\thischapter{#1}%
\edef\temp{{\realbackslash chapentry {#1}{\the\chapno}{\noexpand\folio}}}%
\write \contentsfile \temp  %
}

\def\chapapx #1{\seccheck{appendix}%
\secno=0 \subsecno=0 \subsubsecno=0 \global\advance \appendixno by 1 \message{Appendix \appendixletter}%
\chapmacro {#1}{Appendix \appendixletter}%
\gdef\thischapter{#1}\gdef\thissection{#1}%
\edef\temp{{\realbackslash chapentry {#1}{Appendix \appendixletter}{\noexpand\folio}}}%
\write \contentsfile \temp  %
}

\def\chapxn #1{\seccheck{unnumbered}%
\secno=0 \subsecno=0 \subsubsecno=0 \message{(#1)}
\unnumbchapmacro {#1}%
\gdef\thischapter{#1}\gdef\thissection{#1}%
\edef\temp{{\realbackslash unnumbchapentry {#1}{\noexpand\folio}}}%
\write \contentsfile \temp  %
}

\def\secx #1{\seccheck{section}%
\subsecno=0 \subsubsecno=0 \global\advance \secno by 1 %
\gdef\thissection{#1}\secheading {#1}{\the\chapno}{\the\secno}%
\edef\temp{{\realbackslash secentry %
{#1}{\the\chapno}{\the\secno}{\noexpand\folio}}}%
\write \contentsfile \temp %
}

\def\apsecx #1{\seccheck{appendixsection}%
\subsecno=0 \subsubsecno=0 \global\advance \secno by 1 %
\gdef\thissection{#1}\secheading {#1}{\appendixletter}{\the\secno}%
\edef\temp{{\realbackslash secentry %
{#1}{\appendixletter}{\the\secno}{\noexpand\folio}}}%
\write \contentsfile \temp %
}

\def\secxn #1{\seccheck{unnumberedsec}%
\plainsecheading {#1}\gdef\thissection{#1}%
\edef\temp{{\realbackslash unnumbsecentry %
{#1}{\noexpand\folio}}}%
\write \contentsfile \temp %
}

\def\subsecx #1{\seccheck{subsection}%
\gdef\thissection{#1}\subsubsecno=0 \global\advance \subsecno by 1 %
\subsecheading {#1}{\the\chapno}{\the\secno}{\the\subsecno}%
\edef\temp{{\realbackslash subsecentry %
{#1}{\the\chapno}{\the\secno}{\the\subsecno}{\noexpand\folio}}}%
\write \contentsfile \temp %
}

\def\subsubsecx #1{\seccheck{subsubsection}%
\gdef\thissection{#1}\global\advance \subsubsecno by 1 %
\subsubsecheading {#1}{\the\chapno}{\the\secno}{\the\subsecno}{\the\subsubsecno}%
\edef\temp{{\realbackslash subsubsecentry %
{#1}{\the\chapno}{\the\secno}{\the\subsecno}{\the\subsubsecno}{\noexpand\folio}}}%\
\write \contentsfile \temp %
}

% Define @majorheading, @heading and @subheading

\outer\def\majorheading{\parsearg\majorheadingzzz}

\def\majorheadingzzz #1{%
{\advance\chapheadingskip by 10pt \chapbreak }%
{\chapfonts \line{\chaprm #1\hfill}}\bigskip \par\penalty 200}

\outer\def\heading{\parsearg\headingzzz}

\def\headingzzz #1{\chapbreak %
{\chapfonts \line{\chaprm #1\hfill}}\bigskip \par\penalty 200}

\outer\def\subheading{\parsearg\secheadingi}

\outer\def\subsubheading{\parsearg\subsecheadingi}

% These macros generate a chapter, section, etc. heading only
% (including whitespace, linebreaking, etc. around it),
% given all the information in convenient, parsed form.

%%% Args are the skip and penalty (usually negative)
\def\dobreak#1#2{\par\ifdim\lastskip<#1\removelastskip\penalty#2\vskip#1 \fi}

\def\setchapterstyle #1 {\csname CHAPF#1\endcsname}

%%% Define plain chapter starts, and page on/off switching for it
% Parameter controlling skip before chapter headings (if needed)

\newskip \chapheadingskip \chapheadingskip = 30pt plus 8pt minus 4pt

\def\chapbreak{\dobreak \chapheadingskip {-100}}
\def\chappager{\par\vfill\supereject}
\def\chapoddpage{\chappager \ifodd\pageno \else \hbox to 0pt{} \chappager\fi}

\def\setchapternewpage #1 {\csname CHAPPAG#1\endcsname}
\def\CHAPPAGoff{\global\let\pchapsepmacro=\chapbreak}
\def\CHAPPAGon{\global\let\pchapsepmacro=\chappager}
\def\CHAPPAGodd{\global\let\pchapsepmacro=\chapoddpage}
\CHAPPAGon

\def\CHAPFplain{
\global\let\chapmacro=\chfplain
\global\let\unnumbchapmacro=\unnchfplain}

\def\chfplain #1#2{%
\pchapsepmacro %
{\chapfonts \line{\chaprm #2.\enspace #1\hfill}}\bigskip \par\penalty 5000 %
}

\def\unnchfplain #1{%
\pchapsepmacro %
{\chapfonts \line{\chaprm #1\hfill}}\bigskip \par\penalty 10000 %
}
\CHAPFplain % The default

\def\unnchfopen #1{%
\chapoddpage {\chapfonts \line{\chaprm #1\hfill}}\bigskip \par\penalty 10000 %
}

\def\chfopen #1#2{\chapoddpage {\chapfonts
\vbox to 3in{\vfil \hbox to\hsize{\hfil #2} \hbox to\hsize{\hfil #1} \vfil}}%
\par\penalty 5000 %
}

\def\CHAPFopen{
\global\let\chapmacro=\chfopen
\global\let\unnumbchapmacro=\unnchfopen}

% Parameter controlling skip before section headings.

\newskip \subsecheadingskip  \subsecheadingskip = 17pt plus 4pt minus 4pt
\def\subsecheadingbreak{\dobreak \subsecheadingskip {-30}}

\newskip \secheadingskip  \secheadingskip = 21pt plus 6pt minus 4pt
\def\secheadingbreak{\dobreak \secheadingskip {-50}}

\def\secheading #1#2#3{\secheadingi {#2.#3\enspace #1}}
\def\plainsecheading #1{\secheadingi {#1}}
\def\secheadingi #1{{\advance \secheadingskip by \parskip %
\secheadingbreak} %
{\secfonts \line{\secrm #1\hfill}} %
\ifdim \parskip<10pt \kern 10pt\kern -\parskip\fi \penalty 1000}

\def\subsecheading #1#2#3#4{{\advance \subsecheadingskip by \parskip %
\subsecheadingbreak} %
{\subsecfonts \line{\secrm#2.#3.#4\enspace #1\hfill}}
\ifdim \parskip<10pt \kern 10pt\kern -\parskip\fi \penalty 1000}

\def\subsubsecfonts{\subsecfonts} % Maybe this should change

\def\subsubsecheading #1#2#3#4#5{{\advance \subsecheadingskip by \parskip %
\subsecheadingbreak} %
{\subsubsecfonts \line{\secrm#2.#3.#4.#5\enspace #1\hfill}}
\ifdim \parskip<10pt \kern 10pt\kern -\parskip\fi \penalty 1000}

\message{toc printing,}

\def\Dotsbox{\hbox to 1em{\hss.\hss}} % Used by index macros

\def\finishcontents{%
\ifnum\pageno>0 %
\par\vfill\supereject %
\immediate\closeout \contentsfile%
\pageno=-1		% Request roman numbered pages
\fi}

\outer\def\contents{%
\finishcontents %
\unnumbchapmacro{Table of Contents}
\def\thischapter{Table of Contents}
{\catcode`\\=0
\catcode`\{=1		% Set up to handle contents files properly
\catcode`\}=2
\input \jobname.toc
}
\vfill \eject}

\outer\def\summarycontents{%
\finishcontents %
\unnumbchapmacro{Summary Table of Contents}
\def\thischapter{Summary Table of Contents}
{\catcode`\\=0
\catcode`\{=1		% Set up to handle contents files properly
\catcode`\}=2
\def\smallbreak{}
\def\secentry ##1##2##3##4{}
\def\subsecentry ##1##2##3##4##5{}
\def\subsubsecentry ##1##2##3##4##5##6{}
\def\unnumbsecentry ##1##2{}
\let\medbreak=\smallbreak
\input \jobname.toc
}
\vfill \eject}

\outer\def\bye{\par\vfill\supereject\tracingstats=1\ptexend}

% These macros generate individual entries in the table of contents
% The first argument is the chapter or section name.
% The last argument is the page number.
% The arguments in between are the chapter number, section number, ...

\def\chapentry #1#2#3{%
\medbreak
\line{#2.\space#1\leaders\hbox to 1em{\hss.\hss}\hfill #3}
}

\def\unnumbchapentry #1#2{%
\medbreak
\line{#1\leaders\Dotsbox\hfill #2}
}

\def\secentry #1#2#3#4{%
\line{\enspace\enspace#2.#3\space#1\leaders\Dotsbox\hfill#4}
}

\def\unnumbsecentry #1#2{%
\line{\enspace\enspace#1\leaders\Dotsbox\hfill #2}
}

\def\subsecentry #1#2#3#4#5{%
\line{\enspace\enspace\enspace\enspace
#2.#3.#4\space#1\leaders\Dotsbox\hfill #5}
}

\def\subsubsecentry #1#2#3#4#5#6{%
\line{\enspace\enspace\enspace\enspace\enspace\enspace
#2.#3.#4.#5\space#1\leaders\Dotsbox\hfill #6}
}

\message{environments,}

% @tex ... @end tex    escapes into raw Tex temporarily.

\def\tex{\begingroup
\catcode `\\=0 \catcode `\{=1 \catcode `\}=2
\catcode `\$=3 \catcode `\&=4 \catcode `\#=6
\catcode `\^=7 \catcode `\_=8 \catcode `\~=13 \let~=\tie
\let\{=\lbrace \let\}=\rbrace
\let\nobreak=\ptexnobreak
\let\.=\ptexdot
\let\#=\ptexnumsign
\let\*=\ptexstar
\let\+=\tabalign
\let\-=\ptexminus
\let\b=\ptexb \let\c=\ptexc \let\i=\ptexi \let\t=\ptext \let\l=\ptexl
\let\L=\ptexL
\catcode `\%=14 \let\Etex=\endgroup}

% Define @lisp ... @endlisp.
% @lisp does a \begingroup so it can rebind things,
% including the definition of @endlisp (which normally is erroneous).

% Amount to narrow the margins by for @lisp.
\newskip\lispnarrowing \lispnarrowing=0.3in

% This is the definition that ^M gets inside @lisp
{\obeyspaces%
\gdef\lisppar{ \endgraf}}

% Cause \obeyspaces to make each Space cause a word-separation
% rather than the default which is that it acts punctuation.
% This is because space in tt font looks funny.
{\obeyspaces %
\gdef\sepspaces{\def {\ }}}

\newskip\aboveenvskipamount \aboveenvskipamount=3pt
\def\aboveenvbreak{{\advance\aboveenvskipamount by \parskip
\par \ifdim\lastskip<\aboveenvskipamount
\removelastskip \penalty-50 \vskip\aboveenvskipamount \fi}}

\def\afterenvbreak{\par \ifdim\lastskip<\aboveenvskipamount
\removelastskip \penalty-50 \vskip\aboveenvskipamount \fi}

{\catcode`\@=\active%
% @ must be active when this definition is made,
% so that the \let@ can be parsed properly.
\gdef\lisp{\begingroup\inENV %This group ends at the end of the @lisp body
\aboveenvbreak \hfuzz=12truept % Don't be fussy
% Make spaces be word-separators rather than space tokens.
\sepspaces %
% Single space lines
\singlespace %
% The following causes blank lines not to be ignored
% by adding a space to the end of each line.
\let\par=\lisppar
\catcode`\{=\other \catcode`\}=\other
\def\Elisp{\endgroup\afterenvbreak}
\parskip=0pt \advance \rightskip by \lispnarrowing 
\advance \leftskip by \lispnarrowing
\parindent=0pt
\let\exdent=\internalexdent
\obeyspaces \obeylines \tt}}

\let\example=\lisp
\def\Eexample{\Elisp}

% This is @display; same as @lisp except use roman font.

{\catcode`\@=\active%
% @ must be active when this definition is made,
% so that the \let@ can be parsed properly.
\gdef\display{\begingroup\inENV %This group ends at the end of the @display body
\aboveenvbreak
% Make spaces be word-separators rather than space tokens.
\sepspaces %
% Single space lines
\singlespace %
% The following causes blank lines not to be ignored
% by adding a space to the end of each line.
\let\par=\lisppar
\def\Edisplay{\endgroup\afterenvbreak}
\parskip=0pt \advance \rightskip by \lispnarrowing 
\advance \leftskip by \lispnarrowing
\parindent=0pt
\let\exdent=\internalexdent
\obeyspaces \obeylines}}

% This is @format; same as @lisp except use roman font and don't narrow margins

{\catcode`\@=\active%
% @ must be active when this definition is made,
% so that the \let@ can be parsed properly.
\gdef\format{\begingroup\inENV %This group ends at the end of the @format body
\aboveenvbreak
% Make spaces be word-separators rather than space tokens.
\sepspaces %
\singlespace %
% The following causes blank lines not to be ignored
% by adding a space to the end of each line.
\let\par=\lisppar
\def\Eformat{\endgroup\afterenvbreak}
\parskip=0pt \parindent=0pt
\obeyspaces \obeylines}}

% This is @address; same as @format except put left margin in mid page

{\catcode`\@=\active%
% @ must be active when this definition is made,
% so that the \let@ can be parsed properly.
\gdef\address{\begingroup\inENV %This group ends at the end of the @address body
\aboveenvbreak
% Make spaces be word-separators rather than space tokens.
\sepspaces %
% The following causes blank lines not to be ignored
% by adding a space to the end of each line.
% This also causes @ to work when the directive name
% is terminated by end of line.
\let\par=\lisppar
\def\Eaddress{\endgroup\afterenvbreak}
\def\Eclosing{\endgroup\afterenvbreak}
\parskip=0pt \parindent=0pt
\obeyspaces \obeylines}}

\let\closing=\address

% @flushleft and @flushright

{\catcode`\@=\active%
% @ must be active when this definition is made,
% so that the \let@ can be parsed properly.
\gdef\flushleft{\begingroup\inENV %This group ends at the end of the @format body
\aboveenvbreak
% Make spaces be word-separators rather than space tokens.
\sepspaces %
% The following causes blank lines not to be ignored
% by adding a space to the end of each line.
% This also causes @ to work when the directive name
% is terminated by end of line.
\let\par=\lisppar
\def\Eflushleft{\endgroup\afterenvbreak}
\parskip=0pt \parindent=0pt
\obeyspaces \obeylines}}

{\catcode`\@=\active%
% @ must be active when this definition is made,
% so that the \let@ can be parsed properly.
\gdef\flushright{\begingroup\inENV %This group ends at the end of the @format body
\aboveenvbreak
% Make spaces be word-separators rather than space tokens.
\sepspaces %
% The following causes blank lines not to be ignored
% by adding a space to the end of each line.
% This also causes @ to work when the directive name
% is terminated by end of line.
\let\par=\lisppar
\def\Eflushright{\endgroup\afterenvbreak}
\parskip=0pt \parindent=0pt
\advance \leftskip by 0pt plus 1fill
\obeyspaces \obeylines}}

% @quotation - narrow the margins.

\def\quotation{\begingroup\inENV %This group ends at the end of the @quotation body
\aboveenvbreak
\singlespace
\def\Equotation{\par\endgroup\afterenvbreak}
\advance \rightskip by \lispnarrowing 
\advance \leftskip by \lispnarrowing}

% @undent - make every paragraph have a hanging indentation

\def\undent{\begingroup %This group ends at the end of the @undent body
\def\Eundent{\par\endgroup}
\everypar={\hangindent=\parindent \hskip-\parindent \hangafter=1 }}

% @document - nothing now, later it will read in the aux file, close toc files
\def\document{} \def\Edocument{}

\message{defuns,}
% Define formatter for defuns
% First, allow user to change definition object font (\df) internally
\def\setdeffont #1 {\csname DEF#1\endcsname}

\newskip\defbodyindent \defbodyindent=36pt
\newskip\defargsindent \defargsindent=50pt
\newskip\deftypemargin \deftypemargin=12pt
\newskip\deflastargmargin \deflastargmargin=18pt

\newcount\parencount
% define \functionparens, which makes ( and ) and & do special things.
% \functionparens affects the group it is contained in.
\def\activeparens{%
\catcode`\(=\active \catcode`\)=\active \catcode`\&=\active
\catcode`\[=\active \catcode`\]=\active}
{\activeparens % Now, smart parens don't turn on until &foo (see \amprm)
\gdef\functionparens{\boldbrax\let&=\amprm\parencount=0 }
\gdef\boldbrax{\let(=\opnr\let)=\clnr\let[=\lbrb\let]=\rbrb}

% Definitions of (, ) and & used in args for functions.
% This is the definition of ( outside of all parentheses.
\gdef\oprm#1 {{\rm\char`\(}#1 \bf \let(=\opnested %
\global\advance\parencount by 1 }
%
% This is the definition of ( when already inside a level of parens.
\gdef\opnested{\char`\(\global\advance\parencount by 1 }
%
\gdef\clrm{% Print a paren in roman if it is taking us back to depth of 0.
% also in that case restore the outer-level definition of (.
\ifnum \parencount=1 {\rm \char `\)}\vf \let(=\oprm \else \char `\) \fi
\global\advance \parencount by -1 }
% If we encounter &foo, then turn on ()-hacking afterwards
\gdef\amprm#1 {{\rm\&#1}\let(=\oprm \let)=\clrm\ }
%
\gdef\normalparens{\boldbrax\let&=\ampnr}
} % End of definition inside \activeparens
%% These parens (in \boldbrax) actually are a little bolder than the
%% contained text.  This is especially needed for [ and ]
\def\opnr{{\sf\char`\(}} \def\clnr{{\sf\char`\)}} \def\ampnr{\&}
\def\lbrb{{\tt\char`\[}} \def\rbrb{{\tt\char`\]}}

% First, defname, which formats the header line itself.
% #1 should be the function name.
% #2 should be the type of definition, such as "Function".

\def\defname #1#2{%
\leftskip = 0in  %
\noindent        %
\setbox0=\hbox{\hskip \deflastargmargin{\rm #2}\hskip \deftypemargin}%
\dimen0=\hsize \advance \dimen0 by -\wd0 % compute size for first line
\dimen1=\hsize \advance \dimen1 by -\defargsindent %size for continuations
\parshape 2 0in \dimen0 \defargsindent \dimen1     %
% Now output arg 2 ("Function" or some such)
% ending at \deftypemargin from the right margin,
% but stuck inside a box of width 0 so it does not interfere with linebreaking
\rlap{\rightline{{\rm #2}\hskip \deftypemargin}}%
\tolerance=10000 \hbadness=10000    % Make all lines underfull and no complaints
{\df #1}\enskip        % Generate function name
}

% Actually process the body of a definition
% #1 should be the terminating control sequence, such as \Edefun.
% #2 should be the "another name" control sequence, such as \defunx.
% #3 should be the control sequence that actually processes the header,
%    such as \defunheader.

\def\defparsebody #1#2#3{\begingroup\inENV% Environment for definitionbody
\medbreak %
% Define the end token that this defining construct specifies
% so that it will exit this group.
\def#1{\endgraf\endgroup\medbreak}%
\def#2{\begingroup\obeylines\activeparens\spacesplit#3}%
\parindent=0in \leftskip=\defbodyindent %
\begingroup\obeylines\activeparens\spacesplit#3}

\def\defmethparsebody #1#2#3#4 {\begingroup\inENV %
\medbreak %
% Define the end token that this defining construct specifies
% so that it will exit this group.
\def#1{\endgraf\endgroup\medbreak}%
\def#2##1 {\begingroup\obeylines\activeparens\spacesplit{#3{##1}}}%
\parindent=0in \leftskip=\defbodyindent %
\begingroup\obeylines\activeparens\spacesplit{#3{#4}}}

% Split up #2 at the first space token.
% call #1 with two arguments:
%  the first is all of #2 before the space token,
%  the second is all of #2 after that space token.
% If #2 contains no space token, all of it is passed as the first arg
% and the second is passed as empty.

{\obeylines
\gdef\spacesplit#1#2^^M{\endgroup\spacesplitfoo{#1}#2 \relax\spacesplitfoo}%
\long\gdef\spacesplitfoo#1#2 #3#4\spacesplitfoo{%
\ifx\relax #3%
#1{#2}{}\else #1{#2}{#3#4}\fi}}

% So much for the things common to all kinds of definitions.

% Define \defun.

% First, define the processing that is wanted for arguments of \defun
% Use this to expand the args and terminate the paragraph they make up

\def\defunargs #1{\functionparens \vf #1%
\ifnum\parencount=0 \else \errmessage{unbalanced parens in @def arguments}\fi%
\interlinepenalty=10000
\endgraf\vskip -\parskip \penalty 10000}

% \lastfunction is always defined to the name in the most recennt @defun
% It is used by @kitem as the subtopic to index the keyword under

\def\lastfunction{}

% Do complete processing of one @defun or @defunx line already parsed.

\def\defunheader #1#2{\doind {fn}{#1} % Make entry in function index
\gdef\lastfunction{#1}%
\dosetq {#1-fun}{page}%
{\defname {#1}{Function}\defunargs {#2}}%
}

\def\defmacheader #1#2{\doind {fn}{#1} % Make entry in function index
\gdef\lastfunction{#1}%
\dosetq {#1-fun}{page}%
\begingroup\defname {#1}{Macro}%
\defunargs {#2}\endgroup %
}

\def\defspecheader #1#2{\doind {fn}{#1} % Make entry in function index
\gdef\lastfunction{#1}%
\dosetq {#1-fun}{page}%
\begingroup\defname {#1}{Special form}%
\defunargs {#2}\endgroup %
}

\def\defmessageheader #1#2{\doind {op}{#1} % Make entry in operation index
\gdef\lastfunction{#1}%
\dosetq {#1-message}{page}%
\begingroup\defname {#1}{Operation}%
\defunargs {#2}\endgroup %
}

% Now we can define @defun itself.

\def\defun{\defparsebody\Edefun\defunx\defunheader}
\def\defmac{\defparsebody\Edefmac\defmacx\defmacheader}
\def\defspec{\defparsebody\Edefspec\defspecx\defspecheader}
\def\defmessage{\defparsebody\Edefmessage\defmessagex\defmessageheader}

% This definition is run if you use @defunx
% anywhere other than immediately after a @defun or @defunx.

\def\defunx #1 {\errmessage{@defunx in invalid context}}
\def\defmacx #1 {\errmessage{@defmacx in invalid context}}
\def\defspecx #1 {\errmessage{@defspecx in invalid context}}
\def\defmessagex #1 {\errmessage{@defmessagex in invalid context}}

% @defmethod, @defmetamethod, and so on

% Do complete processing of one @defmethod line already parsed.

\def\defmethodheader #1#2#3{\dosubind {op}{#2}{on {\bf #1}}% Make entry in operation index
\gdef\lastfunction{#2}%
\dosetqflushcolon {#1-}#2-method {page}%
\begingroup\defname {#2}{Operation on {\bf #1}}%
\defunargs {#3}\endgroup %
}

\def\defmetamethodheader #1#2#3{\dosubind {op}{#2}{on #1}% Make entry in operation index
\gdef\lastfunction{#2}%
\dosetqflushcolon {#1-}#2-method {page}%
\begingroup\defname {#2}{Operation on #1}%
\defunargs {#3}\endgroup %
}

\def\defivarheader #1#2#3{%
\gdef\lastfunction{#2}%
\dosubind {iv}{#2}{of {\bf #1}}% Make entry in instance variable index
\dosetq {#1-#2-ivar}{page}%
\begingroup\defname {#2}{Instance variable of {\bf #1}}%
\defvarargs {#3}\endgroup %
}

\def\defmetaivarheader #1#2#3{%
\gdef\lastfunction{#2}%
\dosubind {iv}{#2}{of #1}% Make entry in ivar index
\dosetq {#1-#2-ivar}{page}%
\begingroup\defname {#2}{Instance variable of #1}%
\defvarargs {#3}\endgroup %
}

\def\definitheader #1#2#3{%
\gdef\lastfunction{#2}%
\dosubind {io}{#2}{for {\bf #1}}% Make entry in init option index
\dosetqflushcolon {#1-}#2-init-option {page}%
\begingroup\defname {#2}{Init keyword for {\bf #1}}%
\defvarargs {#3}\endgroup %
}

\def\defmetainitheader #1#2#3{%
\gdef\lastfunction{#2}%
\dosubind {io}{#2}{for #1}% Make entry in init option index
\dosetqflushcolon {#1-}#2-init-option {page}%
\begingroup\defname {#2}{Init keyword for #1}%
\defvarargs {#3}\endgroup %
}

\def\defmethod{\defmethparsebody\Edefmethod\defmethodx\defmethodheader}
\def\defmetamethod{%
\defmethparsebody\Edefmetamethod\defmetamethodx\defmetamethodheader}

\def\defivar{\defmethparsebody\Edefivar\defivarx\defivarheader}
\def\defmetaivar{%
\defmethparsebody\Edefmetaivar\defmetaivarx\defmetaivarheader}

\def\definit{\defmethparsebody\Edefinit\definitx\definitheader}
\def\defmetainit{%
\defmethparsebody\Edefmetainit\defmetainitx\defmetainitheader}

% These definitions are run if you use @defmethodx, etc.,
% anywhere other than immediately after a @defmethod, etc.

\def\defmethodx #1 {\errmessage{@defmethodx in invalid context}}
\def\defmetamethodx #1 {\errmessage{@defmetamethodx in invalid context}}

\def\defivarx #1 {\errmessage{@defivarx in invalid context}}
\def\defmetaivarx #1 {\errmessage{@defmetaivarx in invalid context}}

\def\definitx #1 {\errmessage{@definitx in invalid context}}
\def\defmetainitx #1 {\errmessage{@defmetainitx in invalid context}}

% Now @defvar

% First, define the processing that is wanted for arguments of @defvar.
% This is actually simple: just print them in roman.
% This must expand the args and terminate the paragraph they make up
\def\defvarargs #1{\normalparens #1%
\interlinepenalty=10000
\endgraf\vskip -\parskip \penalty 10000}

% Do complete processing of one @defvar or @defvarx line already parsed.

\def\defvarheader #1#2{\doind {vr}{#1}% Make entry in var index
\gdef\lastfunction{#1}%
\dosetq {#1-var}{page}%
\begingroup\defname {#1}{Variable}%
\defvarargs {#2}\endgroup %
}

\def\defconstheader #1#2{\doind {vr}{#1}% Make entry in var index
\gdef\lastfunction{#1}%
\dosetq {#1-var}{page}%
\begingroup\defname {#1}{Constant}%
\defvarargs {#2}\endgroup %
}

\def\defmeterheader #1#2{\doind {mt}{#1}% Make entry in meter index
\gdef\lastfunction{#1}%
\dosetq {#1-meter}{page}%
\begingroup\defname {#1}{Meter}%
\defvarargs {#2}\endgroup %
}

\def\defresourceheader #1#2{\doind {rs}{#1}% Make entry in resource index
\gdef\lastfunction{#1}%
\dosetq {#1-resource}{page}%
\begingroup\defname {#1}{Resource}%
\defvarargs {#2}\endgroup %
}

% Now we can define @defvar itself.  Also @defconst.

\def\defvar{\defparsebody\Edefvar\defvarx\defvarheader}
\def\defconst{\defparsebody\Edefconst\defconstx\defconstheader}
\def\defmeter{\defparsebody\Edefmeter\defmeterx\defmeterheader}
\def\defresource{\defparsebody\Edefresource\defresourcex\defresourceheader}

% This definition is run if you use @defvarx
% anywhere other than immediately after a @defvar or @defvarx.

\def\defvarx #1 {\errmessage{@defvarx in invalid context}}
\def\defconstx #1 {\errmessage{@defconstx in invalid context}}
\def\defmeterx #1 {\errmessage{@defmeterx in invalid context}}
\def\defresourcex #1 {\errmessage{@defresourcex in invalid context}}

% Now define @defflavor
% Args are printed in bold, a slight difference from @defvar.

\def\defflavargs #1{\bf \defvarargs{#1}}

% Do complete processing of one @defflavor (or similar construct) line.

\def\defflavorheader #1#2{\doind {fl}{#1}% Make entry in flavor index
\gdef\lastfunction{#1}%
\dosetq {#1-flavor}{page}%
\begingroup\defname {#1}{Flavor}%
\defflavargs {#2}\endgroup %
}

\def\defconditionheader #1#2{\doind {cn}{#1}% Make entry in condition index
\gdef\lastfunction{#1}%
\dosetq {#1-condition}{page}%
\begingroup\defname {#1}{Condition}%
\defflavargs {#2}\endgroup %
}

\def\defconditionflavorheader #1#2{%
\gdef\lastfunction{#1}%
\doind {cn}{#1}% Make entry in condition index
\doind {fl}{#1}% Make entry in flavor index
\dosetq {#1-condition}{page}%
\dosetq {#1-flavor}{page}%
\dosetq {#1-condition-flavor}{page}%
\doind {fl}{#1}% Make entry in flavor index
\begingroup\defname {#1}{Condition flavor}%
\defflavargs {#2}\endgroup %
}

\def\defflavor{\defparsebody\Edefflavor\defflavorx\defflavorheader}
\def\defcondition{%
\defparsebody\Edefcondition\defconditionx\defconditionheader}
\def\defconditionflavor{%
\defparsebody\Edefconditionflavor\defconditionflavorx\defconditionflavorheader}

% This definition is run if you use @defflavorx, etc
% anywhere other than immediately after a @defflavor, etc.

\def\defflavorx #1 {\errmessage{@defflavorx in invalid context}}
\def\defconditionx #1 {\errmessage{@defconditionx in invalid context}}
\def\defconditionflavorx #1 {\errmessage{@defconditionflavorx in invalid context}}

\message{cross reference,}
% Define cross-reference macros
\newwrite \auxfile

% Define @setq.  @setq foo page  defines a string-variable named foo
%  whose value is "page nnn".
% Also allowed are
% @setq foo section-page      defines as "section c.s.ss, page nnn"
% @setq foo page-number       defines as "nnn"
% @setq foo chapter-number    defines as "c"
% @setq foo section-number    defines as "c.s.ss"

% @setx foo text              defines foo as "text", literally.

% Turn on \obeylines before parsing the arguments.
% \setqx does the actual parsing.
\def\setq{\begingroup\obeylines \setqx}

\def\setx{\begingroup\obeylines \setxx}

% Define \setqx, which just puts the arguments into a \write
% preceded by \internalsetq, which will expand at write time and do all the real work,

{ \obeylines %
\gdef\setqx #1 #2^^M{%
\let\folio=0\edef\next{\write\auxfile{\internalsetq {#1}{#2}}}%
\next\endgroup}
%
\gdef\setxx #1 #2^^M{%
\write\auxfile{'xrdef {#1}{#2}}\endgroup}}

%% @label for Scribe.  setq's x-pg, x-title, x-snam, x for
%% @pageref, @title, @nameref, @ref
\def\label{\parsearg\labelx} \def\pageref{\parsearg\pagerefx}
\def\nameref{\parsearg\namerefx}
\def\labelx#1{\dosetq{#1-pg}{page-number}\dosetq{#1-title}{title}%
\dosetq{#1-snam}{section-name}\dosetq{#1}{section-number}}
\def\pagerefx#1{\refx{#1-pg}} \def\namerefx#1{\refx{#1-snam}}
\def\titlex#1{\refx{#1-title}}

% \dosetq is the interface for calls from other macros

\def\dosetq #1#2{{\let\folio=0%
\edef\next{\write\auxfile{\internalsetq {#1}{#2}}}%
\next}}

% dosetqflushcolon {foo-}:bar {value} defines foo-bar to value,
% thus flushing the colon from the front of :bar.
% This is how we define the variables for @defmethod, etc.

\def\dosetqflushcolon #1:#2 #3{\write\auxfile{\internalsetq {#1#2}{#3}}}

% \internalsetq {foo}{page} expands into CHARACTERS 'xrdef {foo}{...expansion of \Ypage...}
% When the aux file is read, ' is the escape character

\def\internalsetq #1#2{'xrdef {#1}{\csname Y#2\endcsname}}

% We want to do \edef\Xfoo{\Ypage}
% Define \Ypage to generate "page nnn".

\def\Ypage{page \folio}
\def\Ytitle{\ifnum\secno=0 chapter\else section\fi}
% Define \Ysection-page to generate "section m.n, page n"
% Define \Ysection-number-and-type to generate "section m.n"
% Define \Ysection-number to generate just "m.n"

{\catcode `\-=11
\gdef\Ysection-page{%
\ifnum\secno=0 chapter\xreftie\the\chapno, page\xreftie\folio %
\else \ifnum \subsecno=0 section\xreftie\the\chapno.\the\secno, page\xreftie\folio %
\else \ifnum \subsubsecno=0 %
section\xreftie\the\chapno.\the\secno.\the\subsecno, page\xreftie\folio %
\else %
section\xreftie\the\chapno.\the\secno.\the\subsecno.\the\subsubsecno, page\xreftie\folio %
\fi \fi \fi}

\gdef\Ysection-number-and-type{%
\ifnum\secno=0 chapter\xreftie\the\chapno %
\else \ifnum \subsecno=0 section\xreftie\the\chapno.\the\secno %
\else \ifnum \subsubsecno=0 %
section\xreftie\the\chapno.\the\secno.\the\subsecno %
\else %
section\xreftie\the\chapno.\the\secno.\the\subsecno.\the\subsubsecno %
\fi \fi \fi }

\gdef\xreftie{'tie}

\gdef\Ysection-number{%
\ifnum\secno=0 \the\chapno %
\else \ifnum \subsecno=0 \the\chapno.\the\secno %
\else \ifnum \subsubsecno=0 %
\the\chapno.\the\secno.\the\subsecno %
\else %
\the\chapno.\the\secno.\the\subsecno.\the\subsubsecno %
\fi \fi \fi}}

% Define \Ychapter-number, \Ypage-number, \Ychapter-name \Ysection-name
{\catcode `\-=11
\gdef\Ychapter-number{\the\chapno}
\gdef\Ypage-number{\folio}
\gdef\Ychapter-name{\thischapter}
\gdef\Ysection-name{\thissection}
}

% Define @ref, and alternatively the character ^V, to reference a cross-ref.
\def\ref{\parsearg\refx}
\def\refx#1{%
{%
\setbox0=\hbox{\csname X#1\endcsname}%
\ifdim\wd0>0in \else  	% If not defined, say something at least.
\expandafter\gdef\csname X#1\endcsname {$<$undefined$>$}%
\message {WARNING: Cross-reference "#1" used but not yet defined}%
\message {}%
\fi %
\csname X#1\endcsname %It's defined, so just use it.
}}

%% Mention a Lisp construct, telling the reader the page number
%% The first argument is the construct, the second the name of the construct
%% Sample use: @see[cons][fun] --> cons (see page 23)
\def\see[#1][#2]{{\li #1} (see \refx{#1-#2})}

\catcode`\^^V=\active
\let^^V=\ref

% Read the last existing aux file, if any.  No error if none exists.

% This is the macro invoked by entries in the aux file.
\def\xrdef #1#2{
{\catcode`\'=\other\expandafter \gdef \csname X#1\endcsname {#2}}}

{
\catcode `\^^@=\other
\catcode `\=\other
\catcode `\=\other
\catcode `\^^C=\other
\catcode `\^^D=\other
\catcode `\^^E=\other
\catcode `\^^F=\other
\catcode `\^^G=\other
\catcode `\^^H=\other
\catcode `\=\other
\catcode `\^^L=\other
\catcode `\=\other
\catcode `\=\other
\catcode `\=\other
\catcode `\=\other
\catcode `\=\other
\catcode `\=\other
\catcode `\=\other
\catcode `\=\other
\catcode `\=\other
\catcode `\=\other
\catcode `\=\other
\catcode `\=\other
\catcode `\=\other
\catcode `\^^[=\other
\catcode `\^^\=\other
\catcode `\^^]=\other
\catcode `\^^^=\other
\catcode `\^^_=\other
\catcode `\@=\other
\catcode `\^=\other
\catcode `\~=\other
\catcode `\[=\other
\catcode `\]=\other
\catcode`\"=\other
\catcode`\_=\other
\catcode`\|=\other
\catcode`\<=\other
\catcode`\>=\other
\catcode `\$=\other
\catcode `\#=\other
\catcode `\&=\other

% the aux file uses ' as the escape.
% Turn off \ as an escape so we do not lose on
% entries which were dumped with control sequences in their names.
% For example, 'xrdef {$\leq $-fun}{page ...} made by @defun ^^
% Reference to such entries still does not work the way one would wish,
% but at least they do not bomb out when the aux file is read in.

\catcode `\{=1 \catcode `\}=2
\catcode `\%=\other
\catcode `\'=0
\catcode `\\=\other

'openin 1 'jobname.aux
'ifeof 1 'else 'closein 1 'input 'jobname.aux
'fi
}

% Open the new aux file.  Tex will close it automatically at exit.

\openout \auxfile=\jobname.aux

% End of control word definitions.

\message{and turning on Bolio.}

% Turn off all special characters except @
% (and those which the user can use as if they were ordinary)
% Define certain chars to be always in tt font.

\catcode`\"=\active
\def\activedoublequote{{\tt \char '042}}
\let"=\activedoublequote
\catcode`\~=\active
\def~{{\tt \char '176}}
\chardef\hat=`\^
\catcode`\^=\active
\def^{{\tt \hat}}
\catcode`\_=\active
\def_{{\tt \char '137}}
\catcode`\|=\active
\def|{{\tt \char '174}}
\chardef \less=`\<
\catcode`\<=\active
\def<{{\tt \less}}
\chardef \gtr=`\>
\catcode`\>=\active
\def>{{\tt \gtr}}

%%% CHANGE! Used by \activeparens for \defspec's benefit
\def\nlbrace{\ifmmode \delimiter"4266308 \else {\tensy\char'146}\fi}
\def\nrbrace{\ifmmode \delimiter"5267309 \else {\tensy\char'147}\fi}
 \let\{=\nlbrace \let \}=\nrbrace

\def\mylbrace {{\tt \char '173}}
\def\myrbrace {{\tt \char '175}}

\catcode`\{=\active
\let{=\mylbrace
\catcode`\}=\active
\let}=\myrbrace

%% These look ok in all fonts, so just make them not special.  The @rm below
%% makes sure that the current font starts out as the newly loaded cmr10
\catcode`\$=\other \catcode`\%=\other \catcode`\&=\other \catcode`\#=\other
\catcode`\@=0 \catcode`\\=\other

@textfonts
@rm

@c Customize the fonts to 12pt for readability.

@tex

% There is a problem in \doprintindex due to $\leq$ etc...
\global\def\doprintindex#1{\tex %
\catcode`\%=\other\catcode`\&=\other\catcode`\#=\other
\catcode`\@=\other\catcode`\_=\other
\catcode`\~=\other
\indexfonts\rm \tolerance=9500 \advance\baselineskip -1pt
\begindoublecolumns
\openin 1 \jobname.#1s
\ifeof 1 \else \closein 1 \input \jobname.#1s
\fi
\enddoublecolumns
\Etex}


\global\font\twelveit=amti10 scaled \magstep 1 % 12pt
\global\font\twelvesl=amsl10 scaled \magstep 1 % 12pt
\global\font\twelvett=amtt10 scaled \magstep 1 % 12pt
\global\font\twelvesc=amcsc10 scaled \magstep 1 % 12pt


\global\def\textfonts{\let\rm=\twlrm
		      \let\it=\twelveit
		      \let\sl=\twelvesl
		      \let\sc=\twelvesc
		      \let\vf=\sl
		      \let\defbf=\twelvebf	       % in BOTEX
		      \let\bf=\defbf
		      \let\deftt=\twelvett
		      \let\tt=\deftt
		      \let\defsf=\twelvesf	       % in BOTEX
		      \let\sf=\defsf
		      \DEFsf}


% Front page.
\textfonts
\rm

\line{}

\vfill

\centerline{\chaprm LM-Prolog User Manual}
\vskip 2mm
\centerline{\bf For Lambda Release 3.0: \datestamp}

\vfill

\centerline{\it Mats Carlsson$^1$ and Kenneth M. Kahn$^2$}

\vfill

\centerline{$^1$Swedish Institute of Computer Science}
\centerline{P. O. Box 1263}
\centerline{S-164 28  Kista}
\centerline{Sweden}

\vfill

\centerline{$^2$Xerox Palo Alto Research Center}
\centerline{3333 Coyote Hill Road}
\centerline{Palo Alto, CA 94304}
\centerline{U.S.A.}

\vfill

An earlier version of this manual was written by Mats Carlsson and
Kenneth M. Kahn, while at Uppsala Programming Methodology and
Artificial Intelligence Laboratory, Department of Computing Science,
Uppsala University.

The authors wish to thank the National Swedish Board for
Technical Development (STU) and Uppsala Programming Methodology and Artificial
Intelligence Laboratory (UPMAIL), Uppsala University for their support during 
the building of prototypes of LM-Prolog.

\eject

@end tex

@textfonts 
@rm

@defindex cp
@defindex vr
@defindex fn
@defindex kw
@defindex wo
@defindex pr

@settitle LM-Prolog User Manual
@hsize=6.5in
@parindent=0.4in
@itemindent=0.4in
@lispnarrowing=0.4in
@headings on

@chapter Introduction to the System
LM-Prolog is an implementation of Prolog [Kowalski 1974] on Lisp
Machines [Greenblatt 1974].  Prolog is a programming language based
upon a subset of first-order logic that can be computed with
effectively.  The language is used primarily for symbolic computation
and database applications.  It was originally implemented at the
University of Marseille [Roussel 1975] and the most widely used
implementation on DEC 10/20 was developed at the University of
Edinburgh [Warren 1977].

MIT Lisp Machines and their descendants provide a very rich and
powerful environment for developing and running Lisp programs.  They
are personal work stations with large address spaces, large amounts of
memory and disk storage, high resolution graphics, and high-speed
network interfaces.  The idea behind LM-Prolog is to provide the
environment and features of the Lisp Machine to Prolog users.

The @dfn[pure] part of Prolog is a Horn-clause theorem prover.  A Horn
clause is a universally quantified logical implication.  The
consequent part of a Horn clause is an atomic formula; the antecedent
is a conjunction of atomic formulas.  Pure Prolog is not sufficient
for a practical programming language however and several primitive
predicates need to be defined.  These @dfn[evaluable predicates] perform
i/o, efficient arithmetic, database updating, control, and interfacing
to the underlying Lisp system.
@cindex evaluable predicate
@cindex predicate, evaluable

LM-Prolog consists of an interpreter and a compiler from Prolog to Zetalisp.
Both are written in Zetalisp.
The compiled code and the interpreter
communicate with each other automatically.
LM-Prolog features arbitrary arity of predicates,
optional occur check,
handling of cyclic structures, multiple databases, user-controllable
database indexing, efficient conditionals, lazy collections, 
error signalling and recovery, user extensibility,
the ability to reenter Prolog recursively,
turtle graphics, several debugging aids, and various interfaces to Lisp
and the Zwei editor.
The Zetalisp data types such as arbitrarily long integers, rational
and complex numbers, strings, i/o streams, flavors, etc. 
are all easily accessible from LM-Prolog.
LM-Prolog also consists of experimental
facilities such as constraints
and parallel processing via eager bags and sets.

Compiled LM-Prolog programs are about 5 times faster than interpreted. 
Compiler optimizations
include detection of determinacy,  micro-coded unification and
runtime support, tail recursion elimination,
optional open compilation, and a fast interface to Lisp.
For systems without micro-code support unification is open compiled.
Any predicate can be compiled.

The following chapter is a brief introduction to Horn-clause
programming and contains many examples.  Chapter 3 describes
predicates for list processing, predicates for calling Lisp and Lisp
functions for calling LM-Prolog, predicates of general utility,
arithmetical predicates, input-output predicates, and control
primitives.  Chapter 4 is about adding and compiling definitions,
while Chapter 5 describes predicates for retrieving and changing
definitions.  Chapter 6 describes how users can customize the system.
Chapter 7 deals with the debugging and tracing facilities.  Chapter 8
describes various experimental extensions to LM-Prolog that are
subject to further development.  Various ways of interacting with the
top-level of LM-Prolog are presented in Chapter 9.  Chapter 10
discusses briefly how a sophisticated user can extend or modify the
system in Lisp.  Chapter 11 discusses some performance considerations.
Appendix A gives suggested examples of translations from Prolog-10/20
to LM-Prolog.  Appendix B discusses the output of the compiler.
Appendix C contains release notes and a revision history.  Appendix D
contains references.

@chapter Introduction to Pure Prolog
@section Syntax, Terminology, and Packages
A pure Prolog program is a collection of statements in a subset of 
first-order logic.
The programs are executed by a theorem prover which performs a restricted
form of resolution, see [Robinson 1965] and [Lloyd 1984].
One can compute very efficiently with the subset of logic restricted to 
@dfn[Horn clauses].
An LM-Prolog predicate is defined by a list of Horn clauses.
Each clause is a list of @dfn[predications].
A predication is a list of a @dfn[predicator] and some arguments.
A predicator can be any symbol.
The arguments can be any terms (i.e. either atomic, variables, or conses of
two terms).
@cindex Horn clause
@cindex predication
@cindex predicator

Declaratively, a clause is an implication where the first element is
the consequent and the antecedent is the conjunction of the rest of
the elements.  If the rest is empty, the clause simply means that its
first element is true.  All variables are universally quantified so
they need not be explicitly quantified.  They are distinguished from
symbolic constants by beginning with a single `?'.  (Thus `?' has become
a special character and atoms containing it must be written with
``quoting''.) For example, the logical formula
@tex
$$\forall gc \forall gp \forall p:\ \mathop{\rm grandparent}(gp,gc) \leftarrow \mathop{\rm parent}(gp,p) \land \mathop{\rm parent}(p, gc)$$
@end tex
is written in LM-Prolog as @*
@sp 1
@center @code[((grandparent ?gp ?gc) (parent ?gp ?p) (parent ?p ?gc))]
@sp 1

@cindex argument list
Technically speaking, all LM-Prolog predicates and predications
have exactly one argument, the @dfn[argument list]. 
This interpretation is
needed to account for LM-Prolog's variable arity, discussed below.
Thus the above LM-Prolog clause really represents the logical formula
@tex
$$\forall gc \forall gp \forall p:\ \mathop{\rm grandparent}(gp.gc.\mathop{\rm nil}) \leftarrow \mathop{\rm parent}(gp.p.\mathop{\rm nil}) \land \mathop{\rm parent}(p.gc.\mathop{\rm nil})$$
@end tex

For all practical purposes, however, LM-Prolog predicates should be
viewed as taking a list of arguments.

@cindex anonymous variable
@cindex variable, anonymous
@cindex occurrence
A single @code[?] is LM-Prolog's syntax for an @dfn[anonymous variable].  
An anonymous variable is a variable that is defined to @dfn[occur only once]
in a given clause.
For example, @code[((foo ? ?X ?))] should be interpreted as
@tex
$$\forall u \forall x \forall u^{\prime}: \mathop{\rm foo}(u,x,u^{\prime})$$
@end tex

Anonymous variables differ from other variables only in that their
bindings are not displayed in answer to a top-level query.  By
default, the system produces warnings if a non-anonymous variable
occurs only once in a clause.  (This catches many typing mistakes of
variable names.  Variable names beginning with @code[?ignore] may occur
once without receiving a warning.)  This behavior can be changed.  See
the chapter on System Customization.

The primary means of defining predicates is using @code[define-predicate]
which collects together clauses pertaining to the same @dfn[predicator] and 
provides a place to select options as described in detail in @ref[defpred].
@cindex predicator

The syntax of LM-Prolog follows Lisp closely in spirit and in details.
This makes it possible to use all the system software of the Lisp Machine,
such as the Zwei editor, without extra effort.
As with Lisp, the syntax of the language @dfn[per se] is kept simple and other
notations are developed separately as front-ends.

@cindex packages
LM-Prolog defines the packages @code[pglobal:] and @code[prolog:].
The system is written in the @code[prolog:] package, and ``advertised''
predicates, functions, and variables are ``globalized'' into the
@code[pglobal:] package.
Thus those builtin predicates and functions that are documented herein without
a package prefix are available from any package using @code[pglobal:].
Programs written in LM-Prolog should be in a package that
uses @code[pglobal:] and @code[global:].  For convenience, such a package,
@code[puser:], is provided as the default LM-Prolog user package.
A suitable way of making a new package (say, @code[press:]) is:

@lisp
(defpackage press
  (:use pglobal global))
@end lisp

@section Examples
The part of LM-Prolog corresponding to Horn clauses is very powerful.
Several examples of its use follow.  These examples are run in a Prolog
Listener, which is created by typing @system  (@greek[P]).
More details about the top-level of LM-Prolog can be found in 
@ref[toplevel].

@subsection Grandparent Example
The following is the part of a family tree program in LM-Prolog which defines
@code[grandparent], @code[parent], @code[father], and @code[mother].
The first two definitions are complete; the next two may be extended as more
father- or mother-child relationships are encountered.

@lisp
(define-predicate grandparent
  ((grandparent ?grandparent ?grandchild) @ii[;if]
   (parent ?parent ?grandchild) @ii[;and]
   (parent ?grandparent ?parent)))

(define-predicate parent
  ((parent ?parent ?child) @ii[;if]
   (father ?parent ?child)) 
  ((parent ?parent ?child) @ii[;if]
   (mother ?parent ?child)))

(define-predicate father (:options (:type :dynamic))
  ((father Gustaf-Adolf Carl-Gustaf))
  ((father Carl-Gustaf Victoria))
  ((father Carl-Gustaf Carl-Philip))
  ((father Carl-Gustaf Madeleine)))

(define-predicate mother (:options (:type :dynamic))
  ((mother Sibylla Carl-Gustaf))
  ((mother Silvia Victoria))
  ((mother Silvia Carl-Philip))
  ((mother Silvia Madeleine))
@end lisp

@cindex packages
@code[(:options (:type :dynamic))] informs LM-Prolog that programs are
allowed to update the @code[father] and @code[mother] predicates.
The `:' is to indicate that the symbols belong to the Lisp Machine's 
@dfn[keyword] package of symbols.
By default, predicates are not changeable.
@code[Define-predicate] is both a Lisp macro
and a LM-Prolog predicate so it can be
entered at top-level to either system or included in Lisp files.

The predicate @code[grandparent] can be used as illustrated in the following
sample dialog with the system
@setq king section-page

@format
@tt
(grandparent Sibylla Carl-Philip) @ii[;Is Sibylla a grandparent of Carl-Philip?]
Yes, enough answers? (Y or N) Y
(grandparent Gustaf-Adolf ?who) @ii[;Who is a grandchild of Gustaf-Adolf?]
?WHO = VICTORIA @ii[;one answer is Victoria]
Yes, enough answers? (Y or N) N @ii[;want more answers]
?WHO = CARL-PHILIP @ii[;another answer is Carl-Philip]
Yes, enough answers? (Y or N) N
No (more) answers.
(grandparent ?who Madeleine) @ii[;Who is a grandparent of Madeleine?]
?WHO = GUSTAF-ADOLF
Yes, enough answers? (Y or N) Y
(grandparent ?gp ?gc) @ii[;Who is a grandparent of whom?]
?GP = GUSTAF-ADOLF
?GC = VICTORIA
Yes, enough answers? (Y or N) N
...
@end format


@subsection List Processing Example
A classic Prolog program is one defining the relationship of two lists which
concatenate to a third. In LM-Prolog, 
it is generalized to the concatenation of any number of lists.
@code[Append] holds when its first argument is the concatenation of the 
rest of its arguments. It could be defined as follows.

@setq append section-page
@lisp
(define-predicate append
  ((append ()))) @ii[;zero lists concatenate to the empty list]
  ((append ?total () . ?back) @ii[;an empty list can be simplified away]
   (append ?total . ?back))
  ((append (?first . ?total) (?first . ?front) . ?back)
   @ii[;a non-empty list contributes one element to the concatenation]
   (append ?total ?front . ?back)))
@end lisp

@cindex arity, arbitrary
@cindex arbitrary arity
In order to fully exploit the @dfn[arbitrary arity] of LM-Prolog predicates,
the convention is that a function is represented by a predicate whose
first argument is the value of the function application and whose
other arguments correspond to the arguments of the function.
This is unfortunately the opposite of the
convention in other Prolog implementations.
LM-Prolog's convention facilitates using predicates
representing functions of an arbitrary number of arguments.
Essentially any associative function (e.g. plus, times, intersection,
union, concatenation) is well-suited for having an arbitrary number of
arguments.
This provides the same power as APL's reduction operator `/'
[Iverson 1962]
by enabling one to reduce a list by ``applying'' a predicate to it.
The call @code[(append ?list . ?list-of-lists)], for
example, will bind @code[?list] to a list of the elements of each of
the @code[?list-of-lists].
The only restriction on variable
arity is that compiled calls (predications) may not end with a
``dotted pair'' that ends with an unbound variable.

Arbitrary arity is not only very convenient for the user but it often
decreases the need to be concerned with run-time efficiency.
Consider the situation where one wants to concatenate three lists.
An experienced Prolog programmer, especially one thinking ``declaratively'',
might code this as 
@code[(and (append ?ab ?a ?b) (append ?abc ?ab ?c))]
while the logically equivalent coding
@code[(and (append ?bc ?b ?c) (append ?abc ?a ?bc))]
performs proportional to the length of @code[?a] fewer operations.
@code[Append] is declaratively associative, 
but it is not associative with respect to computational complexity.
The cognitive burden of thinking about such issues is 
alleviated by building the
``correct'' coding into the implementation of an arbitrary arity @code[append].

The power of Prolog means that a simple definition of concatenation
as given above can be used in so many different ways.
For efficiency or clarity some of the uses that follow should be 
specialized and named [Kahn and Carlsson 1984].

@description
@item @code[(append ?list ?x ?difference)]
subtracts the list @code[?x] from the beginning of @code[?list] to obtain 
@code[?difference].

@item @code[(append ?list ?difference ?x)]
subtracts the list @code[?x] from the end of @code[?list] to obtain @code[?difference].

@item @code[(append ?list ?x ?y)]
generates pairs of lists @code[?x] @code[?y] which append to @code[?list].

@item @code[(append ?list ? (?term) ?)]
decides whether @code[?term] is a member of @code[?list].

@item @code[(append ?list ? (?x) ?)]
generates elements of @code[?list] called @code[?x].

@item @code[(append ?list ? (?x))]
finds @code[?x] which is the last element of @code[?list].

@item @code[(append ?list ? (?term-1 ?term-2) ?)]
asks if @code[?term-1] and @code[?term-2] are successive elements of @code[?list].

@item @code[(append ?a-list ? ((?key ?value)) ?)]
asks if @code[?value] is associated with @code[?key] in @code[?a-list].

@item @code[(append ?list ? ?sub ?)]
asks if @code[?sub] is a sub-list of @code[?list].

@item @code[(append ?list ? (?term) ? (?term) ?)]
asks if @code[?term] occurs twice in @code[?list].

@item @code[(and (append ?list ?before (?term) ?after) (append ?remaining ?before ?after))]
removes @code[?term] from @code[?list] to obtain @code[?remaining].

@item @code[(append ?list ?a ?b)]
where @code[?list], @code[?a], and @code[?b] are
bound will answer the question of whether @code[?a] and @code[?b] append to 
@code[?list].
It will answer the question efficiently without any consing or searching.

And so on.
@end description

@subsection Peano Arithmetic Examples
@setq peano section-page

Another classic example of logic programming is the Peano definition of
arithmetic in terms of successor.
In the following example, the constructor @code[1+] represents the successor
function.
The predicate @code[plus] is a relation which holds when its first argument is
the sum of its other arguments.
The predicate @code[times] is a relation which holds when its first argument is
the product of its other arguments.
The predicate @code[factorial] is a relation which holds when its 
first argument is the factorial of its second.

@lisp
(define-predicate plus
  @ii[;Zero is the sum of zero numbers]
  ((plus 0)) 
  @ii[;A zero term is simplified away]
  ((plus ?sum 0 . ?addends) 
   (plus ?sum . ?addends)) 
  @ii[;A nonzero term contributes to the sum]
  ((plus (1+ ?sum) (1+ ?x) . ?addends)
   (plus ?sum ?x . ?addends)))
@end lisp

@lisp
(define-predicate times
  @ii[;One is the product of zero numbers]
  ((times (1+ 0))) 
  @ii[;A zero factor makes a zero product]
  ((times 0 0 . ?)) 
  @ii[;(1+x)*y is (x*y)+y]
  ((times ?product (1+ ?x-1) . ?multiplicands) 
   (times ?product-m . ?multiplicands)
   (times ?product-y ?x-1 ?product-m)
   (plus ?product ?product-y ?product-m)))
@end lisp

@lisp
(define-predicate factorial
  @ii[;1 is the factorial of 0]
  ((factorial (1+ 0) 0)) 
  @ii[;the factorial of n is n times the factorial of n-1]
  ((factorial ?factorial (1+ ?n-1))    
   (factorial ?factorial-of-n-1 ?n-1) 
   (times ?factorial (1+ ?n-1) ?factorial-of-n-1)))
@end lisp

@code[Factorial] can be used to compute factorials, e.g. 
@code[(factorial ?f (1+ (1+ (1+ 0))))], compute inverse factorials
(if they exist, otherwise it will not terminate), e.g.
@code[(factorial (1+ (1+ (1+ (1+ (1+ (1+ 0)))))) ?n)],
or generate pairs of integers
such that one is the factorial of the other, @code[(factorial ?f ?n)]. 
It can even answer the question ``are there any factorials greater than four'',
i.e. @code[(factorial (1+ (1+ (1+ (1+ ?x)))) ?n)].

@subsection Database Query Example
@setq dbquery section-page
[Warren 1977] contains a data base query example.
It contains
facts about areas and populations of some countries, and a rule for computing
the density of a country.
In this example, the primitive predicates 
@code[product], @code[quotient], < and >,
to be described later, are used.
These are used for arithmetic and could be replaced by less efficient
predicates written directly in pure Prolog.

@lisp
(define-predicate density
  @ii[;Density in 1000 people per square mile.]
  ((density ?density ?country)
   (population ?population ?country)
   (area ?area ?country)
   (quotient ?density ?population ?area)))
@end lisp

@lisp
(define-predicate population
  @ii[;Population in millions.]
  ((population 825.0 china))
  ((population 586.3 india))
  ((population 252.1 ussr))
  ((population 211.9 usa))
  ((population 127.6 indonesia))
  @ii[;and so on...]
  )
@end lisp

@lisp
(define-predicate area
  @ii[;Area in 1000 square miles.]
  ((area 3380 china))
  ((area 1139 india))
  ((area 8708 ussr))
  ((area 3609 usa))
  ((area 570 indonesia))
  @ii[;and so on...]
  )
@end lisp

It could be used for answering queries like

@description
@item @t[(density ?density USA)]
What is the density of the USA?

@item @t[(and (population ?pop ?country) ( ?pop 500))]
What countries have half a billion people or more?

@end description

The following predicate finds pairs of countries whose densities are
within 5% of each other.

@lisp
(define-predicate similar-density
  ((similar-density (?country-1 ?country-2))
   (density ?density-1 ?country-1)
   (density ?density-2 ?country-2)
   ( ?country-1 ?country-2)
   (product ?density-2-more 1.05 ?density-2)
   ( ?density-2 ?density-1 ?density-2-more)))
@end lisp

@section The Execution of Pure Prolog
When the user types in a predication to Prolog, Prolog tries to @dfn[prove]
the predication using its database.  (Formally, it tries to refute the
predication's negation.)
The predication becomes the @dfn[goal].
@cindex goal

Prolog achieves a goal by searching its database of Horn clauses for ones
whose consequent unifies with the goal, i.e. whose head matches the goal.
The body of the first of these clauses then becomes a new, conjunctive
goal and is achieved recursively.
If this fails, backtracking to the next matching clause occurs.
If the predicate of a goal is built-in then the program associated with that
predicate takes over.
One can depend upon the fact that Prolog will go through the clauses in the
order in which they are defined in the database (by default in LM-Prolog
that is the order of the clauses inside the @code[define-predicate]).
The order in which the elements of the body of a clause will be considered is
the same as they appear within the clause.
For a more detailed discussion of the underlying deduction mechanism of Prolog
see [Kowalski 1974] and [Lloyd 1984].

Prolog programs are sometimes said to have both a @dfn[declarative] and a
@dfn[procedural] reading.
Consider the clause defining @code[grandparent] in terms of @code[parent]:
@cindex declarative reading
@cindex reading, declarative
@cindex procedural reading
@cindex reading, procedural

@lisp
((grandparent ?grandparent ?grandchild)
 (parent ?parent ?grandchild)
 (parent ?grandparent ?parent))
@end lisp

This clause can be read declaratively as
``for all @code[grandparent] and @code[grandchild]
the relation @code[grandparent] holds between them if there exists a
@code[parent] such that the relation @code[parent] holds between it and
@code[grandchild] and the relation @code[parent] holds between the
@code[grandparent] and it''.
The clause can be read procedurally as
``if your goal is to determine if the @code[grandparent] relation holds between
@code[grandparent] and @code[grandchild] then replace the goal by two new
goals: determine the relation @code[parent]
between @code[parent] and @code[grandchild] and determine the relation
@code[parent] between @code[grandparent] and @code[parent]''.

@chapter The Predicates
This chapter contains most of the predefined predicates of LM-Prolog.
Some of these are written completely in LM-Prolog and are provided because
of their general usefulness.
Others interface to Lisp to provide efficient arithmetic, i/o, and the
like.
The control predicates extend the language beyond the left-to-right
depth-first control regime of Pure Prolog.

The convention for describing the arguments of predicates follows closely
the Lisp Machine convention.
The keyword @code[&rest] indicates that the following variable represent
the rest of the caller's argument list.
The keyword @code[&optional] indicates that the arguments that follow need
not be given.
If any of the arguments that follow @code[&optional] are a list of two elements
then the first element is the variable name and the second describes the
default value.
Additionally, if a variable name begins with `+' then it must be
a term or a bound
variable, if it begins with `-' then it must be unbound, otherwise
it can be either.
@section List Processing Predicates
@defpredicate append ?appended &rest +parts
This predicate is discussed in @ref[append]. @i[?Appended] is the result of
appending the elements of @i[?parts].
@end defpredicate

@defpredicate reverse ?reversed ?list
This predicate holds for lists such that @i[?reversed] 
is @i[?list] reversed.  It happens to be written to run more efficiently
if @i[?list] is the ``input'' and @i[?reversed] is the ``output''
argument.
@end defpredicate

@defpredicate length ?length ?list
This predicate is true when @i[?list] is a @i[?length] long list.
@end defpredicate
@nopara
The goal @code[(length ?n (a b c))] will succeed and bind @code[?n] to
3.
The goal @code[(length 3 ?l)] will succeed and bind @code[?l] to
@code[(?A ?B ?C)].
It could be defined as

@lisp
(define-predicate length
  ((length 0 ()))
  ((length ?n (? . ?rest))
   (length ?n-1 ?rest)
   (sum ?n ?n-1 1)))
@end lisp

@defpredicate nth ?nth ?n ?list
This predicate says that @i[?nth] is the @i[?n]th element of
@i[?list].  Zero indexing is used.
@end defpredicate
@nopara
It can be used backwards to find the position of something in a list.
For example, the goal @code[(nth i ?position (m i s s i s s i p p i))]
will first bind @code[?position] to 1, then 4 then 7 then 10.

@defpredicate member ?x ?list
This predicate is true if @i[?x] is a member of the list @i[?list].
@end defpredicate

@defpredicate delete ?left ?out +list
This predicate is true if @i[?left] is @i[+list] with @i[?out] spliced 
out.
@end defpredicate

@defpredicate remove ?left ?out +list
@setq remove section-page
This predicate is true if @i[?left] is those elements of @i[+list] 
which do not match @i[?out].
@end defpredicate

@defpredicate intersection ?intersection &rest +lists
This predicate holds when @i[?intersection] is the intersection of
the @i[+lists].
@end defpredicate
@nopara
It could be defined as

@lisp
(define-predicate intersection
  @ii[;The intersection of no lists is the empty list.]
  ((intersection ()))
  @ii[;The intersection of an empty list and others is the empty list.]
  ((intersection () () . ?))
  @ii[;The intersection of a non-empty list L and some lists Ls,]
  ((intersection ?intersection (?first . ?rest) . ?lists)
   (cases (@ii[;where the first element of L is a member of Ls,]
           (member-of-each ?first . ?lists)           
           (intersection ?rest-intersection ?rest . ?lists)
           @ii[;includes that element,]
           (= ?intersection (?first . ?rest-intersection)))
          (@ii[;otherwise the element is not included.]
           (intersection ?intersection ?rest . ?lists)))))

(define-predicate member-of-each
  ((member-of-each ?))
  ((member-of-each ?x ?first . ?rest)
   (member ?x ?first)
   (member-of-each ?x . ?rest)))
@end lisp

@code[Cases] is an efficient conditional construct, described in @ref[cases].

@defpredicate union ?union &rest +lists
This predicate holds when @i[?union] is the union of
the @i[+lists].
@end defpredicate

@defpredicate sort ?sorted-list +unsorted-list +predicator
@code[Sort] holds if @i[?sorted-list] is a permutation of 
@i[+unsorted-list] such that @i[+predicator] holds between adjacent
elements of @i[?sorted-list].
@end defpredicate
@nopara
@i[+predicator] should correspond to a binary predicate defining a
total order over the elements of the list. @code[standard-order],
discussed in @ref[stord], is a suitable predicate.

@defpredicate substitute ?substituted ?old ?new ?term &optional (+predicate identical)
@i[?substituted] is same as @i[?term] except that all occurrences of
a term @code[?x] for which @code[(?predicate ?x ?old)] holds inside 
are replaced by @i[?new].
@end defpredicate

For example, the goal @code[(substitute ?new-alist (apple ?old-color)
(apple brown) ?alist =)] will replace the old a-list entry of @code[apple] with
one associating it with @code[brown] and find the old entry's association at
the same time.
@code[Substitute] could be defined as follows.

@lisp
(define-predicate substitute
  ((substitute ?s ?old ?new ?form ?test-predicate)
   (cases ((?test-predicate ?form ?old) (= ?s ?new))
          ((and (not-variable ?form) (= ?form (?first . ?rest)))
           (substitute ?first-s ?old ?new ?first)
           (substitute ?rest-s ?old ?new ?rest)
           (= ?s (?first-s . ?rest-s)))
          ((= ?form ?s))))
  ((substitute ?s ?old ?new ?form)
   (substitute ?s ?old ?new ?form identical)))
@end lisp

@section Utility Predicates
@defpredicate same &rest +terms
This succeeds only if all the @i[+terms] unify.
@end defpredicate
@nopara
It could be defined as follows

@lisp
(define-predicate same
  ((same ?x ?x . ?more)
   (same ?x . ?more))
  ((same ?)))
  ((same)))
@end lisp

@defpredicate = ?x ?y
@i[?x] and @i[?y] are unified.
This is equivalent to, but faster than, @code[(same ?x ?y)].
@end defpredicate

@defpredicate identical &rest +terms
@setq id section-page
This tests that all @i[+terms] are @dfn[currently] identical.
@end defpredicate
@nopara
Another way of expressing it is that the most general unifier for
@i[+terms] is the identity substitution.  This will terminate even if
some of the @i[+terms] are cyclic list structures, provided the
@code[circularity-mode] is @code[:handle].  Cyclic structures are discussed
in more detail in @ref[circ].

@defpredicate not-identical ?x ?y
This tests that @i[?x] and @i[?y] are not currently identical.
@end defpredicate


@defpredicate atomic ?x
Tests that @i[?x] is not currently a list.
@end defpredicate

@defpredicate symbol ?x
Tests that @i[?x] is currently a symbol.
@end defpredicate

@defpredicate number ?x
Tests that @i[?x] is currently a number, e.g. a fixnum, a bignum, a flonum,
a rational or complex number, etc.
@end defpredicate


@defpredicate variable ?term
This is true only if @i[?term] is currently unbound.
@end defpredicate


@defpredicate not-variable ?term
This is true only if @i[?term] is currently bound.
Note that it may be bound to a list containing variables.
@end defpredicate


@defpredicate variables ?variables ?term
This is true only if @i[?variables]
is a list of the currently unbound variables in @i[?term].
There are no duplicates in @i[?variables]
and the order of its elements is implementation-dependent.
@end defpredicate

@defpredicate ground ?term
This is true only if @i[?term] is currently variable-free.
It is equivalent to @code[(variables () ?term)].
@end defpredicate


@defpredicate finite ?term
This is true only if there are no cycles inside of @i[?term].
@end defpredicate
@nopara
See the description of @code[circularity-mode] in @ref[circ].

@defpredicate infinite ?term
This is true only if there are cycles inside of @i[?term].
@end defpredicate

See the description of @code[circularity-mode] in @ref[circ].

@defpredicate generate-name ?new-name +old-name
This generates a unique interned symbol whose print name begins with
@i[+old-name].
@i[+old-name] can be any kind of term.
@end defpredicate


@defpredicate generate-symbol ?new-name +old-name
This generates a unique un-interned symbol whose print name begins with
@i[+old-name].
@i[+old-name] can be any kind of term.
@end defpredicate

@defpredicate exploden ?integers ?symbol
This predicate is true if the list @i[?integers] contains the ASCII
codes for the characters in the print name of @i[?symbol].
At least one of its arguments should be bound.
@end defpredicate
@nopara
This is equivalent to Prolog-10/20's @code[name] predicate.


@subsection Comparison of Terms
LM-Prolog has a standard total ordering of terms, which is defined as follows:
First come variables, ordered ``oldest first''. Then come other terms,
ordered first alphabetically by datatype name defined by @code[lisp-type].
Within datatypes, the following rules apply:
Conses are ordered by their two arguments.
Numbers are ordered by standard arithmetic relations.
Strings and symbols are ordered alphabetically by their printnames.

@defpredicate standard-order ?term1 ?term2
@setq stord section-page
This holds if @i[?term1] is not after @i[?term2] in the standard order.
@end defpredicate

@defpredicate compare ?op ?term1 ?term2
This succeeds if the result of comparing @i[?term1] with @i[?term2]
is @i[?op], where @i[?op] is:@*
@sp 1
=, if @code[(identical @i(?term1) @i(?term2))] holds,@*
<, if @i[?term1] is before @i[?term2] in the standard order,@*
>, if @i[?term1] is after @i[?term2] in the standard order.
@end defpredicate

@section Interface to Lisp
@subsection Calling Lisp from LM-Prolog
This section describes ways of calling Lisp functions from Prolog, how to
pass Prolog arguments to Lisp, and how to pass values of Lisp functions 
back to Prolog.  This involves the issue of how Prolog terms are represented
and stored internally.

@cindex datatype
When a Lisp function is called, its arguments sometimes need to be translated 
from ``Prolog form'' to ``Lisp form''.  The main purpose of the translation
is to deal with any logical variables occurring inside the arguments.
The translation is not always needed and is under user control by a keyword
which is added to the three predicates described below.  If the keyword is 
omitted, no translation happens.

@cindex source variable
@cindex variable, source
The translation maps logical variables into instances of the flavor
@code[prolog:source-variable], which are named consistently throughout the
computation (see the @code[read] predicate).  The `?' read macro also creates
such flavor instances.  The translated form will not change behind the
user's back if backtracking occurs, as the untranslated form can.
Thus it is safe to omit the translation only if the arguments are known
to be ground and are not stored away as a side effect.
Logical variables are represented as locatives, and so ``Lisp locatives''
are ``boxed'' in Prolog form to keep them distinct from logical variables.

The implementation also introduces certain datatypes for Prolog variables, for
@dfn[lazy collections] (see @code[lazy-bag-of] and @code[lazy-set-of]), 
@dfn[eager collections] 
(see @code[prolog:eager-bag-of] and @code[prolog:eager-set-of]), 
and @dfn[constrained variables] (see @code[prolog:constrain] and @code[prolog:freeze]).
Lazy and eager collections and constraints are collectively called
@dfn[special terms].  Special terms should normally not be passed to Lisp
without conversion.
@setq dvm section-page
@cindex lazy collection
@cindex collection, lazy
@cindex eager collection
@cindex collecation, eager
@cindex constrained variable
@cindex variable, constrained
@cindex special term
@cindex term, special

The available keywords are:

@description

@kitem :dont-invoke
Boxed locatives are unboxed,
logical variables are translated but special terms are left alone.
Thus lazy collections are not run, eager collections are not waited for,
and constraints are not run.  This mode is good for most purposes.

@kitem :invoke
Special terms are ``invoked'', i.e.
lazy collections are run, eager collections are waited for,
and constraints are woken.  Logical variables are translated.

@kitem :query
Each special term encountered asks the user whether to invoke it or
not.  This mode of interfacing is used e.g. in the Prolog top levels
--- they display answers to user queries, but if the answers contain
lazy collections, the user is queried about how they should be
handled.

@kitem :copy
This mode is equivalent to @code[:invoke] but also ensures that the
translated result is @dfn[ground] i.e. free of (translated) variables.

@end description

The usual way to call to Lisp from Prolog is with one of the three
predicates documented below.  They all take the optional keyword argument.

@defpredicate lisp-command +statement &optional +mode
@i[+statement] is instantiated and evaluated.  This always succeeds.
@end defpredicate
@nopara
An example using @code[lisp-command] follows.

@lisp
(define-predicate clear-window
 ((clear-window ?window) 
  (lisp-command (send '?window ':clear-screen))))
@end lisp

Note the convention of quoting Prolog variables before passing them to Lisp:
this is because the Lisp expression is instantiated before it is evaluated.
Instantiation should be viewed as just substituting variables for their 
bindings. The quote is needed in e.g. @code[(symbolp '?x)] to protect 
@i[?x]'s @dfn[value] from being evaluated.
The compiler arranges so that as much as possible of Lisp expressions like
this are compiled.

@defpredicate lisp-predicate +form &optional +mode
@i[+form] is instantiated and evaluated.
@end defpredicate
@nopara
This succeeds only if the result of the evaluation is non-@code[nil].
One can define a predicate @code[alphaless] which is true if its
arguments are both symbols and the first argument is alphabetically
before the second as follows:


@lisp
(define-predicate alphaless
  ((alphaless ?x ?y)
   (symbol ?x)
   (symbol ?y)
   (lisp-predicate (alphalessp '?x '?y))))
@end lisp

@defpredicate lisp-value ?value +form &optional +mode
@i[+form] is instantiated and evaluated and the result is unified with
@i[?value].
@end defpredicate
@nopara
The predicate @code[product] of any number of arguments can be defined as:

@lisp
(define-predicate product
  ((product ?product . ?arguments)
   (lisp-value ?product (apply 'times '?arguments))))
@end lisp

@code[Lisp-value] can return any kind of Lisp object.  However, the value
is not @dfn[parsed] before unifying with @code[lisp-value].  If the value
contains source variables or locatives, use the following fuction
which does suitable conversions:

@defun prolog:parse-term expression
This translates the Lisp @code[expression] into a Prolog @code[term], by
converting source variables to logical variables, in a context which is
held consistent throughout the computation.  Locatives are converted into
boxed locatives.
@end defun

@defpredicate lisp-type ?type ?term
This unifies @i[?type] with the Lisp type of @i[?term], as returned by the
function @code[type-of].
It fails if @i[?term] is a variable.
@end defpredicate
@nopara
It could be defined as follows:

@lisp
(define-predicate lisp-type
  ((lisp-type ?type ?term)
   (not-variable ?term)
   (lisp-value ?type (type-of '?term))))
@end lisp

If one wishes to write new primitive predicates in Lisp that cannot be written
using the above, see @ref[exten].

@subsection Calling LM-Prolog from Lisp
The following functions are provided for calling LM-Prolog from Lisp.

@defun prolog:make-query predication &optional &key (lisp-term predication)
This function returns an instance of the flavor @code[prolog:query] which when
sent the message @code[:next-answer] will attempt to prove the
@i[predication].
If successful it will multiple value return 
an instantiation of @i[lisp-term] and @code[t].
If it fails it returns @code[nil].
@end defun
@nopara
Sending the instance another @code[:next-answer] will cause it to look for the
next proof of @i[predication].
The message @code[:flush] should be sent to a query instance so that it can
free up resources that it is holding on to.
The @code[prolog:make-query] takes several more optional
keyword arguments but they are not normally needed.


@defun prolog:query-once predication &optional &key (lisp-term predication)
@setq qonce section-page
This function essentially creates a query and sends it the message
@code[:next-answer] and returns the results as described above.
It also sends a @code[:flush] message.
@end defun

@section Arithmetical Predicates
An experimental version of smart arithmetic exists and is described in
@ref[smart].
The following is a simpler, more efficient, implementation which 
severely restricts which arguments can be variables.
Much is missing (trigonometric functions, exponentiation, etc.) 
because it is so
straight-forward to use @code[lisp-value] as in the definition of @code[sum]
below. 
The following predicates require that all but the first argument be numbers.

@defpredicate sum ?sum &rest +numbers
This unifies @i[?sum] with the sum of the @i[+numbers].
@end defpredicate
@nopara
It could be defined as

@lisp
(define-predicate sum
  ((sum ?sum . ?numbers)
   (lisp-value ?sum (apply 'plus '?numbers))))
@end lisp

@defpredicate product ?product &rest +numbers
This unifies @i[?product] with the product of the @i[+numbers].
@end defpredicate


@defpredicate difference ?difference +number +decrement
This unifies @i[?difference] with the @i[+number] minus @i[+decrement].
@end defpredicate

@defpredicate quotient ?quotient +dividend +divisor
This unifies @i[?quotient] with @i[+dividend] divided by @i[+divisor].
@end defpredicate


The following four predicates require that all arguments be numbers.

@defpredicate < &rest +numbers
The list of @i[+numbers] must be in strictly 
ascending order for this to succeed.
@end defpredicate


@defpredicate > &rest +numbers
The list of @i[+numbers] must be in strictly 
descending order for this to succeed.
@end defpredicate

@defpredicate  &rest +numbers
The list of @i[+numbers] must be in ascending order for this to succeed.
@end defpredicate

@defpredicate  &rest +numbers
The list of @i[+numbers] must be in descending order for this to succeed.
@end defpredicate

The above four predicates hold even if given zero arguments, as
opposed to the corresponding Lisp functions.

@section Input-Output Predicates
@setq io section-page
The optional arguments of these predicates are described in 
the Zetalisp manual [Moon @i[et al.@/] 1983].

@cindex global name
@cindex name, global
@defpredicate read ?term &optional +stream +eof-option
@i[?Term] is unified with a term which is read in.  The scope of bindings
for syntactic variables is over the entire computation until backtracking,
including the top-level query.  I.e. the same syntactic variable stands for
the same logical variable every time it is seen, but it is ``forgotten''
if its first occurrence is backtracked over.
@end defpredicate
@nopara
For example, @code[(and (read 1) (read 2))] will fail if the
terms read are both @code[?x].

@defpredicate format ?stream +format-control-string &rest +format-arguments
This provides an interface to the Zetalisp function @code[format].
It always succeeds.  Logical variables are converted
to syntactic variables before printing.
@end defpredicate

@defpredicate write +term &optional +stream
@end defpredicate
@defpredicate princ +term &optional +stream
@end defpredicate
@defpredicate prin1 +term &optional +stream
@end defpredicate
@defpredicate prin1-then-space +term &optional +stream
@end defpredicate
@defpredicate print +term &optional +stream
@end defpredicate
@defpredicate pprint +term &optional +stream
@end defpredicate
@defpredicate write +term &optional +stream
These predicates always succeed.  They apply the corresponding
Zetalisp function to the given arguments.
@end defpredicate


@defpredicate tyi ?ascii-code &optional +stream +eof-option
This unifies @i[?ascii-code] with the code of the next character read,
as returned by the Zetalisp function @code[tyi].
@end defpredicate

@defpredicate tyo +ascii-code &optional +stream
This prints the character with the @i[+ascii-code] by calling the
Zetalisp function @code[tyo].
@end defpredicate

@defpredicate y-or-n +format-control-string &rest +format-arguments
This predicate applies the Zetalisp function @code[format] to the given arguments,
and asks the user to confirm by typing a single character. The characters which
mean success are @t[Y], @t[T], @spacekey, and @handup.
The characters which mean failure are @t[N], @rubout, and @handdown.
@end defpredicate

@defpredicate break
This provides a recursive Prolog toploop.  It is exited by hitting
@resume.
@end defpredicate

@defpredicate open-file ?stream +file-name &rest +options
This predicate opens the file named @i[+file-name] as a stream and unifies
it with @i[?stream].
@end defpredicate
@nopara
A very common option is @code[:direction] to indicate whether the file
is being opened for @code[:output] or @code[:input] (the default).
Refer to the Zetalisp manual for a complete description of the function
@code[open].
It is guaranteed that the stream is closed upon
backtracking or abnormal return.
It could be defined as

@lisp
(define-predicate open-file
  ((open-file ?stream . ?arguments)
   (lisp-value ?stream (apply 'open '?arguments))
   (unwind-protect (true) (lisp-command (close '?stream)))))
@end lisp

@defpredicate load +file-name &rest +options
This predicate always succeeds.
It loads a file by calling the Zetalisp function @code[load].
@end defpredicate

@defpredicate compile-file +file-name &rest +options
This compiles the @i[+file-name] using the Lisp primitive 
@code[qc-file] or @code[compile-file].
@end defpredicate

@defpredicate compile ?predicator &optional ?world ?universe
This ensures that a particular predicate is compiled.
The arguments are as for @code[predicator].
@end defpredicate


@section Control Primitives

@defpredicate true
@end defpredicate
@defpredicate comment &rest +ignore
These predicates succeed.
@end defpredicate

@defpredicate false
@end defpredicate
@defpredicate fail
These predicates fail.
@end defpredicate

@nopara
Unlike some other Prolog implementations, undefined predicates, by default,
signal errors rather than failing, so it is necessary to define @code[false] as
a predicate with no clauses.  See @ref[upm] for how to change the default
behavior.

@defpredicate and &rest +predications
The conjunction of the elements of
@i[+predications] is satisfied in the order presented.
@end defpredicate
@nopara
@code[(And)] is true.  It could be defined as

@lisp
(define-predicate and
  ((and))
  ((and ?goal . ?more-goals)
   ?goal
   (and . ?more-goals)))
@end lisp

@defpredicate prove-once &rest +predications
This succeeds if there is at least one proof of the conjunction of the
@i[+predications].
Only the first proof found will be used.
@end defpredicate

@cindex meta-level predicate
@cindex predicate, meta-level
@code[And] and @code[prove-once] are @dfn[meta-level] predicates i.e. 
their arguments are goals,
rather than terms.  Please note that the syntax for goals allows 
variables anywhere in it.  In the example above, the second clause has two
goals. The first one is just a variable, this means that whatever
goal that variable is bound to at run time will be tried. The
second one is a recursive call to @code[and] with an argument list with
one element less.

@defpredicate or &rest +predications
The @i[+predications] are tried in the order presented.
@end defpredicate
@nopara
@code[(Or)] does not succeed.  It could be defined as

@lisp
(define-predicate or
  ((or ?goal . ?) ?goal)
  ((or ? . ?more-goals) (or . ?more-goals)))
@end lisp

@defpredicate bag-of ?bag ?term &rest +predications
@cindex deterministic predicate
@cindex predicate, deterministic
This predicate generates exactly one proof for a given goal.
It  unifies @i[?bag] with the list created by
instantiating @i[?term] in every proof of the conjunction of the elements of
@i[+predications].
@end defpredicate
The order of the elements of the bag is the same as they are found.
@nopara
For example, the call @code[(bag-of ?b (?x ?y) (append (a b) ?x ?y))]
will bind @code[?b] to @code[((() (a b)) ((a) (b)) ((a b) ()))] which is 
the list of pairs of lists which append to form @code[(a b)].
A common way to execute a goal until it fails is as @code[(bag-of ? ?
@i(predication))].
This compiles well and is clearer than @code[(or (and @i(predication) (false))
(true))].

@defpredicate quantified-bag-of ?bag +quantified ?term &rest +predications
@i[+Quantified] should be a list of variables that are instantiated to some 
values such that @i[+predications] hold, and @i[?bag] is the bag of all
instantiations of @i[?term] such that @i[+predications] still hold.
I.e. this predicate can backtrack to find new values of @i[+quantified]
and @i[?bag].  All values of @i[?bag] are non-empty lists.

@end defpredicate
@nopara
Consider for example the database

@lisp
(define-predicate likes
  ((likes dick beer))
  ((likes harry beer))
  ((likes tom beer))
  ((likes bill cider))
  ((likes jan cider))
  ((likes tom cider)))
@end lisp

@nopara and the conversation

@lisp
(quantified-bag-of ?bag (?beverage) ?who (likes ?who ?beverage))
?BAG = (DICK HARRY TOM)
?BEVERAGE = BEER
Yes, enough answers? (Y or N) N
?BAG = (BILL JAN TOM)
?BEVERAGE = CIDER
Yes, enough answers? (Y or N) N
No (more) answers.
@sp 1
(quantified-bag-of ?bag () ?who (likes ?who ?beverage))
?BAG = (DICK HARRY TOM BILL JAN TOM)
Yes, enough answers? (Y or N) N
No (more) answers.
@end lisp

@defpredicate lazy-bag-of ?bag ?term &rest +predications
This differs from the previous predicate only in the manner in which the bag is
computed. 
The bag is computed only as needed or ``looked at''.
A @code[lazy-bag-of] goal succeeds immediately and @i[?bag] is bound
to a @dfn[lazy value], 
i.e. data structure which computes its elements when necessary to decide a
unification.
@end defpredicate
@cindex lazy value
@cindex value, lazy
@nopara
It is the unification of a delayed value and a non-variable term that
causes the delayed value to compute its value.  Lazy bags when being
unified with non-variable term become either @code[()] or a cons of a
term and a new delayed value.  For example, the list of all factorials
can be described by
@code[(lazy-bag-of ?all-factorials ?factorial (factorial ?factorial ?))].
@code[?All-factorials] behaves indistinguishably from the list of all primes
numbers. 
For example, the goal @code[(member 24 ?all-factorials)] will cause the
first five factorials to be computed before it succeeds.

A classic co-routining problem is the Same Fringe Problem.
Given two binary trees represented as conses and atoms,
we wish to determine whether they have the same leaves. 
We assume the trees are variable-free.
We solve this by constructing for each tree a lazy bag of its leaves and
then unifying the collections.

The entire LM-Prolog program follows: 

@lisp
(define-predicate same-fringe
  ((same-fringe ?tree-1 ?tree-2 ?lazy-or-eager) @ii[;if]
   (lazy-bag-of ?fringe ?leaf (has-leaf ?leaf ?tree-1)) @ii[;and]
   (lazy-bag-of ?fringe ?leaf (has-leaf ?leaf ?tree-2))))

(define-predicate has-leaf
  ((has-leaf ?leaf ?leaf)
   (atomic ?leaf))
  ((has-leaf ?first (?first . ?)) @ii[;if]
   (has-leaf ?leaf ?first))
  ((has-leaf ?leaf (? . ?rest)) @ii[;if]
   (has-leaf ?leaf ?rest)))
@end lisp

In the lazy bag computation, in contrast to the normal sequential
computation, as soon as a difference is detected the predicate
@code[same-fringe] will fail and unnecessary computation involving the
rest of the two trees will have been avoided.  This program depends
upon the fact that the bag contains duplicates and the order of its
elements are those produced by a depth-first search.

Another use of lazy bags is for implementing input streams.
Consider the following program for creating a list containing all the
characters in a file.

@lisp
(define-predicate file-characters
  ((file-characters ?stream ?file-name)
   (open-file ?lisp-stream ?file-name)
   (lazy-bag-of ?stream
                ?character
                (tyi-until-eof ?character ?lisp-stream))))
@end lisp

@lisp
(define-predicate tyi-until-eof
  ((tyi-until-eof ?character ?stream)
   (tyi ?char ?stream -1) @ii[;binds @t(?char) to @t(-1) if end of file.]
   ( ?char 0)
   (or (= ?char ?character) (tyi-until-eof ?character ?stream))))
@end lisp

An experimental parallel processing package described in @ref[exper] provides
versions of @code[bag-of] and @code[set-of]
that compute their elements in independent parallel processes.
For a detailed discussion of lazy and eager collections see [Kahn 1983a].

@defpredicate set-of ?set ?term &rest +predications
This differs from @code[bag-of]
only in that duplicates in @i[?set] are removed.
Two elements are considered duplicates if the relation @code[identical] 
(@ref[id]) holds between them.
@end defpredicate
@nopara
For example, the intersection of the elements of two lists can be computed by 
@code[(set-of ?intersection ?x (member ?x ?list-1) (member ?x ?list-2))].

@defpredicate quantified-set-of ?set +quantified ?term &rest +predications
This is just as @code[quantified-bag-of] but copies are suppressed in @i[?set] and
@i[?set] is in the standard order as defined in @ref[stord].
@end defpredicate

@defpredicate lazy-set-of ?set ?term &rest +predications
This is the ``lazy'' version of @code[set-of].
It differs from @code[lazy-bag-of] in that duplicates are eliminated.
@end defpredicate

@defpredicate cannot-prove &rest +predications
This is true only if the conjunction
of the elements of @i[+predications] cannot be achieved.
No bindings are created.
@end defpredicate


@defpredicate can-prove &rest +predications
This is the same as @code[prove-once] except that no bindings are made.
When one is interested only in the truth value of a goal or conjunction 
of goals then this predicate prevents the system from upon backtracking 
looking for other proofs of the goals.
@end defpredicate

@defpredicate cases &rest +alternatives
@setq cases section-page
This is equivalent to the conjunction of the elements of the first alternative
whose first element is provable.
If there is no such clause it is false.
The alternatives are considered mutually exclusive.
@end defpredicate
@nopara
For example, the predicate @code[maximum] could be defined using @code[cases]
as:

@lisp
(define-predicate maximum
  ((maximum ?max ?x ?y)
   (cases ((> ?x ?y) (= ?max ?x))
          ((= ?max ?y)))))
@end lisp


The call to @code[cases] above can be viewed as short-hand for

@lisp
(or (and (> ?x ?y) (= ?max ?x))
    (and (cannot-prove (> ?x ?y)) (= ?max ?y)))
@end lisp

The important difference is that @code[cases] avoids the recomputation of the
goal @code[(> ?x ?y)].
@code[Cases] is due to IC-Prolog [Clark and McCabe 1982a].

@defpredicate if +test +then &optional +else
@code[If] is syntactic variant of @code[Cases].
@code[Maximum] could be defined using @code[if] as

@lisp
(define-predicate maximum
  ((maximum ?max ?x ?y)
   (if (> ?x ?y) (= ?max ?x) (= ?max ?y))))
@end lisp

@code[If] could be defined as
@lisp
(define-predicate if
  ((if ?test ?then ?else) (cases (?test ?then) (?else)))
  ((if ?test ?then) (cases (?test ?then) ((true)))))
@end lisp
@end defpredicate

@defpredicate either &rest +predications
This finds the first provable goal and ultimately finds all
proofs of only it.  Contrast this with @code[or] which ultimately finds all
proofs of all its arguments.
@end defpredicate

@defpredicate rules ?term &rest +rules
Each rule is of the form @code[(?term-n . ?conditions-n)].
@code[Rules] succeeds with
the first rule for which @i[?term-i] unifies with @i[?term] and the
conjunction of @i[conditions-i] are satisfiable.
It fails if there is no such rule.
@end defpredicate
@nopara
@code[Maximum] could be defined using @code[rules] as

@lisp
(define-predicate maximum
  ((maximum ?max ?x ?y)
   (rules ?max
          (?x (> ?x ?y))
          (?y))))
@end lisp


@code[Rules] could be defined as

@lisp
(define-predicate rules
  ((rules ?term (?term-1 . ?conditions-1) . ?more-clauses)
   (either (and (= ?term ?term-1) . ?conditions-1)
           (rules ?term . ?more-clauses))))
@end lisp

Note that if @code[either] were replaced by @code[or] this would correspond to
the resolution of the @i[?term] against the clauses.

@defpredicate unwind-protect +predication &rest +undo-predications
This is logically equivalent to @*
@code[(or +predication (and (and . +undo-predications) (false)))]@*
The system guarantees however that the @i[+undo-predications]
will be attempted even in the case of
abnormal return due to the use of @code[cut],
lazy collections, or the user hitting @abort.
@end defpredicate
@nopara
The @i[+undo-predications] usually undo some side-effect done by the @i[+predication].
For example, the predicate @code[assume] which
adds a clause to the database and
retracts it upon backtracking could be defined as:

@lisp
(define-predicate assume
  ((assume ?clause)
   (generate-symbol ?clause-name clause)
   (unwind-protect (assert (?clause-name . ?clause))
                   (retract (?clause-name . ?clause)))))
@end lisp

The optional naming of clauses is described in @ref[clname].

@defpredicate map +predicator &rest +list-of-lists
This applies the @i[+predicator] to successive elements in each of the
@i[+list-of-lists].
@end defpredicate
@nopara
This predicate is due to Prolog/KR [Nakashima 1982].
Some examples follow:

@lisp
(map sum ?x (1 2 3) (9 12 15) (-2 -1 0)) @ii[;binds @t(?x) to @t"(8 13 18)"]

(map > (6 7 8) (1 2 99)) @ii[;fails since 8 is not greater than 99]
@end lisp

@defpredicate cut
This is provided to aid efficiency and for compatibility with other
Prolog systems.  It succeeds.  As a side-effect, it removes all
backtracking alternatives introduced from the invocation of the
predicate in which it occurs.  In other words, if a @code[cut] is
encountered in a clause then upon backtracking the goal which invoked
the clause will fail even if there are more possibilities within the
clause or if there are more clauses to consider.

The scope of a @code[cut] is lexical.
@cindex scope
@end defpredicate
@nopara
The most common use of @code[cut] is to declare that only some solutions are of
interest.  In particular, if placed @dfn[before] the @dfn[last] goal of a clause,
it commits the execution to the final, tail-recursive call, which compiles
well.  See also the description of @code[prove-once] above.

Note that this interpretation of @code[cut] makes its behavior very peculiar if
it occurs inside of a collection (@code[bag-of], @code[set-of], etc.) and such
uses should be avoided.

Compare the standard definition of @code[cannot-prove]
(called @code[not] or @code[\+] in other Prolog implementations) using
@code[cut]

@lisp
(define-predicate cannot-prove
  ((cannot-prove . ?goals) 
   (and . ?goals) 
   (cut) 
   (fail))
  ((cannot-prove . ?)))
@end lisp

and the definition using @code[lazy-bag-of]

@lisp
(define-predicate cannot-prove
  ((cannot-prove . ?goals) 
   (lazy-bag-of () ? . ?goals)))
@end lisp

These two implementations of @code[cannot-prove] compile to equivalent code.

One common use of @code[cut] is to avoid recomputing predicates.
Consider the following definition of @code[Maximum] using no control
primitives.

@lisp
(define-predicate maximum
  ((maximum ?x ?x ?y) (> ?x ?y))
  ((maximum ?y ?x ?y) ( ?x ?y)))
@end lisp

It is common practice to use a @code[cut]
to avoid computing @code[( ?x ?y)] as follows.

@lisp
(define-predicate maximum
  ((maximum ?x ?x ?y) (> ?x ?y) (cut))
  ((maximum ?y ?x ?y)))
@end lisp

Compare this definition to those using @code[cases] and @code[rules] presented
earlier.

@defpredicate repeat &optional +n
@code[Repeat] succeeds @i[+n] times. If @i[+n] isn't given, it succeeds
forever.
@end defpredicate

@section Lambda Expressions

It is possible to create and refer to a clause as an object, rather
than referring to a predicate by its name.  This is useful if a rule
is created dynamically and is used only during the scope of the current
computation.  This is analogous to lambda expressions or closures in
Lisp.

@cindex closure
@cindex lambda expression
@cindex expression, lambda
@defpredicate lambda -closure +globals +arg-list &rest +predications
This unifies @i[-closure] with an object which is used exactly as
a predicate with one clause whose argument list and body are
@i[+arg-list] and @i[+predications].  The list @i[+globals] should
contain those variables which are not to be considered quantified
over the closure.  Any other unbound variables occurring inside
@i[+arg-list] and @i[+predications] will denote variables local to
the closure.
@end defpredicate

@nopara
For example, the conjunction
@lisp
(and (lambda ?clos (?y) ((?x ?x ?y)) (print ok))
     (?clos ?l)
     (?clos ?m))
@end lisp

@nopara
will print OK twice and generate the answer
@lisp
?L = (?E ?E ?Y)
?M = (?F ?F ?Y)
?CLOS = #<@plambda ((?A ?A ?Y)) (PRINT OK)>
@end lisp


@chapter Definition of Predicates
@setq defpred section-page
@section Introduction

@cindex world
@cindex universe
A predicate in LM-Prolog is a list of clauses.
A @dfn[world] is a collection of predicates.
The current database is described by the @dfn[universe]
which is a list of worlds.
To find a predicate's definition, LM-Prolog behaves as if it were 
searching linearly down the list of worlds in the current universe.
(It actually searches a data structure on the property list of the
predicator.)

@cindex memoization
Associated with every definition are instructions as what to do if there are
other definitions in later worlds.
By default they are ignored.
This whole process of finding the definition of a predicate is @dfn[memoized]
so that the associated overhead is paid only the first time a predicate is
referred to in a new universe.

Worlds can be used in a static fashion to achieve modularity.
Modularity can also be achieved using the Zetalisp package facility.
The universe can also be changed during a session.  A large system,
for example, can have two or more versions of some module.  Perhaps
one keeps expensive records for later use in providing explanations,
another runs substantially faster, and a third is an experimental
version under development.  It becomes very simple to switch between
these modules if each one has its predicates defined in different
worlds.  The universe can also be changed dynamically during the
program execution.  While this ability can be abused by users, it can
be a powerful programming technique.  The trace package (@ref[trace]),
for example, keeps the traced versions of predicates in one world and
leaves the original definitions untouched.  When trace is enabled the
world with traced predicates is the first in the universe and
essentially masks the originals.  The stepper often ``binds'' the
universe to include or exclude the traced predicates to ``creep'' and
``leap'' respectively.

@section Define-Predicate

@defpredicate define-predicate +predicator (:options . +option-alist) &rest +clauses
@end defpredicate

@defpredicate define-predicate +predicator &rest +clauses 
The principle means of creating and modifying predicates is through the
use of the predicate @code[define-predicate].  Each clause in
@i[+clauses] must have a head whose first element is
@i[+predicator].  They are added to the database in the same order in
which they appear in the @i[+clauses].  The @i[+option-alist]
is a list of options, where each option is a list beginning with an
option name.  All options and values are in the @code[keyword:]
package.  (Remember to put `:' in front of all options.)

@code[Define-predicate] can be used as a Lisp macro as well as a
Prolog predicate.  Consequently files containing predicate definitions
can be loaded, compiled, etc. just as ordinary Lisp files.  If seen in
the context of the Lisp compiler, a predicate definition is first
compiled to Lisp by LM-Prolog's compiler. Later, when the result of
the compilation is loaded, the predicate definition is added to the
database as a compiled predicate.  If executed as a Prolog predicate
or as a Lisp form, the predicate definition is added to the database
as an interpreted predicate.  
@end defpredicate

@section Define-Predicate Options
The processing of each definition is affected by the
@code[define-predicate] options currently in effect.  These come from
three sources, listed in decreasing order of precedence.

@enumerate

@item The @code[:options] part of the definition.

@item Any options that you have included in the File Attribute List. 
The word @code[options] is known to the system as a file attribute and
can be included in the ``-*-'' line of source files. Its value should
be an alist of some of the possibilities listed below.  The Zwei
command @metax[Set Options] is a convenient way of setting this
attribute for a buffer.  File attributes are explained in the Zetalisp
manual.
@cindex file attribute

@item The default options, documented below.

@end enumerate

@defineoption :type +type
The possible values of @i[+type] are @code[:static] and @code[:dynamic].

@code[:Static] is the default and declares that no clauses may be added or removed
from this definition.
A type of @code[:dynamic] means that the predicate is not 
protected against change.
@end defineoption
@nopara
The compiler fully supports dynamic
predicates as well as static ones.
However, optimizations that would not
be feasible for dynamic predicates are performed on static ones.

@defineoption :world +world
@i[+world] should be a symbol and is used to name a collection of predicates,
to which the predicate being defined is added.
@end defineoption
@woindex :user
@woindex :system
@nopara
The default world is called @code[:user].
Most system predicates belong to the world @code[:system].

@defineoption :if-another-definition +action
This option allows one to control what happens if a predicate is defined in
more than one world in the current universe. The value of this option should be
either @code[:before], @code[:after], or @code[:ignore].
At run time, the other definition
will be tried before the current one, after the current one, or not at all, 
respectively.
@code[:Ignore] is the default.
@end defineoption
@nopara
This facility can be used to, for example, ``cache'' the results of
computations.
(The logic programming community sometimes calls this ``adding lemmas''.)
The following predicate will cache its argument in the world called @code[:cache]
which is earlier in the universe than the original predicate and its
traced version.
It uses the trace facility described in @ref[trace].

@lisp
(define-predicate cache-predicate
 ((cache-predicate ?predicator)
  (define-predicate ?predicator
    (:options (:type :dynamic)
              (:world :cache)
	      (:if-another-definition :try-after)))
  (trace ?predicator
         (:print-if) @ii[;don't print anything when called.]
         (:after cache-predication))))
@end lisp

@lisp
(define-predicate cache-predication
  ((cache-predication ?predication)
   (assert (?predication) :cache)))
@end lisp

The goal @code[(cache-predicate fibonacci)], for example,
would cause the predicate
@code[fibonacci] to be cached in the world called @code[:cache].

@defineoption :if-old-definition +keep-or-flush
If there already is a predicate of the same name in the same world then
if @i[+keep-or-flush] is @code[:keep] the new clauses are added to the old
predicate, otherwise if it is @code[:flush] then the new definition replaces 
the old one. The default is @code[:flush].
@end defineoption
@nopara
@setq clname section-page
Optionally a clause can have a name which is a symbol in the first position.
The name is very useful in updating and maintaining the database of clauses.
The name is unique within a definition.
The recommended way to have a predicates definition spread out between
different files (or parts of the same file) is by using named clauses and the
@code[(:if-old-definition :keep)] option.
This way the file can be reloaded.
If one of the clauses is edited, then the new version will replace the old.
For example, an predicate called @code[equivalent] which decides if its
arguments are equivalent may be have its definition spread out so that those
clauses dealing with arithmetic are in one file, those with geometric figures
another and so on.
A typical clause could be described as

@lisp
(define-predicate equivalent
  (:options (:if-old-definition :keep))
  (zero-is-plus-identity
    (equivalent (plus ?zero ?x-1) ?x-2)
    (equivalent ?zero 0)
    (equivalent ?x-1 ?x-2))
  (plus-is-commutative
    (equivalent (plus ?x-1 ?y-1) (plus ?y-2 ?x-2))
    (equivalent ?x-1 ?x-2)
    (equivalent ?y-1 ?y-2)))
@end lisp

@defineoption :place-clauses +where
This option can be used together with @code[(:if-old-definition :keep)].
It controls exactly where the new clauses go.  The value @code[:before]
means before previous ones, @code[:after] means after previous ones, and
@code[:anywhere], which is the default, lets LM-Prolog decide.
@end defineoption

@defineoption :indexing-patterns &rest +patterns
@setq indpat section-page
This option allows for indexed database search when goals of the
current definition are to be satisfied.  @i[+Patterns] is a list
of search patterns that can be used by the theorem prover.
@end defineoption
@nopara
As an example consider a predicate @code[area] used as a database of areas
of countries.  It could have an indexing-pattern @code[(area +area -)]
meaning that the first argument may be used as a search key @dfn[if
bound].  It could have another indexing pattern @code[(area - +country)]
which could be used if the second argument happens to be bound.
In general, an indexing pattern specifies a particular component of
the goal to be satisfied.  If the component is bound,
it is used as a hash key for the predicate.  There may still be several
clauses whose heads match the goal. 
In an indexing pattern there should be exactly one component beginning with
a `+' to specify what the key is.

For example, the following definition of @code[father] creates indexes 
such that the system can find the clauses for a given father or son in constant
time.

@lisp
(define-predicate father
  (:options (:type :dynamic)
            (:indexing-patterns (father +father -son)
                                (father -father +son))))
@end lisp

The cost of finding an indexed clause is independent of the number of clauses
in the definition.
For the interpreter, indexing is worthwhile if there are about 50 
clauses, for compiled predicates it depends on the type of predicate.
For static predicates the cost is like searching about 120 
clauses, for dynamic predicates it's like searching 25 clauses or so.
These numbers are largely due to the fact that unification is in micro-code.

@defineoption :deterministic +when
This option is used to declare to the system that certain optimizations are
possible since the goals of the predicate being defined have at most one
solution.
@end defineoption
@nopara
Both the interpreter and compiled code perform significant optimizations if 
the option @code[(:deterministic :always)] is provided.
The compiler can often determine on its own that a predicate is deterministic,
especially if the @code[:argument-list] option described below is provided.
This can be used as an alternative to @code[cut] when one is interested in only
the first solution found for all calls to the predicate.

@defineoption :compile-method &rest +method
This option affects the compilation of calls to the predicate being defined. 
It can take one of the following forms.

@description
@item @code[(:compile-method :closed)]
@kindex :closed
This is the default and means that in the sequel, any calls to
the predicate being defined shall be compiled in the normal way i.e. as a
sort of function call to the ``entrypoint'' function of the predicate.
@cindex closed compilation
@cindex compilation, closed

@item @code[(:compile-method :open)]
@kindex :open
This works like a generalization of the @code[defsubst] Zetalisp feature.
After a definition with this option has been encountered by the compiler, 
any goals of the predicate being defined are @dfn[opened]; 
i.e. they are replaced by the 
result of resolving the goal with the definition of the predicate.
@cindex open compilation
@cindex compilation, open

@item @code[(:compile-method :intrinsic @i(function-name))]
@kindex :intrinsic
This is a handle for having the compiler generate hand-tailored code.
It is discussed in @ref[exten].
@cindex intrinsic compilation
@cindex compilation, intrinsic

@end description
@end defineoption
@nopara
Semantically, there is no difference between open and closed
compilation.  The unification of actual and formal argument lists is
partially evaluated at compile time to win efficiency.  However, in
order for the compiler to compile calls open, the definition of a
predicate to be open coded must precede its use.  If it does not, an
ordinary closed call will be generated, and the code will still run
although somewhat slower.  This compile-method suffers from the same
problem as Zetalisp's @code[defmacro], @code[defsubst] and the like,
namely if you change a definition whose compile-method is open, any
definitions using it have to be recompiled too.  There is an obvious
space-time trade-off that has to be considered: if a definition to be
open-coded is large, each use of it will compile to rather lengthy
code.

Open compilation is useful if you want to define meta-level predicates, 
that is to say,
predicates that take arguments that are being used as goals.  Those
goals would not be compiled as such in closed compilation.
Another useful case for open compilation is when a predicate is used
merely as an alias for another one, perhaps with a different order of
arguments.

@code[true], @code[false], @code[=], @code[sum] and @code[cannot-prove] 
compile open, for instance. Of course, @code[:compile-method :open] must not 
be used in a recursive predicate, or else
the compiler will go on opening goals forever.

@defineoption :argument-list +pattern
It is good practice to declare the argument list, and to give
the arguments mnemonic names. This option
is used for self-documentation of the predicate.
@end defineoption
@nopara
The @csh[A] command in
both the editor and the rubout handler will show the @i[+pattern]
as does the predicate @code[documentation].

This option also affects the compilation of the pattern-matching part
(unification) of the predicate being compiled.  It is used to give the
compiler partial knowledge about possible argument lists. This knowledge is
used to reduce the time consumed by the unification, and the size
of the code for it. More importantly, `+'s help detect mutually exclusive
clauses within a predicate which in turn helps detect that a 
predicate is deterministic.
Deterministic predicates can be compiled to run significantly faster than
non-deterministic predicates.

An argument list is a list of symbols and lists.
A symbol beginning with `+' indicates that the corresponding element must be
bound, a symbol beginning with `-' indicates that it must be unbound, and
other symbols allow anything in the corresponding position.
For `+' and `-', the corresponding element must not be
a special term (see @ref[dvm]).
The key words @code[&rest] and @code[&optional] are also allowed.
They indicate that what follows corresponds to the rest of the arguments.
A list in an argument list (except after @code[&optional]) means that the
corresponding argument must be a list satisfying the constraints described
within the list. A list following @code[&optional] may be 
provided for documentation purposes.

Examples:

@description
@item @code[(+x +y -z)]
means that there are three arguments, the first two are
bound and the third one is unbound.

@item @code[((x +y) +z &rest +)]
means that there are at least two arguments, the first one
being a list of anything and a non-variable element.

@item @code[(&rest anything)]
is the default and means that any argument list is handled.
@end description

If you define a @code[:static] predicate without declaring its argument list,
the compiler will look at the heads of the clauses and compute
a suitable argument list.  This is because Prolog code compiles to Lisp
functions with corresponding argument lists.
@code[(&rest anything)] would become just a Lisp rest argument, with
a fairly expensive unification of it and the heads of the clauses.

If you lie when you use this @code[define-predicate] option,
then calls not satisfying the argument list will either fail or
signal an error.

@defineoption :lisp-macro-name +macro-name
This defines a Lisp macro named @i[+macro-name] which executes
the goal
corresponding to the predicator being defined and the macros arguments.
The call expands to a call to @code[prolog:query-once] (@ref[qonce])
so it does not backtrack.
@end defineoption
@nopara
This is primarily useful for including goals in files that either change
the database or customize the system.
If the predicate @code[assert], for example, had the option
@code[(:lisp-macro-name assert)] 
then @code[assert] could be used at top level in files and to Lisp.

@defineoption :documentation +documentation-string
It is good practice to include a brief documentation of predicates.
This option is used for self-documentation of the predicate.
@end defineoption
@nopara
The @i[+documentation-string] is accessed by the @csh[D] command in 
both the editor and the rubout handler.
It can also be retrieved by a program as 
@code[(option (:documentation ?d) ?predicator)].
The @dfn[predicate] @code[documentation] uses this.

@section Grammar Rules
@setq gram section-page
There has been much work within the Prolog community on devising grammars that
translate straight-forwardly into Prolog programs (Pereira and Warren, 1980).
The LM-Prolog file for defining definite clause grammars has the logical
pathname @code[LMP: LIBRARY; GRAMMAR].
It can also be loaded by executing @code[(demos)] and mousing
@mouseditem[Grammar Kit].
The grammar kit contains a predicate called @code[-->] which transforms its
arguments into an ordinary Prolog clause and adds it to the database:

@defpredicate --> non-terminal &rest constituents
This asserts the grammar rule that the goal @i(non-terminal)
may be built up from the @i(constituents).
Behind the scenes, LM-Prolog adds four implicit arguments to the 
@i(non-terminal), as explained below,  and adds the transformed rule
as a clause.
The way each @i(constituent) is treated depends on its syntax:
@description
@item[(call @i(predication))]
is simply transformed into just @i(predication),

@item[(is-word @i(word) @i(category))]
is transformed into a test that the next terminal symbol is of
the correct @i(category),

@item[(terminal @i(word))]
is transformed into a test that the next terminal symbol is @i(word),

@item[@i(nonterminal)]
Anything else is transformed into a call to a grammar relation.
@end description
@end defpredicate

Consider the following definitions of
the rule that a verb phrase can consist of a verb followed by a
noun phrase, first with, then without, @code[-->].

@lisp
(--> (verb-phrase (vp ?verb ?np))
     (verb ?verb)
     (noun-phrase ?np))
@end lisp

In Prolog-10/20, the above rule would,
modulo Prolog syntax and placement of arguments,
translate into a clause
extending the @code[verb-phrase] predicate which is a relation between
a tree structure and a part of a list of words:

@lisp
(assert ((verb-phrase ?words0 ?words (vp ?v ?np))
         (verb ?words0 ?words1 ?v)
         (noun-phrase ?words1 ?words ?np)))
@end lisp

Our grammar predicates are actually relations between
tree structures, parts of a list of words,
and parts of lists of graphics commands.
The graphics commands can be fed into the Backtracking Turtle program.
Our @code[-->] predicate translates the above rule into the following:

@lisp
(assert
 ((verb-phrase ?words0 ?words
               ((:here ?x ?y ?) 
                (:place-turtle ?x ?y 240.0) . ?coms0) 
               ?coms
               (vp ?v ?np))
  (display-constituent ?coms0 ?coms1 verb)
  (verb ?words0 ?words1
        ?coms1 ((:place-turtle ?x ?y 120.0) . ?coms2)
        ?v)
  (display-constituent ?coms2 ?coms3 verb)
  (noun-phrase ?words1 ?words ?coms3 ?coms ?np)))
@end lisp

@nopara
where the predicate @code[display-constituent] is responsible for generating
graphics commands for displaying a constituent.

@chapter Database Predicates

@defpredicate assert +clause &optional +world
@code[Assert] can only be performed upon predicates of type @code[:dynamic].
It adds a clause to @i[+world].  
@i[+World] defaults to the current defining world.
@end defpredicate
@nopara
This could be defined as follows:

@lisp
(define-predicate assert
  ((assert ?clause)
   (option (:world ?default-world))
   (assert ?clause ?default-world))
  ((assert ?clause ?world)
   (prolog:clause-predicator ?predicator ?clause)
   (define-predicate ?predicator
     (:options (:world ?world)
               (:if-old-definition :keep)
     ?clause)))
@end lisp

@code[Prolog:clause-predicator] is a system internal
predicate that gets the predicator of the head of a clause whether named or
not.

@defpredicate asserta +clause &optional +world
@end defpredicate
@defpredicate assertz +clause &optional +world
These predicates are like @code[assert] with the additional control that
the asserted clause is stored first or last, respectively, in its
definition.
@end defpredicate

@defpredicate assume +clause &optional +world
This is just like @code[assert] but in addition it is guaranteed that 
@i[+clause] is deleted upon backtracking or abnormal return.
@i[+World] defaults to the current defining world.
@end defpredicate
@nopara
See the description of @code[unwind-protect] earlier.

@code[Assume] is typically used for asserting lemmas that are only relevant
to a few predicates of a program.
The alternative would be to pass around those
results as arguments everywhere. 
This is both awkward and exploring the data structure representing new lemmas
is typically much slower than using LM-Prolog's clause indexing.
Of course, the cost of asserting and retracting @i[+clause] must
be taken into account.
These considerations are discussed in @ref[retract].

@code[Assume] is for example used in the trace package (@ref[trace]) to keep
track of indentation levels. The current indentation level is stored in the
predicate @code[*indentation*] which is updated as follows when a predicate
is entered.

@lisp
(define-predicate indentation
  ((indentation ?indentation)
   (prove-once (*indentation* ?indentation))
   (sum ?next-indentation ?indentation 1)
   (assume ((*indentation* ?next-indentation)))))
@end lisp

When a predicate succeeds, the indentation level is decremented by

@lisp
(define-predicate decrement-indentation
  ((decrement-indentation)
   (prove-once (*indentation* ?next-indentation))
   (difference ?indentation ?next-indentation 1)
   (assume ((*indentation* ?indentation)))))
@end lisp

Upon backtracking, it is guaranteed by the @code[assume] mechanism that
the indentation levels are restored.

@defpredicate retract +clause &optional +world
This predicate deletes any clause matching @i[+clause] in @i[+world].
@i[+World] defaults to the current defining world.
It backtracks, trying each time to delete another clause.
@end defpredicate

@defpredicate retract-all +predicator &optional +world
This predicate deletes all clauses of a given predicate in @i[+world].
@i[+World] defaults to the current defining world.
It always succeeds.
@end defpredicate
@nopara
Note that the definition of the predicate is not destroyed.
The tables associated with indexing patterns for example are cleared but are
still available for indexing new clauses.

@defpredicate remove-definition +predicator +world
This predicate does everything that @code[retract-all] does, in addition
it deletes the (now empty) definition itself. This operation is
necessary if a predicate was mistakenly defined either with the wrong
name or in the wrong world, to undo the definition. An empty definition
could mask a definition of the same predicator in a different world.
@end defpredicate

@defpredicate all-worlds ?worlds
@i[?Worlds] are all worlds ever created.
@end defpredicate

@defpredicate universe ?worlds
@i[?Worlds] is the current universe, i.e. the list of current worlds.
@end defpredicate

@defpredicate add-world +world
@i[+World] is removed from the universe if it's already there. It is then 
added to the front of the universe.
@end defpredicate

@defpredicate remove-world +world
@i[+World] is deleted from the current universe.
@end defpredicate

@defpredicate save-world +world +file-name
All definitions in @i[+world] are written onto @i[+file-name].
@end defpredicate
@nopara
This could be defined by

@lisp
(define-predicate save-world
  ((save-world ?world ?file-name)
   (bag-of ? ?
     (open-file ?stream ?file-name :out)
     (format ?stream ";; -*- Mode: Lisp; -*-~%")
     (definition ?definition ? ?world)
     (grind ?definition ?stream))))
@end lisp

@defpredicate destroy-world ?world
This runs @code[remove-definition] on every predicate defined in @i[?world]
and then deletes any reference to @i[?world].
@end defpredicate
@nopara
This should be done if one wishes the garbage collector
to reclaim the space used by a world.

@defpredicate with-world +world &rest +predications
@i[+Predications] are satisfied with @i[+world] temporarily added
to the current universe.
If @i[+predications] calls any predicate that changes the universe the
change will only have the scope of the @code[with-world].
@cindex scope
@end defpredicate
@nopara
This could have been implemented as follows

@lisp
(define-predicate with-world
  ((with-world ?new-world . ?predications)
   (universe ?current-worlds)
   (with (universe (?new-world . ?current-worlds)) 
         . ?predications)))
@end lisp

@code[With] is described in @ref[with].

@defpredicate without-world +world &rest +predications
@i[+Predications] are satisfied with @i[+world] temporarily removed
from the current universe.
@end defpredicate
@nopara
@defpredicate call +predication &optional +world
@code[Call] of one argument is provided for compatibility with other Prologs
and because some people prefer it over the default implicit call of a variable.
@code[Call] is implicit for any goal.
For example, @code[(print 1)], @code[(call (print 1))],
@code[(and (= ?g (print 1)) (call ?g))], @code[(and (= ?g (print 1)) ?g)],
and @code[(and (= ?p print) (?p 1))] are all equivalent.

@code[Call] of two arguments allows one to specify explicitly which world the
definition of the predicator of the @i[+predication] should be found.
@end defpredicate
@nopara
The implementation of @code[trace], for example, uses this.
It creates a version of a traced predicate in a special world called
@code[prolog:traced-predicates] which is in the beginning of the universe.
Inside the new predicate it needs to call the original predicate and does so
using @code[call] of two arguments.
@windex prolog:traced-predicates

@defpredicate predicator ?predicator &optional ?world ?universe
This predicate succeeds if @i[?predicator] is a predicator in @i[?world]
in @i[?universe]. @i[?Universe] defaults to the current universe, and
@i[?world] is a member of @i[?universe].
@end defpredicate

@defpredicate defined-in-world ?world ?predicator
@i[?World] is unified with a world where @i[?predicator] is
defined.
It could be defined as

@lisp
(define-predicate defined-in-world
  ((defined-in-world ?world ?predicator)
   (all-worlds ?worlds)
   (predicator ?predicator ?world ?worlds)))
@end lisp
@end defpredicate

@defpredicate apropos ?predicator +substring &optional ?world ?universe
This predicate is like @code[predicator] with the additional constraint
that @i[?predicator] must contain @i[+substring] in its print name.
@end defpredicate

@defpredicate definition ?definition ?predicator &optional ?world ?universe
This predicate is like @code[predicator] with @i[?definition] being unified
with each definition found.
The @i[?definition]'s option list is the merger of those given when defined
and the defaults at the time of creation.
@end defpredicate

@defpredicate print-definition ?predicator &optional ?world ?universe
This predicate prints out appropriate definition. Only those options
that differ from the current set of defaults are printed.
This backtracks if the definition is under specified.
@end defpredicate
@nopara
For example, @code[(bag-of ? ? (print-definition ?))] will print out all
definitions.

@defpredicate documentation ?predicator &optional ?world ?universe
This predicate prints out the @code[:argument-list] and @code[:documentation]
options in an easy to read format.
@end defpredicate
@nopara
For example, the goal @*
@code[(bag-of ? ? (apropos ?predicator ``world'' ?w) (documentation ?predicator ?w))]@*
will print out the documentation of all predicates whose name contains the
string @code[``world''].

@defpredicate clause ?clause &optional ?world ?universe
This predicate is reminiscent of @code[predicator] but unifies @i[?clause]
with clauses found.
@end defpredicate
@nopara
For example, the goal

@lisp
(and (clause (?head . ?body))
     (member (print . ?arguments) ?body))
@end lisp

will find all clauses that contain a top-level @code[print] goal.
@defpredicate prolog:clauses ?clauses +predication
This unifies @i[?clauses] with those clauses that the system considers
possibly appropriate for the @i[+predication].
It does not unify the @i[+predication] with the heads of the @i[?clauses]
but it does use any indexing patterns provided (see @ref[indpat]).
Except for the indexing patterns it could be defined as

@lisp
(define-predicate clauses
  ((clauses ?clauses (?predicator . ?))
   (definition (define-predicate ?predicator ? . ?clauses)
               ?predicator)))
@end lisp
@end defpredicate

@chapter System Customization
This chapter contains primitives for inspecting and changing the various modes
and global parameters of the system.

@defpredicate option (?option-name . ?values) &optional ?predicator ?world ?universe
If called without any ``optional'' arguments 
this unifies @i[?values]
with the default value for @code[define-predicate] option
@i[?option-name].
@end defpredicate
@nopara
For example, @code[(option (:world ?w))] will bind @code[?w]
to the default world for new definitions.
If the @i[?predicator] is given then the @i[?values]
for that predicator are given.
One can further specify the @i[?predicator]'s @i[?world] or
@i[?universe].

@setq circ section-page
@defpredicate circularity-mode ?mode
@i[?Mode] is unified with the current circularity mode, which is either
@code[:ignore], @code[:handle], or @code[:prevent].
The issue this addresses is that unification is defined to prevent cyclic
bindings such as @code[(= ?x (f ?x))].
This is normally prevented by an ``occur check'' which most Prolog
implementations neglect to make in the interests of efficiency.
LM-Prolog users decide whether they want the occur check or not.
In addition, if the occur check is ignored then the system can handle
cyclic structures.
If the circularity-mode is @code[:handle] then the system will be able to unify,
print, and instantiate cyclic structures.
@end defpredicate
@nopara
The default circularity mode is @code[:ignore].  The circularity mode can
be altered by @code[make-true], @code[let], and @code[with] described below.
A discussion of this issue can be found in [Colmerauer 1982a] and
[Hansson @i[et al.@/] 1982] among others.

@defpredicate undefined-predicate-mode ?mode
@i[?Mode] is unified with the current undefined predicate mode, 
which is either @code[:signal] or @code[:fail], and governs what should
happen if an undefined predicate is called.  If the mode is @code[:signal],
an error is signalled, and the user is prompted for another predicator.
If the mode is @code[:fail], the calls simply fails.
@end defpredicate
@nopara
The default undefined predicate mode is @code[:signal].  The
undefined predicate mode can be altered by @code[make-true], @code[let], and
@code[with] described below.
@setq upm section-page

@defvar prolog:*warn-if-variable-occurs-once*
The system's default behavior is to issue warnings about
non-anonymous variables that only occur once per clause.
If this is felt annoying or unnecessary, it can be turned off
by setting this variable to @code[nil].  The default value is @code[t].
@end defvar


@defpredicate protected-worlds ?worlds
@cindex protected world
@cindex world, protected
This unifies @i[?worlds] with the list of worlds with are @dfn[protected].
If a predicate is defined which would mask the definition of a predicate in
a protected world, then the user is queried first.
The default value is @code[(:system)] so that a user is queried before
redefining a system primitive.
@end defpredicate

@defpredicate make-true &rest +predications
This predicate is an operator on goals.  Rather than having
predicates for changing the @code[circularity-mode], the
@code[undefined-predicate-mode], the default define-predicate @code[option],
the current @code[universe], the handling of @code[interrupts], 
and the @code[protected-worlds], the predicate
@code[make-true] is used.  The goal @code[(circularity-mode
:handle)], for example, asks ``is the circularity mode @code[:handle]''.
The goal @code[(make-true (circularity-mode :handle))] always
succeeds and @dfn[sets] the mode to @code[:handle].  @code[Make-true] also
works for @code[lisp-value] if its second argument is a symbol.
@end defpredicate

@nopara
For example, the goal @code[(make-true (lisp-value 8 *print-base*))]
changes the print base to octal.
If the predicator of the @i[+predication] is a dynamic predicate then the
@i[+predication] is asserted.
@code[Make-true] simply succeeds if its @i[+predication] is true to begin
with. 

@defpredicate with +temporarily-true-predication &rest +predications
@setq with section-page
This is an operator which builds upon the @code[make-true] operator.
The @i[+predications] are proved with the
@i[+temporarily-true-predication] ``made true'' only for the scope of
the @code[with].  @i[+temporarily-true-predication] can be any
goal that @code[make-true] accepts.
@cindex scope
@end defpredicate
@nopara
For example, in a theorem-proving application, to specify that a part of
the computation should be run with occurs check turned on, so as to prevent
the creation of cyclic structures, you could say:

@lisp
(with (circularity-mode :prevent) 
      ...)
@end lisp

@nopara
This is equivalent to

@lisp
(and (circularity-mode ?current-value)
     (unwind-protect (make-true (circularity-mode :prevent))
                     (make-true (circularity-mode ?current-value)))
     ...
     (unwind-protect (make-true (circularity-mode ?current-value))
                     (make-true (circularity-mode :prevent))))
@end lisp

@defpredicate let +temporarily-true-predication
@setq let section-page
This is another operator built upon @code[make-true].
The @i[+temporarily-true-predication] is assumed true for the rest
of the computation, until backtracking past the @code[let].
@i[+temporarily-true-predication] can be any goal that
@code[make-true] accepts.
@cindex scope
@end defpredicate
@nopara
For example, to globally bind a Lisp variable to some value
for the rest of the current computation, use:

@lisp
  ...
  (let (lisp-value <some-value> <some-Lisp-variable>))
  ...
@end lisp

@nopara 
The old value of the Lisp variable will be restored upon backtracking.

@chapter Debugging Facilities
@setq debug section-page
@section Trace Facility
@setq trace section-page
LM-Prolog has a Lisp-like trace facility.  Any predicate, interpreted or
compiled, can be traced.  Tracing compiled predicates doesn't always
work, however, since some predicates compile in-line, and because 
tail-recursive predicates compile to iterative code.  

LM-Prolog also has a facility for single-stepping the execution of Prolog 
programs, pausing at calls to all predicates except calls from
certain compiled predicates to certain compiled predicates.

The execution of Prolog programs is in general more complex than that
of Lisp because of the non-deterministic nature of the language.  We
introduce therefore the following terminology.

@cindex port
An invocation of a goal is said to have four @dfn[ports]. When first
invoked, the @code[trying] port is entered. When a solution is found we exit
through the @code[succeeding] port. Upon backtracking we reenter through the 
@code[re-trying] port.  Eventually we fail, leaving through the @code[failing] port.

The ``ports'' terminology is due to Lawrence Byrd [Byrd 1980].

Invocations are @dfn[numbered].  The number is incremented at
@code[trying] ports and decremented at @code[failing] ports.  This number
is what the user is expected to answer for the @code[Print], @code[Grind], 
and @code[Retry] commands below.  

@defpredicate step +predication 
The execution of @i[+predication] is single-stepped.  The following
step commands are available at every @code[trying] port.  The @code[Retry]
command is particularly useful.  If you find a bug in your program
while stepping through it, you may correct the faulty predicate,
redefine it, and back up the computation to the faulty point.

@description
@item Creep
type @spacekey, @return, or @l[C] to stay in step mode,

@item Leap
type @linefeed or @l[R] to turn off stepping until the current goal
succeeds, is retried, or fails.  Predicates inside are still traced.

@item Skip
type @altmode or @l[S] to turn off stepping and tracing until the current goal
succeeds, is retried, or fails.  

@item Refresh
type @clrscrn to refresh the screen,

@item Menu
type @l[M] to get a menu for setting parameters for trace printouts,

@item Print
type @l[P] to print a goal which is prompted for,

@item Grind
type @l[G] to grind a goal which is prompted for,

@item Fail
type @l[F] to cause the current goal to be FALSE,

@item True
tyep @l[T] to cause the current goal to be TRUE,

@item Retry
type @l[R] to retry a goal from scratch, the goal is prompted for,

@item Ancestors
type @l[A] to display invocation numbers of ancestor goals,

@item Backtrace
type @l[B] to display a backtrace of ancestor goals,

@item Unify
type @l[=] to give a term to unify with the current goal,

@item Break
type @breakkey to get a Prolog Listener recursive toplevel,

@item Abort
type @abort to abort the current query,

@item Edit
type @l[.] (period) to edit the definition of the current goal.

@end description
@end defpredicate

@defpredicate trace +name-or-names &rest +options
@i[+name-or-names] is either a single predicator or a list of predicators
to be traced.

@i[+options] is zero or more of the following.

@description
@item @code[(:print-if &rest +port-names)]
@mykindex :print-if
the default is all four ports 
(@code[:trying], @code[:succeeding], @code[:re-trying], and @code[:failing]). 

@item @code[(:when @r(&rest) +patterns)]
@mykindex :when
the tracing/stepping applies only if the goal unifies with at least 
one of the @i[+patterns].

@item @code[:step]
@mykindex :step
enter step mode when the predicate is entered.

@item @code[(:before +predicator)]
@mykindex :before
@i[+predicator] is called on the goal upon entry.

@item @code[(:after +predicator)]
@mykindex :after
@i[+predicator] is called on the goal after entry.

@item @code[(:world +world)]
@mykindex :world
by default traced predicates are added to a world called 
@code[prolog:traced-predicates]. This option allows one to control 
where they will be put.
@windex prolog:traced-predicates
@end description
@end defpredicate

@defpredicate untrace &optional ?predicator
@end defpredicate
@defpredicate untrace-query
The first form untraces one predicator, backtracking if @i[?predicator]
is unbound. If @i[?predicator] is omitted, all predicates are untraced. 
@code[Untrace-query] is selective
i.e. it asks you for each traced predicate whether to untrace it.
@end defpredicate

@defun prolog:reset-trace
Should the trace facility enter a confused state
due to abnormal exits or interference between processes, this
Lisp function restores things to normal.
@end defun

@defvar prolog:*history*
@end defvar
@defvar prolog:*ancestors*
These variables represent the list of active goals and list of ancestor goals
being traced, respectively.  Actually they are lists of lists whose second
element is the goal.  These can be explored by the adventurous user who wishes
to implement exotic control structures.
@end defvar

@defpredicate who-state +label
This predicate sets the ``who-state'' at the bottom of the Lisp machine console
to @i[+label].  Upon backtracking or abnormal return, the original value
is restored.
@end defpredicate

@section Error Handler
It is possible to fall into the Lisp error handler from Prolog.  For
instance, calling undefined predicators by default causes a Lisp error.
Currently errors can be inspected, given
handlers, or corrected interactively only from Lisp.  A Prolog
interface to these facilities would be preferable.
Currently, the following predicates are provided. See also
@code[unwind-protect] and @code[undefined-predicate-mode].

@defpredicate error +signal-name &rest +signal-arguments
This signals a non-proceedable error of type @code[+signal-name] with
arguments @code[+signal-arguments].
@end defpredicate

@nopara
For instance, aborting to the top level is achieved by:
@lisp
  (error si:abort)
@end lisp

@defpredicate prolog:condition-protect +condition-name +goal &rest +undo-goals
This is logically equivalent to @i[+goal].
If during its execution a @i[condition] named @i[+condition-name] is 
signalled, @i[+goal] is aborted and @i[+undo-goals] are executed instead.
@end defpredicate

@chapter Experimental Facilities
@setq exper section-page
@cindex constrained variable
@cindex variable, constrained
@cindex constraint
LM-Prolog contains facilities which, while still under development
and subject to change, provide an
adventurous user with some powerful extensions to the system.
Until these facilities settle down, the names of the predicates 
in these extensions are in the @code[prolog:] package.
One of the major extensions is the ability to freeze goals so that they are
postponed until more is known.
This facility is good for @dfn[constraining] variables to satisfy certain
predicates.
Another major extension consists of versions of
@code[set-of] and @code[bag-of] that
compute in independent parallel processes.
Lately, parallel execution schemes for Prolog has attracted great
research interest. Among others, members of the Japanese Fifth Generation
Computers project are investigating this, see e.g. 
[Furukawa @i[et al.@/] 1982].

Also included in this chapter is a brief description of the ``demo'' facility
which can be used to load and run various applications and extensions.

@section Constraints and Freezing Goals
@setq freeze section-page
The standard execution model of Prolog is to always execute goals
in the order as they appear in the program.
As a complement to this, LM-Prolog has the ability to selectively delay
the execution of a goal until enough variables in it are bound.
The facility is defined in the file @code[LMP: EXPERIMENTAL; FREEZE] and is
due to Prolog II [Colmerauer 1982b].
The facility has lots of applications in logic programming, e.g.
to implement sound versions of inequality and negation.
It is used in several of the demo programs (@ref[demfac]) to
implement IO streams, to write ``generate and test'' problems in
a clear yet efficient way, to solve equations, and to compute
with infinite lists.

The predicates below provide ways of delaying the execution.
Note however that these predicates do not guarantee nor detect that
all delayed goals are eventually run.  Thus it is the user's responsibility
to do so.  However, they do guarantee that delayed goals are not
woken more than once.

@defpredicate prolog:constrain ?variable +predication
If @i[?variable] is bound then @i[+predication] is executed.
Otherwise @i[?variable] is constrained to satisfy @i[+predication].
If @i[?variable] is subsequently bound when unifying a goal
with the head of a clause then the @i[+predication]
is executed as if it were the first element of the body of that clause.
Currently, the system behaves as if all constraints were deterministic.
@end defpredicate
@nopara
For example, we can constrain a variable to be greater than 7 with
@code[(prolog:constrain ?x (> ?x 7))].
This constrained variable can be used anywhere.
For example, to remove all negative numbers from a list we can use
@code[remove] (@ref[remove]) as follows.

@lisp
(and (prolog:constrain ?x (< ?x 0))
     (remove ?only-positive ?x ?list))
@end lisp

Another rather different use is to connect a list to a physical output
stream such that elements are printed as they become instantiated.
@setq stream section-page

@lisp
(define-predicate output-stream
  ((output-stream () ?)
   (output-stream (?elt . ?rest) ?lisp-stream)
   (lisp-command (send '?lisp-stream ':tyo '?elt))
   (prolog:constrain ?rest (output-stream ?rest ?lisp-stream))))
@end lisp

Yet another use is to compute an infinite list of the Fibonacci numbers 
in a demand-driven fashion:

@lisp
(define-predicate fib
  ((fib (1 1 . ?rest))
   (prolog:constrain ?rest (fib 1 1 ?rest)))
  ((fib ?i ?j (?k . ?rest))
   (sum ?k ?i ?j)
   (prolog:constrain ?rest (fib ?j ?k ?rest))))
@end lisp

The goal @code[(fib ?r)] will instantiate @code[?r] to an infinite
list whose elements are computed as they are needed.


@defpredicate prolog:freeze ?predication &optional (+unbounds 0)
This predicate postpones the execution of @i[+predication] until
at most @i[+unbounds] distinct variables in it are still unbound.
@end defpredicate

@defpredicate not ?predication
Sound negation.
This is equivalent to @code[(cannot-prove @i(?predication))]
but postpones the execution until @i[?predication] is ground.
@end defpredicate

@defpredicate  ?x ?y
Sound inequality.
This is true if @i[?x] and @i[?y] are terms that do not unify.
This is false if @i[?x] and @i[?y] unify without binding any variables.
Otherwise, it delays.
@end defpredicate

@defpredicate prolog:*combine-constraints* -union +predications-1 +predications-2
This predicate is called when two constrained variables are unified,
or when a constrained variable is incrementally constrained.
It may compute one of the following values for @i[-union]:
@code[:fail], @code[:invoke], or a list.
If it fails, the combined constraint is
the union of @i[+predications-1] and @i[+predications-2].
If it computes @i[-union] as a list, 
@i[-union] is considered to be
the new constraints for the two variables involved.
If it computes @i[-union] as @code[:fail],
the unification that caused the invocation of @code[prolog:*combine-constraints*]
fails.
If it computes @i[-union] as @code[:invoke],
the new constraints are not added incrementally but run immediately.
@end defpredicate
@nopara
For example, the following clause simplifies two constraints using >.

@lisp
(define-predicate prolog:*combine-constraints*
  (:options (:if-old-definition :keep))
  (simplification-of-two-greater-thans @ii[;the clause name]
   (prolog:*combine-constraints* ((> ?x ?bigger))
                                 ((> ?x ?smaller))
			         ((> ?x ?bigger)))
   ( ?bigger ?smaller)))
@end lisp

The following clause causes the unification of two incompatible constraints to
fail. 

@lisp
(define-predicate prolog:*combine-constraints*
  (:options (:if-old-definition :keep))
  (incompatible-greater-and-less-thans
   (prolog:*combine-constraints* :fail
            	                 ((< ?x ?smaller))
			         ((> ?x ?bigger)))
   ( ?bigger ?smaller)))
@end lisp

@setq smart section-page
An example of a problem in Prolog which @code[prolog:freeze] solves is that
arithmetic primitives are not general.
LM-Prolog's @code[sum] for example requires that all but its first argument
(the sum itself) be numbers.
This is because it essentially passes the problem along to Lisp's @code[plus]
which requires that its arguments be numbers.
Using @code[prolog:freeze] we can implement a predicate that if too many 
arguments are unknown simply constrains the variables in those arguments.
The following predicate computes the sum relation for any number of arguments
any of which may be unbound.

@lisp
(define-predicate +
  ((+ . ?args)
   (prolog:freeze (solve (+ . ?args)) 1))) @ii[;Freeze until zero or one unknown.]
@end lisp

@lisp
(define-predicate solve
  ((solve (+ ?sum . ?addends))
   @ii[;split @i(?addends) into numbers and occurrences of @i(X)]
   (vars-and-nonvars ?vars ?nonvars ?addends) 
   (cases ((= () ?vars) @ii[;simply compute @i(?sum) if no variables]
           (lisp-value ?sum (apply '+ '?nonvars)))
          ((not-variable ?sum) @ii[;compute @i(X) if @i(?sum) is known]
           (= ?vars (?var . ?)) 
           (length ?length ?vars)
           (lisp-value ?var
                       (// (- '?sum (float (apply '+ '?nonvars)))
		           ?length)))
          ((variable ?sum) @ii[;Here  @i(X = n*X+Y)]
           (lisp-value 0 (apply '+ '?nonvars)) @ii[;@i(Y) must be 0]
           (rules (?sum)
                  (?vars)
                  ((0))) @ii[;@i(X=0) if @i(n>1)]
           ))))
@end lisp

@section Eager Collections
This experimental facility is for doing parallel processing and
is described in detail in [Kahn 1983a] and is defined in the file
@code[LMP: EXPERIMENTAL; EAGER].

@defpredicate prolog:eager-bag-of ?bag ?term &rest +predications
@setq eager section-page
@cindex eager value
@cindex value, eager
This predicate is logically the same as @code[bag-of] and @code[lazy-bag-of].
Operationally it creates a separate Lisp Machine process to compute the
elements of the @i[?bag].
Unification is extended so that if a non-variable term is unified with
an @dfn[eager value] the unification waits for it to complete.
Eager bags are eager values that either become @code[()] or a cons of an
instantiation of @i[?term] and a new eager value.
@end defpredicate
@nopara
The @code[same-fringe] predicate, for example, defined under the
description of @code[lazy-bag-of] could be modified to explore the two
trees in parallel by replacing @code[lazy-bag-of] with @code[prolog:eager-bag-of].
As soon as a difference is detected the processes will automatically
be terminated.

@defpredicate prolog:eager-set-of ?bag ?term &rest +predications
This is the same as @code[prolog:eager-bag-of] described above except 
duplicates are eliminated from @i[?bag].
@end defpredicate

@defpredicate prolog:faster ?term-1 ?term-2
If @i[?term-1] is anything other than an eager value this predicate
succeeds.
If it is an eager value and @i[?term-2] is not then it fails.
If both are eager it waits for one of them to become a non-eager value.
@end defpredicate
@nopara
Using @code[faster] and @code[lazy-bag-of] one can define a parallel version of
@code[or] as follows.

@lisp
(define-predicate parallel-or
  ((parallel-or ?goal-1) ?goal-1)
  ((parallel-or ?goal-1 ?goal-2)
   (eager-bag-of ?bag-1 ?goal-1 ?goal-1)
   (eager-bag-of ?bag-2 ?goal-2 ?goal-2)
   (cases ((faster ?bag-1 ?bag-2)
           (alternate ?goal-1 ?bag-1 ?goal-2 ?bag-2))
          ((alternate ?goal-2 ?bag-2 ?goal-1 ?bag-1))))
  ((parallel-or ?goal-1 ?goal-2 ?goal-3 . ?more-goals)
   (parallel-or (parallel-or ?goal-1 ?goal-2)
                (parallel-or ?goal-3 . ?more-goals))))
@end lisp

@lisp
(define-predicate alternate 
  ((alternate ?goal-1 (?goal-1-instantiated . ?bag-left-1)
              ?goal-2 ?bag-2)
   (or (= ?goal-1 ?goal-1-instantiated)
       (cases ((faster ?bag-2 ?bag-left-1)
               (alternate ?goal-2 ?bag-2 ?goal-1 ?bag-left-1))
              ((alternate ?goal-1 ?bag-left-1 ?goal-2 ?bag-2)))))
  ((alternate ? () ?goal-2 ?bag-2)
   (member ?goal-2 ?bag-2)))
@end lisp

@defpredicate interrupts ?on-or-off
If interrupts are allowed then @i[?on-or-off] is @code[:on] otherwise its
@code[:off].  @code[Make-true], @code[let], and @code[with] work with this so that
if one wants a goal to be executed as an atomic unit then this can be
expressed as @code[(with (interrupts :off) ...)].
@end defpredicate

@section Mutable Arrays
@subsection Introduction
LM-Prolog provides a completely logical interface to the arrays of
Zetalisp, defined in the file @code[LMP: EXPERIMENTAL; ARRAY].  A mutable
array is, just like a Zetalisp array, a (possibly very large) data
structure with direct access to its elements.  However, mutable arrays
are logical objects and so are not amenable to destructive updates.
LM-Prolog instead provides updating predicates which are not
destructive, but creates a new virtual array while keeping the old.
As much storage as possible is shared between the old and the new
virtual array, in particular they both point to the same Zetalisp
array which normally reflects the new virtual array.  The old virtual
array normally keeps a record of the difference between it and the new
virtual array.

The implementation is optimized so that in the best case array references and
updates are nearly as efficient as using them directly in Zetalisp.
More details on mutable arrays can be found in [Eriksson and Rayner 1984].

@subsection Array Predicates
@defpredicate prolog:is-array ?array ?array-description
This is true if @i[?array] satisfies the @i[?array-description].
Can be used to create arrays or to determine their properties.
An array description is a cons of @code[:array] and a list of keywords
paired with their values as follows:

@description
@item @code[(:dimensions . @i(list--of--integers))]
@mykindex :dimensions
Obligatory list of dimensions (Zetalisp supports up to 7 dimensions).

@item @code[(:type @i(type))]
@mykindex :type
Possible types are
@code[:normal, :1-bit, :2-bit, :4-bit, :8-bit, :16-bit, :string]
and @code[:fat-string].
Default is @code[:normal].
The other types are more efficient if the vast majority of values fit the
description.

@item @code[(:initial-value @i(value))]
@mykindex :initial-value
If an initial value is not given then each element becomes an independent
logical variable.
If the array is not @code[:normal] then the default initial value is 0.

@item @code[(:usage . @i(list--of--declarations))]
@mykindex :usage
The @i[list--of--declarations] may contain the flags 
@code[:deterministic], @code[:recent-only], and @code[:check].  
Or it may contain only the flag @code[:lisp-like].

@end description

@description
@kitem :deterministic
declares that old versions of the array
will never be backtracked to i.e. will never be used by alternative
computations,

@kitem :recent-only
declares that old versions of arrays are never 
referenced in the current computation,

@kitem :check
tells the system to signal an error if either of the above 
declarations are violated,

@kitem :lisp-like
means that the array will be used 
@code[:deterministic], @code[:recent-only],
and that indices are always ground and if the @code[:type] is not 
@code[:normal] that all the values will fit.

@end description

These @code[:usage] declarations can significantly improve the efficiency of 
programs using arrays.
@end defpredicate
@nopara
For example, the goal

@lisp
(prolog:is-array ?a 
  (:array (:dimensions 10) 
          (:initial-value 0)
          (:usage :recent-only :deterministic)))
@end lisp

unifies @code[?a] with a one-dimensional array 10 long, initialized to
0. It will be updated destructively, since old versions of it
are not going to be referenced neither by any subcomputations 
(@code[:recent-only]) nor by alternative computations (@code[:deterministic]).

@defpredicate prolog:array-element ?value +array &rest ?indices
The element of @i[+array] described by @i[?indices] is unified with
@i[?value].
If @i[?indices] contain variables then the predicate will search for
elements who unify with @i[?value].
This fails if the @i[?indices] are out of the range of the
@code[:dimensions] of the @i[?array].
@end defpredicate
@nopara
For example, the goal @code[(array-element 7 ?a 6)]
unifies the 6th element of @code[?a] with 7.

@defpredicate prolog:array-difference +old-array ?new-array (?new-value . ?indices)
This unifies @i[?new-array] with a new virtual array
which differs from @i[+old-array]
in that the element indicated by @i[?indices] is replaced by
@i[?new-value].

@end defpredicate
@nopara
For example, the goal @code[(array-difference ?a ?a-1 (not-x 6))]
creates a new virtual array @code[?a-1] which is equal to @code[?a] except
for its 6th element which has been changed to @code[not-x].

@defpredicate prolog:array-difference-old-real +old-array ?new-array (?new-value . ?indices)
This is equivalent to @code[array-difference]
but is more efficient if one expects to use @i[+old-array] more often than
@i[?new-array]. Otherwise use @code[array-difference].
@end defpredicate

@subsection Unification
Two mutable arrays unify if and only if they have the same @code[:type]
and the same @code[:dimensions] and their elements unify pairwise.

@section The Demo Facility
@setq demfac section-page
@defpredicate demos
The goal @code[(demos)] will put up a menu of several self-documenting
demos. 

@end defpredicate
@nopara
Currently, the list includes

@description
@item Peano arithmetic.
@cindex Peano arithmetic,
Predicates are defined for doing addition, 
multiplication, factorial, and the like of natural numbers.
Some of these predicates are presented in @ref[peano].

@item Ask about.
@cindex Ask about.
A predicate called @code[ask-about] is loaded which when 
applied to
a predicator defines that predicator so that it queries the user when it is
called unless the user has already answered the query.

@item Backtracking Turtle.
@cindex Backtracking Turtle.
A Logo-like turtle [Lieberman 1980]
is interfaced so that it reacts to a stream of commands
such as @code[(:forward 100.)], @code[(:right 90)], etc.
The commands are executed as elements of the stream become instantiated
and upon backtracking undo whatever graphics they performed.
This facility builds upon constraints (see @ref[stream], where a
stream programming example appears).

@item Grammar Kit.
@cindex Grammar Kit.
A facility for defining definite clause grammars 
[Pereira and Warren 1980]
and a small sample grammar.  This uses the ``ask about'' and
``backtracking turtle'' features.  It optionally provides a backtracking
graphical trace of its execution.  This kit is described in detail in
@ref[gram] and in [Kahn 1983b].

@item Alternative Top Levels.
@cindex Alternative Top Levels.
These alternative Prolog top levels are described in @ref[toplevel].

@item Eager Collections and some examples.
@cindex Eager Collections.
These parallel processing versions of 
@code[bag-of] and @code[set-of] are described in @ref[eager].

@item Constraints and some examples.
@cindex Constraints.
This control extension of Prolog is described 
in @ref[freeze].

@item Prime Numbers.
@cindex Prime Numbers.
A list of all prime numbers is generated with either
lazy or eager bags.  The sieve of Eratosthenes is modeled by infinite
bags and infinite operations on those bags.

@item Eight Queens.
@cindex Eight Queens.
A program to solve the puzzle of how one can place 8 queens on 
a chess board so that none can capture another. The solution uses constraints.
It optionally displays its states while running, using a similar stream-based
technique as in the Backtracking Turtle.

@item Knights Tour.
@cindex Knights Tour.
A program to solve the puzzle of how a knight can traverse a
chess board so that each square is visited exactly once.  It backtracks to
find more solutions and displays its states.

@item Cryptarithmetics.
@cindex Cryptarithmetics.
A program to solve the SEND + MORE = MONEY puzzle
where each letter stands for a unique digit.  It is written in terms of
constraints and optionally displays its states while running.

@item Concurrent Prolog with Bagel Simulator.
@cindex Concurrent Prolog.
@cindex Bagel Simulator.
An interpreter for the language
Concurrent Prolog [Shapiro 1983a, 1983b] is defined and several
example programs are run.
The allocation of processes on a virtual doubly twisted torus (a ``bagel'')
is displayed.

@item Benchmarks.
@cindex Benchmarks.
This runs LM-Prolog on a series of published benchmarks and
prints out the cpu and paging times involved.

@end description

@chapter Top Levels
@setq toplevel section-page

@section Editing
Normally, definitions of predicates are typed into Zwei buffers.
The normal mode for the buffers is @code[Lisp].
Predicate definitions can be intermixed with @code[defun]s, @code[defprop]s, 
or any other Lisp forms that you want to have in the same file.  
Prolog goals can be included if the predicator uses the
@code[:lisp-macro-name] @code[define-predicate] option.
Definitions can be compiled or evaluated in Zwei, 
@code[qc-file]d, @code[compile-file]d, or @code[load]ed, just as Lisp code.
The commands @meta[.] (edit definition),
@csh[A] (display argument list), and @csh[D] (show documentation) 
work for Prolog predicates as well as Lisp objects.

@section The Prolog Listener
Analogous to the Lisp Listener concept, interaction with LM-Prolog
normally takes place in a Prolog Listener.  By typing @system  (@greek[P]),
a Prolog Listener is created if necessary, and exposed.

A Prolog Listener accepts goals, displays answer substitutions
for the variables that occurred in the top-level query, and prompts the
user for rejection or acceptance.  See the Introduction for a sample dialog.

A Lisp forms may be entered if preceded by a single comma (,),
in which case the value(s) of the form are displayed, as in a Lisp Listener.
Thus, to get a Lisp toploop in a Prolog Listener, type

@lisp
,(break)
@end lisp

@nopara
Typing @resume reverts to Prolog.  In a Prolog Listener,
the @breakkey key invokes a recursive Prolog toploop.  See also the
@code[break] predicate.

@section Zlog Mode in Zmacs
It is possible to convert a Zmacs buffer into a Prolog Listener
by the command @metax[Zlog Mode].  In this mode, all interaction is saved
in the buffer, and the full editing capability of the Zmacs editor is
available.  In fact, it is completely analogous to @code[Ztop Mode].

@section Alternative Top Levels
Unlike Lisp for which there is wide-spread agreement that the top level should
be a ``read-eval-print'' loop there is less agreement about Prolog top levels.
The default LM-Prolog top level checks first if the predicate
@code[top-level-predication] is defined and if so passes the problem on to it.
This predicate should then interpret the query and print out results.
If the query is non-deterministic then upon rejection it will be reentered.
Several such top levels are predefined in LM-Prolog and to use them one just
adds the world they live in to the universe (using @code[add-world]).
They can also be selected using the @code[demos] predicate.

@deftoplevel :unique-answers-top-level
This top level differs only in that the same set of bindings is never printed
more than once even if there are redundant ways of achieving it.
@end deftoplevel
@nopara
For example, the goal
@code[(member ?x (a a b))] will display 
the binding @t[?X = A] and if that is rejected it will display @t[?X = B].

@deftoplevel :compute-ahead-top-level
This top level depends upon the experimental parallelism facilities described
in @ref[eager].
The idea is that in response to a top-level query the system creates an
independent process to compute all the solutions.
As soon as one solution is ready it is displayed and while the user inspects
the results more solutions are being sought.
If a new query is given then the old process is killed.
@end deftoplevel

@deftoplevel :parallel-prolog-top-level
This top level is very experimental and uses the experimental parallelism
facilities described in @ref[eager].
The query is solved by exploring all alternatives in parallel and by creating
separate processes for each goal in a conjunction.
@end deftoplevel
@nopara
This is useful for experimentation with parallelism but there has been little
attempt to bring this facility up to ``production'' standards.


@chapter User Extensibility of the System
@setq exten section-page
The predicate @code[lisp-value] is nearly always adequate for user extensions
to the system.
Such extensions require only a knowledge of Zetalisp.
Occasionally a control extension is desired.

@cindex intrinsic compilation
@cindex compilation, intrinsic
The supported way of extending the compiler is by using @dfn[intrinsic] compile
methods.  Whenever the compiler encounters goals for a predicate with
an intrinsic compile method, it will simply call 
the corresponding function whose responsibility it is to emit suitable code.

It should accept the following arguments:

@description
@item @i[cont1]
will contain compiled code for the remainder of the conjunction
in which the current goal occurs,

@item @i[cont2]
is only needed for the implementation of @code[cut].

the rest of the arguments should be the arguments of the predicate itself.
At compile time, these are terms which look just like the Lisp form
of the source representation except for variables which are represented
by locatives.

@end description

The intrinsic function should return two values: 
@enumerate
@item @i[cont1] augmented with code for the goal being compiled.
If the predicate corresponding to the intrinsic function
is deterministic, you could return @code[`(and ,code-for-this ,@i(cont1))].
If the predicate is to be non-deterministic, you will need a deeper 
understanding of the implementation. Loosely speaking, @i[cont1] will
have to be transformed to a continuation to be passed to the code for the
goal being compiled by the intrinsic function.

@item Just return @i[cont2]. There is only one intrinsic function that breaks
this convention: the one for @code[cut].
@end enumerate

@code[Lisp-predicate], @code[and], and @code[or] have 
intrinsic compile-methods, for instance.
@code[Define-predicate] takes special action for definitions with
intrinsic compile-methods, so that calls from within the same file will be
compiled correctly both with editor commands and with @code[compile] or
@code[qc-file].

For an example of a non-deterministic predicate with an intrinsic compile
method, consult the source code for @code[cases].
@code[Lisp-value] is an example of a deterministic predicate with an intrinsic
compile method. Its definition follows.

@lisp
(define-predicate lisp-value
  (:options (:compile-method :intrinsic compile-lisp-value)
            (:deterministic :always))
  @ii[;The clauses are needed so that @code(lisp-value) can be called directly]
  ((lisp-value ?value ?form) 
   (lisp-value ?value ?form))
  ((lisp-value ?value ?form ?mode) 
   (lisp-value ?value ?form ?mode)))
@end lisp

@lisp
(defun compile-lisp-value (cont1 cont2 value form &optional mode)
  (values `(and (%unify-term-with-term 
                   ,(constructor value t)
                   ,(compile-lisp-form form mode))
                ,cont1)
          cont2))
@end lisp

The system is designed to allow modular definitions of new data types
as Lisp Machine flavors.
This requires however, a good understanding of the implementation.
Illustrative examples can be found in the implementation of constraints
(see the file @code[LMP: EXPERIMENTAL; FREEZE]) and eager values
(see the file @code[LMP: EXPERIMENTAL; EAGER]).

@chapter Performance Considerations
@section LIPS Speed

@cindex benchmarks
@cindex LIPS speed
The file @code[LMP: LIBRARY; BENCHMARKS] contains some benchmark programs
that were published in [Warren 1977].  The programs are @code[reverse-30], a
naive @i[O(n@+(2))] reverse of a list 30 long; @code[qsort-50], 
a quicksort of a list 50 long; 
four differentiations; a palindrome of a list 25 long;
and a database query similar to the one described in @ref[dbquery].
The table below shows the timings in milliseconds
for compiled (LM-C) and interpreted (LM-I) predicates on a CADR or LM-2,
and for Prolog-10/20 on a KI10 (10-C and 10-I).
The LM-Prolog times are followed in parentheses by the timings without
micro-code support.
Note that the hardware
of a KI10 is much faster than that of a CADR.

The first benchmark involves 498 inferences,
suggesting a LIPS rate of around 12000 for compiled code and
2000 for interpreted. 

@settabs 5 @columns
@sp 1
@<@\@hfill LM-C@\@hfill LM-I@\@hfill 10-C@\@hfill 10-I@\@cr
@sp 1
@<reverse-30@\@hfill 43(134)@\@hfill 250(1170)@\@hfill 53.7@\@hfill 1160@\@cr
@<qsort-50@\@hfill 56(197)@\@hfill 644(2380)@\@hfill 75.0@\@hfill 1344@\@cr
@<d-times-10@\@hfill 7(16)@\@hfill 36(110)@\@hfill 3.0@\@hfill 76.2@\@cr
@<d-divide-10@\@hfill 8(18)@\@hfill 38(126)@\@hfill 2.9@\@hfill 84.4@\@cr
@<d-log-10@\@hfill 5(12)@\@hfill 21(77)@\@hfill 1.9@\@hfill 49.2@\@cr
@<d-op-10@\@hfill 5(12)@\@hfill 26(83)@\@hfill 2.2@\@hfill 63.7@\@cr
@<palin-25@\@hfill 37(167)@\@hfill 294(1250)@\@hfill 40.2@\@hfill 602@\@cr
@<query@\@hfill 404(1746)@\@hfill 1780(5858)@\@hfill 185.0@\@hfill 8888@\@cr

@section Costs of Various Features

In building large systems in LM-Prolog a user should be aware of what
various features cost so that they can be used in a rational way.
The following is a list of features and their costs.

@description
@item Assert.
Adding clauses to the interpretive definition of a predicate
takes about .02 seconds per clause depending on the complexity of the clauses.
Adding clauses to compiled definitions is the same, however, the real
cost of the assert is the first time the clause is used.
It is then that it is translated into Lisp and compiled into machine language.

@item Retract.
@setq retract section-page
Little effort has been put in to making clause retraction happen quickly.
A simple retraction takes about .01 seconds.

@item Occur check and circularity handling.
Putting on the occur check will typically slow down
interpretive LM-Prolog by about 30%   and compiled code by about 65%.
A circularity mode of @code[:handle] slows down the interpreter and compiled
code by 1% to 2%.

@item Lazy collections.
The cost of using lazy bags and sets depends very much on how they are used and
the size of the terms involved.
If the first argument is given, then several optimizations are performed.
For example, the goal @code[(cannot-prove ?predication)]
compiles equally
well if it is defined as @code[(lazy-bag-of () ? ?predication)]
or it is defined in terms of @code[cut].
In general, however, a lazy collection requires its own @dfn[trail] for
@cindex trail
backtracking.
In some cases it will need to run in its own @dfn[stack group].
@cindex stack group
All but the first argument may need to be copied.

@item Multiple worlds.
The system caches its search for the definition of a predicator in a universe.
If the universe is not changed then the overhead is minimal.
The cost of frequently changing the universe depends upon how many worlds are
in the universe and how often predicate definitions can be found in the first
few worlds.

@item Open compilation of predicates.
The speedup gained by open compilation is important if the open-coded
predicate is a meta-level predicate. Its arguments are then goals,
and with closed compilation, the compiler would compile them as terms
which is almost equivalent to letting them be interpreted at run time.
@cindex object-level predicate
@cindex predicate, object level
If the open-coded predicate is an ordinary @dfn[object-level] predicate,
the gain in speed corresponds to a couple of function calls and is
usually of minor importance, unless it is very heavily used.  One case 
where open-coding can be worthwhile
is when the predicate is an alias for another predicate, e.g. with a different
order of its arguments.
Open coding can of course lead to increases in the size of the compiled code.

@item Garbage Collection.
The system does all its temporary consing in a dedicated area with
high ``volatility'' for efficient garbage collection.  Definitions are
kept in their own area which is the value of the Lisp variable
@code[prolog:*structure-area*].


@item Non determinacy.
Prolog's inherent non-determinacy occasionally causes the stack to grow very
large.  In most Prolog programs, however, the non-determinacy is just
artificial and the program is really deterministic.
If this is the case, then the problem of stack growth is easily cured by
declaring predicates deterministic and/or using @code[cut].
Declaring a predicate @i[P] deterministic ensures that the stack depth at
@i[P]'s call ``port'' equals the stack depth at @i[P]'s exit ``port''.
This is not the case for general predicates.
The compiler and the interpreter both capitalize on such declarations.
Using @code[cut] immediately before tail-recursive calls in @i[P] ensures that 
the stack does not grow at all when @i[P] recurses. Note that @code[cut]
has no declarative semantics, but can be used (with care) for cutting out 
unwanted deduction paths and for preventing excessive stack growth.

@end description

@section Time And Meter Predicates

@defpredicate time-and-print +predication
This satisfies @i[+predication] and prints out timing information 
when a proof is found.
It  backtracks to find the next proof of @i[+predication], and finally to print
out timings of its eventual failure. 
@end defpredicate
@nopara
E.g. if @i[+predication] succeeds @i[n] times,
@code[time-and-print] succeeds @i[n+1] times.

@defpredicate time (?cpu ?disk) +predication
This predicate unifies @i[?cpu] and @i[?disk] with the cpu and disk
time consumption for finding a proof of @i[+predication].  It backtracks in the
same way as @code[time-and-print].  @i[?Cpu] and @i[?disk] will not
contain accumulated values but just the time to find the first proof, the
next proof, or to ultimately fail.
@end defpredicate
@nopara
Occasionally, @i[?cpu] will be a negative number.  This is because the 
accuracy of metering disk time consumption is rather poor and influences the
computation of cpu time consumption. It may also happen that the machine's
microsecond clock wraps around.

@defpredicate meter +predication
This predicate satisfies @i[+predication] with metering switched on.
A meter
analysis is written into Zwei buffer @code[Meter Information].  Again, this
predicate backtracks to meter the next solution or ultimate failure.
@end defpredicate
@nopara
The Lisp Machine's metering facility seems somewhat fragile.
It is recommended that one meter only computations that take a second or less.

@appendix Translations from Prolog-10/20

This Appendix gives @b[approximate] translations from other Prolog
systems such as Prolog-10/20 into LM-Prolog.  LM-Prolog comes with a
translator program that does a source-to-source translation from
Prolog-10/20 to LM-Prolog.  Due to major syntactic and semantic
differences however, there is no guarantee that the translation is
correct.  The translator can be found in the file @code[LMP: SYSTEM;
LISPIFY PL].

@sf
@settabs 2 @columns
@sp 1
@<@r[Dec-10 Prolog]@\@r[LM-Prolog]@cr
@sp 1
@<consult(F)@\(load F)@cr
@<reconsult(F)@\(load F)@cr
@<read(X)@\(read X)@cr
@<write(X)@\(princ X)@cr
@<writeq(X)@\(prin1 X)@cr
@<print(X)@\(print X)@cr
@<nl@\(tyo #\return)@cr
@<display(X)@\(format t ``~n'' X)@cr
@<ttynl@\(tyo #\return t)@cr
@<(*)ttyget0(N)@\(tyi N t)@cr
@<(*)ttyput(N)@\(tyo N t)@cr
@<(*)get0(N)@\(tyi N)@cr
@<(*)put(N)@\(tyo N)@cr
@<Z is X@\(lisp-value Z X)@cr
@<X+Y@\(+ X Y)  @ii[;These are defined in arithmetic contexts.]@cr
@<X-Y@\(- X Y)@cr
@<X*Y@\(* X Y)@cr
@<X/Y@\(// X Y)@cr
@<X mod Y@\(\ X Y)@cr
@<XY@\(logand X Y)@cr
@<XY@\(logior X Y)@cr
@<\X@\(lognot X)@cr
@<X@BllY@\(lsh X Y)@cr
@<X@BggY@\(lsh X (- Y))@cr
@<P,Q@\(and P Q)@cr
@<P;Q@\(or P Q)@cr
@<true@\(true)@cr
@<X=Y@\(= X Y)@cr
@<length(L,N)@\(length N L)@cr
@<!@\(cut)@cr
@<\+(P)@\(cannot-prove P)@cr
@<P1Q1;P2Q2 ...@\(cases ((prove-once P1) Q1) ((prove-once P2) Q2) ...)@cr
@<repeat@\(repeat)@cr
@<fail@\(fail)@cr
@<X=:=Y@\(lisp-predicate (= 'X 'Y))@cr
@<X=\=Y@\(lisp-predicate ( 'X 'Y))@cr
@<X==Y@\(identical X Y)@cr
@<X\==Y@\(not-identical X Y)@cr
@<var(X)@\(variable X)@cr
@<nonvar(X)@\(not-variable X)@cr
@<atom(X)@\(symbol X)@cr
@<atomic(X)@\(atomic X)@cr
@<number(X)@\(number X)@cr
@<(*)functor(T,F,N)@\(and (= T (F . U)) (length N U))@cr
@<arg(I,T,X)@\(nth X I T)@cr
@<X=..Y@\(= X Y)@cr
@<(*)name(X,L)@\(exploden L X)@cr
@<call(X)@\(call X)@cr
@<assert(Clause)@\(assert Clause)@cr
@<asserta(Clause)@\(asserta Clause)@cr
@<assertz(Clause)@\(assertz Clause)@cr
@<clause(Head,Body)@\(clause (Head . Body))@cr
@<retract(Clause)@\(retract Clause)@cr
@<listing(P)@\(print-definition P)@cr
@<listing@\(bag-of ? ? (print-definition ?))@cr
@<setof(X,P,S)@\(quantified-set-of S Q X P)@cr
@<@\@ @ii[;where @l(Q) is universally quantified variables in P]@cr
@<bagof(X,P,S)@\(quantified-bag-of S Q X P)@cr
@<@\@ @ii[;where @l(Q) is universally quantified variables in P]@cr
@<compare(Op,T1,T2)@\(compare Op T1 T2)@cr
@<abolish(N,A)@\(retract-all N)@cr
@<(*)current_predicate(N,F)@\(predicator N)@cr
@<compile(Files)@\(map compile-file Files)@cr
@<incore(Goal)@\(call Goal)@cr
@sp 1

@rm

Dec-10/20 Prolog predicates marked with (*) above may be machine dependent.

Most Dec-10/20 Prolog predicates that are omitted from this table
are purely environmental and simply do not map well to the Lisp Machine.

Debugging facilities are described in @ref[debug].

Input-output is performed on streams on the Lisp Machine. There are no
equivalent objects in Dec-10/20 Prolog. Predicates for
this are described in @ref[io].  The predicate @l[open-file], for
example, subsumes @l[see, seeing, seen, tell, telling], and @l[told]
of Dec-10/20 Prolog.

LM-Prolog's predicate @l[sort] takes an extra argument, a predicate
defining the ordering.

``Public'' declarations are not supported. The ZetaLisp package
facility is a more structured way of maintaining multiple name spaces.

Support for defining definite clause grammars is built into Dec-10/20
Prolog [Pereira and Warren 1980].  The LM-Prolog syntax of the grammar
rules is quite different from Dec-10/20 Prolog. The grammar kit
includes a facilities for obtaining a dynamic backtracking graphical
trace of the parse.  See the documentation in @ref[gram] and in [Kahn
1983b] for more details.

@appendix Example Compilations
This chapter exemplifies how LM-Prolog's compiler works.  It features
compiled unification, tail recursion optimization, automatic detection
of determinacy, and more.
Examples show how Prolog predicates are compiled into Lisp procedures and
further into Lisp Machine code (CADR system 93).

@section Introduction

@cindex backtracking
@cindex continuation
@cindex success continuation
The basic control strategy of LM-Prolog is called
``success continuations'' [Carlsson 1984].  
A success continuation is a function that is passed
as an extra argument to predicates.  
When a predicate has succeeded in proving its task, 
it invokes the continuation rather than returning.  If the continuation 
returns non-@l[nil], the predicate goes on to find more proofs, 
and so backtracking is achieved.

This scheme is used in compilation in the following way: a body of a clause
with, for example, three predications @l[((p ...) (q ...) (r ...))]
compiles to a structure like

@lisp
(funcall (entrypoint 'p)
         (continuation
           (funcall (entrypoint 'q)
                    (continuation
                      (funcall (entrypoint 'r)
                               original-continuation ...))
                               ...))
         ...)
@end lisp


@noindent
where @l[continuation] is a macro 
that expands its arguments to a function
of zero arguments, and @l[entrypoint] macro-expands to a
cached search for the compiled code of its argument.

An important special case is when the predicate is deterministic,
in which case the compiler often arranges that the continuation is
the function @l[true]. This makes it possible to compile a conjunction
of predications sequentially rather than as a nested construct. If all 
three predicates in the example above were deterministic, the compiled
code would look like

@lisp
(and (funcall (entrypoint 'p) (continuation (true)) ...)
     (funcall (entrypoint 'q) (continuation (true)) ...)
     (funcall (entrypoint 'r) (continuation (true)) ...)
     (invoke original-continuation))
@end lisp


The latter solution clearly consumes less stack space and constructs no
continuations.

The compiler uses argument list declarations and knowledge about previously
defined predicates to compute determinacy of static predicates. 
@c Note that generality rather than
@c maximum efficiency has been the important consideration in many system
@c predicates like @l[append] or @l[member] so they are not deterministic.
@c You may want to write specialized versions of them for 
@c specific argument list patterns, if you have a time-critical application.
To make best use of the determinacy optimization, definitions of
deterministic predicates should precede their uses.  This may not be possible
due to mutual recursion. 

@c You can use the Lisp form @l[declare]
@c to inform the compiler that certain predicates are deterministic.
@c The format is @l[(declare (deterministic p1 p2 ...))]
@c The declaration remains in effect during a @l[Compile Buffer/Region] editor
@c command, or during a @l[qc-file] operation and should of course be given 
@c before the first use of the declared predicates.

We give here a sample compilation of a ``pure'' LM-Prolog program:
the predicate @l[concatenate] which appends two lists,
and the predicate @l[naive-reverse] which reverses a list using
@l[concatenate].  Lisp is used as target language.  The compiled code
contains several macros and microcoded instructions:

@description
@item deffun
Macro that transforms tail recursive calls to iteration.

@item %unify-term-with-template
Instruction to unify a term with a ``template structure''.

@item %unify-term-with-term
Instruction to unify two terms.

@item %cell0
Instruction to allocate a value cell.

@item %prolog-list
Instruction to allocate a list.

@item %prolog-list*
Instruction to allocate a ``dotted list''.

@item %dereference
Instruction to dereference a variable.

@item %reference
Instruction to convert a variable before storing
a reference to it in a list.

@item %invoke
Instruction to invoke a continuation.

@item %current-entrypoint
Instruction that does a cached search for
the current definition of a predicate.

@end description

@section Concatenate
@lisp
(define-predicate concatenate
  (:options (:world :benchmarks) 
            (:execution :compiled) 
            (:argument-list (+ + -)))
  ((concatenate (?first . ?rest) ?back (?first . ?all-but-first))
   (concatenate ?rest ?back ?all-but-first))
  ((concatenate () ?back ?back)))
@end lisp

Mapping of variables:

@lisp
; ?first  (car g0634)
; ?rest  (cdr g0634)
; ?back  ?variable1
; ?all-but-first  ?variable5
@end lisp

Compiled code:

@lisp
(deffun concatenate-in-/:benchmarks-prover
        (.continuation. ?variable0 ?variable1 ?variable2
         &aux ?variable5)
  (let ((g0364 (%dereference ?variable0)))
    (cond ((and (and (consp g0364) t t))
           (and (%unify-term-with-template ?variable2
                                           '(14000002 . 4000000))
                (funcall (entrypoint 'concatenate)
                         (continuation 
                           (invoke .continuation.))
                         (cdr g0364)
                         ?variable1
                         ?variable5)))
          ((and (eq g0364 nil))
           (and (unify (%dereference ?variable1)
                       (%dereference ?variable2))
                (invoke .continuation.))))))
@end lisp

which further compiles to

@lisp
20 DEREFERENCE D-PDL ARG|1  ;?VARIABLE0
21 POP LOCAL|1
22 BR-ATOM 36
23 MOVE D-PDL ARG|3       ;?VARIABLE2
24 CAR D-PDL LOCAL|1
25 POP LOCAL|2
26 UNIFY-WITH-TEMPLATE D-PDL FEF|6       ;'(14000002 . 4000000)
27 BR-NIL-POP 34
30 CDR D-PDL LOCAL|1
31 MOVE D-PDL LOCAL|0     ;?VARIABLE5
32 POP ARG|3                  ;?VARIABLE2
33 BR 21
34 MOVE D-RETURN PDL-POP
35 MOVE D-IGNORE LOCAL|1
36 BR-NOT-NIL 46
37 DEREFERENCE D-PDL ARG|2  ;?VARIABLE1
40 DEREFERENCE D-PDL ARG|3  ;?VARIABLE2
41 UNIFY-WITH-TERM D-PDL PDL-POP
42 BR-NIL-POP 45
43 MOVE D-PDL ARG|0       ;.CONTINUATION.
44 (MISC) %INVOKE D-RETURN
45 MOVE D-RETURN PDL-POP
46 (MISC) FALSE D-RETURN
@end lisp

Instruction 26 references the template for @l[(?first . ?all-but-first)].

Tail recursion was transformed to iteration (instructions 30--33). In
performing that transformation, the compiler noted that dereferencing
@l[?variable0] is only needed initially since @l[cdr] 
(instruction 30) subsumes dereferencing.

Note the efficient compilation of the matching of @l[?variable0]
(instructions 22 and 36).

@section Naive-Reverse
@lisp
(define-predicate naive-reverse
  (:options (:world :benchmarks) 
            (:execution :compiled) 
            (:argument-list (+ -)))
  ((naive-reverse (?first . ?rest) ?reverse)
   (naive-reverse ?rest ?rest-reversed)
   (concatenate ?rest-reversed (?first) ?reverse))
  ((naive-reverse () ())))
@end lisp

Mapping of variables:

@lisp
; ?first  (car g0366)
; ?rest  (cdr g0366)
; ?reverse  ?variable1
; ?rest-reversed  ?variable4
@end lisp

Compiled code:

@lisp
(deffun naive-reverse-in-/:benchmarks-prover
        (.continuation. ?variable0 ?variable1 &aux ?variable4)
  (let ((g0366 (%dereference ?variable0)))
    (cond ((and (and (consp g0366) t t))
           (and
            (progn
             (setq ?variable4 (%cell0))
                   (and (funcall (entrypoint 'naive-reverse)
                                 (continuation (true))
                                 (cdr g0366)
                                 ?variable4)
                        (funcall (entrypoint 'concatenate)
                                 (continuation 
                                   (invoke .continuation.))
                                 ?variable4
                                 (prolog-list (car g0366))
                                 ?variable1)))))
          ((and (eq g0366 nil))
           (and (unify (%dereference ?variable1) nil)
                (invoke .continuation.))))))
@end lisp

which further compiles to:

@lisp
30 DEREFERENCE D-PDL ARG|1  ;?VARIABLE0
31 POP LOCAL|1
32 BR-ATOM 57
33 CELL0 D-PDL
34 POP LOCAL|0            ;?VARIABLE4
35 MOVE D-PDL FEF|11      ;'NAIVE-REVERSE
36 CURRENT-ENTRYPOINT D-PDL FEF|12      ;'#<DTP-LOCATIVE 14304554>
37 CALL D-PDL PDL-POP
40 MOVE D-PDL FEF|8       ;'(TRUE)
41 CDR D-PDL LOCAL|1
42 MOVE D-LAST LOCAL|0    ;?VARIABLE4
43 BR-NIL-POP 55
44 MOVE D-PDL FEF|9       ;'CONCATENATE
45 CURRENT-ENTRYPOINT D-PDL FEF|10      ;'#<DTP-LOCATIVE 14304536>
46 CALL D-RETURN PDL-POP
47 MOVE D-PDL ARG|0       ;.CONTINUATION.
50 MOVE D-PDL LOCAL|0     ;?VARIABLE4
51 CAR D-PDL LOCAL|1
52 REFERENCE D-PDL PDL-POP
53 PROLOG-LIST D-PDL, 1 long
54 MOVE D-LAST ARG|2      ;?VARIABLE1
55 MOVE D-RETURN PDL-POP
56 MOVE D-IGNORE LOCAL|1
57 BR-NOT-NIL 66
60 DEREFERENCE D-PDL ARG|2  ;?VARIABLE1
61 UNIFY-WITH-TERM D-PDL 'NIL
62 BR-NIL-POP 65
63 MOVE D-PDL ARG|0       ;.CONTINUATION.
64 (MISC) %INVOKE D-RETURN
65 MOVE D-RETURN PDL-POP
66 (MISC) FALSE D-RETURN
@end lisp

Aided by the given call-pattern, the compiler noted that @l[concatenate]
is deterministic, and so it compiled a deterministic call from 
@l[naive-reverse] to @l[concatenate]. @l[(continuation (true))]
means there will only be one solution.

Tail recursion optimization is not applicable here.
One variable has to be initialized by an explicitly allocated value cell
(instructions 33--34).
Instructions 35--36 and 44--45 show examples of the @l[current-entrypoint] 
instruction. The locative in instruction 45 is a pointer to the
definitions of @l[concatenate]. The pointer is installed at load time.

In the call to @l[concatenate], a list 1 long is constructed by our
@l[prolog-list] instruction (instructions 51--53).

@section Discussion

The Naive Reverse program reverses a list 30 long in about .043 seconds on a
CADR.

Since continuations are always passed as arguments to predicates,
they could probably be allocated as stack lists.  There could be
problems, however, with stack lists being passed to lazy collections
or problems with tail recursive predicates.  For now, all continuations
are allocated in LM-Prolog's temporary area.

@cindex scope
@cindex lexical scoping
New ZetaLisp versions support lexical scoping of variables. With lexical
scoping, continuations are not represented as lists but as compiled
functions without arguments which refer to variables in lexically
enclosing functions.

@appendix Release Notes and Revision History

@heading Release Notes for Lambda Release 2.0

The predicates @code[compile], @code[compile-file], and
@code[prolog:condition-protect] were added.

The @code[step] debugging facility was added.

The @code[:save-compiled-code] and @code[:execution-mode]
@code[define-predicate] options no longer exist.  Predicates are
compiled if seen in a ``compiler context'', i.e. in a compile
definition, region, buffer, or file operation.  Indexing on multiple
keys now take all keys into account.  Adding interpreted clauses
was sped up by an order of magnitude.

The backtracking graphics primitives were rewritten to use
stream-based primitives rather than Lisp calls.

@heading Release Notes for Lambda Release 3.0

The @code[let], @code[write], @code[princ], @code[prin1],
@code[prin1-then-space], @code[print],
@code[undefined-predicate-mode], @code[break], @code[error], and
@code[lambda] predicates were added.  The @code[print] predicate was
renamed to @code[pprint].  The @code[first], @code[second],
@code[third], @code[fourth], and @code[rest] predicates and
@code[:lazy-set-top-level] toplevel were not useful and no longer
exist.

The @code[quantified-bag-of], @code[quantified-set-of], @code[repeat],
@code[exploden], @code[asserta], @code[assertz],
@code[standard-order], and @code[compare] predicates have become part
of the standard system.

Lambda expressions were added, as an alternative to @code[assert], and
can be used as predicators.

The semantics of the Lisp interface was changed.  The conversion
keyword @code[:copy] now denotes that the argument has to be ground.
An omitted keyword now denotes no conversion.  Programs that rely on
the previous representation of source variables as symbols have to be
updated.  The @code[delayed-value-mode] and @code[global-read]
predicates no longer exist.

Rest arguments cannot cause potential dangling stack pointers any more.
Temporary areas are not used any more.  Compiler problems with lazy
and eager collections and @code[unwind-protect] have been corrected.
Problems with the interaction between interpreted code, the step
facility, and @code[define-predicate] have been corrected.

Prolog Listeners were introduced.  Lisp Listeners do not accept Prolog
statements now.  A Prolog Listener under Zmacs is available by the
command @metax[Zlog Mode].

The @code[A], @code[B], @code[G], @code[M], @code[P], and @code[R]
commands were added to the trace package.

The @code[prolog:freeze] package now ensures that the frozen goal is
woken at most once.  The predicate @code[prolog:combine-constraints]
was renamed in the documentation to its real name,
@code[prolog:*combine-constraints*].

The predicate @code[prolog:is] was renamed to @code[prolog:is-array].

@appendix References
@begingroup
@tolerance 2000
@parskip 3mm
@parindent 0mm

[Byrd 1980] Byrd L.,
``Understanding the Control Flow of PROLOG Programs'', in 
@i[Logic Programming Workshop], ed. T@"arnlund S.-@AA.,
Debrecen, Hungary, July 1980.

[Carlsson 1984] Carlsson M.,
``On Implementing Prolog in Functional Programming'',
UPMAIL Technical Report No. 5B,
Computing Science Department, Uppsala University, Sweden, and in 
@i[Proc. 1984 International Symposium on Logic Programming],
Atlantic City NJ, and in
@i[New Generation Computing], vol. 2 no. 4 pp. 347--360,
Ohmsha, Ltd. and Springer-Verlag, Tokyo.

[Clark and McCabe 1982a] Clark K., McCabe F.,
``IC-Prolog --- Language Features'', in 
@i[Logic Programming], eds. Clark K., T@"arnlund S.-@AA., Academic Press, 
1982.

[Colmerauer 1982a] Colmerauer A.,
``PROLOG and Infinite Trees'', in 
@i[Logic Programming], eds. Clark K., T@"arnlund S.-@AA., Academic Press, 
1982.

[Colmerauer 1982b] Colmerauer A.,
``PROLOG II Manuel de R@'eference et Mod@`ele Th@'eorique'',
@i[Proc. Prolog Programming Environments Workshop], 
Link@"oping, Sweden, March 1982.

[Eriksson and Rayner 1984] Eriksson L.-H., Rayner M.,
``Incorporating Mutable Arrays Into Logic Programming'',
@i[Proc. Second International Logic Programming Conference],
Uppsala, Sweden, July 1984.

[Furukawa @i[et al.@/] 1982] Furukawa K., Nitta K., Matsumoto Y.,
``Prolog Interpreter Based on Concurrent Programming'',
@i[Proc. First Logic Programming Conference], 
Marseille, France, September 1982.

[Greenblatt 1974] Greenblatt R.,
``The Lisp Machine'',
MIT AI Laboratory Working Paper 79, November 1974.

[Hansson @i[et al.@/] 1982] Hansson A., Haridi S., T@"arnlund S-@AA.,
``Properties of a Logic Programming Language'', in 
@i[Logic Programming], eds. Clark K., T@"arnlund S.-@AA., Academic Press, 
1982.

[Iverson 1962] Iverson K., @i[A Programming Language], Wiley, New York, 1962.

[Kahn 1983a] Kahn K.,
``A Primitive for the Control of Logic Programs'',
UPMAIL Technical Report No. 16, Computing Science Department,
University of Uppsala, Sweden, and in
@i[Proc. 1984 International Symposium on Logic Programming],
Atlantic City, 1984.

[Kahn 1983b] Kahn K.,
``A Grammar Kit in Prolog'',
UPMAIL Technical Report No. 14, Computing Science Department,
University of Uppsala, Sweden, and in 
@i[New Horizons in Educational Computing],
ed. Yazdani M., Ellis Horwood Ltd., Chichester, Great Britain, 1983,
and in
@i[Proc. AISB Easter Conference on Artificial Intelligence and Education],
April 1983.

[Kahn and Carlsson 1984] Kahn K., Carlsson M.,
``The Compilation of Prolog Programs without the Use of a Prolog
Compiler'', @i[Proc. International Conference on Fifth Generation
Computer Systems], Tokyo, 1984.

[Kowalski 1974] Kowalski R.,
``Predicate Logic as Programming Language'',
@i[Proc. IFIP 1974], Elsevier North Holland, Amsterdam, pp. 569-574.

[Lieberman 1980] Lieberman H.,
``Logo Turtle Graphics for the Lisp Machine'',
MIT AI Laboratory Working Paper 214, May 1981.

[Lloyd 1984] Lloyd J.W.,
@i[Foundations of Logic Programming],
Springer-Verlag, 1984.

[Moon @i[et al.@/] 1983] Moon D., Stallman R. M., Weinreb D.,
``Lisp Machine Manual'', MIT AI Laboratory, January 1983.

[Nakashima 1982] Nakashima H.,
``Prolog/KR Language Features'',
@i[Proc. First Logic Programming Conference], 
Marseille, France, September 1982.

[Pereira @i[et al.@/] 1979] Pereira L., Byrd L., Pereira F., Warren D.H.D.,
``User's Guide to DECsystem-10 Prolog'', DAI Occasional Paper 15,
Department of Artificial Intelligence, University of Edinburgh,
Edinburgh, Scotland, 1979.

[Pereira and Warren 1980] Pereira F.C.N., Warren D.H.D.,
``Definite Clause Grammars for Language Analysis --- A Survey of the 
Formalism and a Comparison with Augmented Transition Networks'',
@i[Artificial Intelligence] Vol. 13, pp. 231-278, 1980.

[Robinson 1965] Robinson J.,
``A Machine-oriented Logic Based on the Resolution Principle'',
@i[Journal of the ACM], Vol. 12, no. 1, January 1965.

[Shapiro 1983a] Shapiro E.,
``A Subset of Concurrent Prolog and Its Interpreter'',
ICOT Technical Report, TR-003, ICOT, Tokyo, 1983.

[Shapiro 1983b] Shapiro E.,
``Lecture Notes on the Bagel: A Systolic Concurrent Prolog Machine'',
ICOT Technical Memorandum, TM-0031, ICOT, Tokyo, 1983.

[Roussel 1975] Roussel P.,
``Prolog: Manuel de R@'eference et d'Utilisation'',
Groupe d'Intelligence Artificielle, Universit@'e d'Aix-Marseille,
Luminy, France, 1975.

[Warren 1977] Warren D.,
``Implementing Prolog --- compiling predicate logic programs'',
Department of Artificial Intelligence, University of Edinburgh,
D.A.I. Research Report Nos. 39--40, May 1977.

@endgroup

@unnumbered Concept Index
@printindex cp
@unnumbered Variable Index
@printindex vr
@unnumbered Function Index
@printindex fn
@unnumbered Keyword Index
@printindex kw
@unnumbered World Index
@printindex wo
@unnumbered Predicate Index
@printindex pr

@contents
@bye
